\chapter{Conclusão} 
\label{Cap:conclusao}

Este trabalho apresentou uma nova proposta de compressão de dados sísmicos,
valendo-se da característica heteroscedásticas de tais dados. O modelo
estatístico ARIMA-GARCH foi usado na primeira parte de compressão com o objetivo
de reduzir a variância do dado original, descorrelacioná-lo e reduzir sua
entropia.

Comparando-se com modelos de predição linear já utilizados atualmente, o método
proposto foi capaz de reduzir a entropia do dado original na maioria dos testes.
Contudo, não conseguiu descorrelacioná-lo. Isso fez com que a razão de
compressão obtida não fosse satisfatória como se imaginava, embora tenha sido
superior na maioria dos testes rodados.

Vinte metodologias de testes foram rodadas. Percebeu-se que a utilização do
modelo ARMA torna-se suficiente, já que dados sísmicos são estacionários. Além
disso, é sempre necessário estimar os parâmetros do modelo. Estimar parâmetros
fixos para evitar o seu envio ao decodificador não trouxe um bom conjunto de
resíduo.

Ao se analisar o dado com relação à natureza, esperava-se que dados pós-stack
tivessem resultados superiores a pré-stack. Isso porque dados pós-stack possuem
menos ruído. Isso não ocorreu, dados pré e pós-stack apresentaram resultado
estatisticamente similares.
