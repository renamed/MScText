\chapter{A Geofísica e o Método Sísmico}
\label{Capitulo:sismicadereflexao}

A~\emph{Geofísica} trata do estudo do planeta Terra através de
métodos físicos. Mais especificamente, o termo é comumente utilizado para
denotar a física aplicada ao estudo do planeta, excetuando-se a hidrosfera e a
atmosfera. Uma das subáreas em que a geofísica pode ser particionada é a
Geofísica de Exploração que se preocupa com a aplicação de técnicas de
geofísica aos problemas de exploração de gás/petróleo, água e alguns
minerais~\citep{Livro:telford1990}.

A Terra é formada pela crosta, manto e núcleo~\citep{Livro:crostamantonucleo1,
Livro:crostamantonucleo2}. A crosta terrestre é formada por um conjunto de
camadas que foram se depositando uma sobre a outra ao longo das diversas eras
geológicas. Cada uma dessas
camadas é denominada~\emph{subsuperfície}, a área de contato entre duas
subsuperfícies é chamada de~\emph{interface}.

Para que um geofísico possa determinar a possibilidade de se encontrar petróleo,
ou qualquer outro tipo de mineral, em uma dada região, é necessário que ele
disponha de um mapa das subsuperfícies. Esse mapa é uma espécie de radiografia
do subsolo~\citep{Livro:racionalenergia} daquela localidade. Existem diversos
métodos de prospecção de subsuperfícies, os quais são chamados de~\emph{métodos
geofísicos}.

Cada uma das subsuperfícies é formada por materiais diferentes; argila, areia,
sal, rocha etc. e, por isso, possui características físicas distintas;
resistividade elétrica, permeabilidade magnética, densidade, dentre outras.
Essas características são utilizadas pelos métodos
geofísicos para criar o mapa de subsuperfícies da região. Dentre os
métodos geofísicos existentes, destacam-se os~\emph{métodos sísmicos}, que se
utilizam de~\emph{ondas sísmicas}. Uma onda sísmica é uma perturbação mecânica
do meio que se propaga sem deslocamento de material, apenas de energia. Como
qualquer onda, suas principais características são frequência, comprimento de
onda, amplitude e fase~\citep{Dissertacao:puc2}.

O método sísmico é o mais utilizado atualmente~\citep{Livro:matlab,
Livro:yilmaz_vol_i} para fins de prospecção de petróleo. Ele compreende três
estágios, a aquisição, o processamento e a interpretação dos
dados~\citep{Livro:yilmaz_vol_i}. Neste trabalho, o processamento e a
interpretação dos dados não serão contemplados. A aquisição de dados para o
método sísmico pode ocorrer de forma marítima ou
terrestre~\citep{Livro:yilmaz_vol_i}. As duas abordagens são parecidas, sendo
a principal diferença os aparelhos utilizados, mas convergentes em relação à
metodologia.

O método sísmico necessita de ondas sísmicas que são gerados por um componente
chamado~\emph{fonte}. Um~\emph{tiro} é uma perturbação mecânica gerada
artificialmente pela fonte. Durante uma aquisição, vários tiros são efetuados,
em geral com intervalos de tempo ou distâncias constantes. As ondas formadas
pelos tiros viajam pelas subsuperfícies do terreno e, eventualmente, voltam para
a superfície (esse fenômeno será abordado mais adiante). As ondas que regressam
são coletadas por~\emph{geofones}, componentes que captam a energia mecânica no
solo e a transformam em energia elétrica. A distância entre a fonte e um geofone
é chamada de~\emph{offset}.

As ondas geradas pela fonte penetram a crosta terrestre, atravessando diferentes
subsuperfícies. Essas subsuperfícies apresentam características acústicas
distintas e, por isso, as ondas sofrem diversos fenômenos físicos. Para a
geofísica um dos fenômenos mais importantes é a~\emph{reflexão}. Parte da
energia contida em uma onda será refletida na interface entre duas camadas
geológicas e o restante seguirá seu caminho dentre as diversas subsuperfícies
existentes.

Esse processo continuará até que a onda perca sua energia. A energia da onda que
é refletida será captada por um geofone. Este mede a variação de
pressão na superfície em um~\emph{traço sísmico}. Um traço sísmico representa,
portanto, a variação da pressão em um geofone gravada em função do
tempo~\citep{Livro:matlab}. Um conjunto de traços sísmicos é chamado
de~\emph{sismograma}. Para cada tiro dado na fase de aquisição é gerado um
sismograma. Uma vez que os dados sísmicos foram obtidos, estes são transportados
em fitas magnéticas, ou HDs, até o centro de
processamento~\citep{Livro:yilmaz_vol_i}. Um formato de arquivo pré-definido
chamado SEG-Y é tipicamente utilizado. 

\section{O Formato SEG-Y}
\label{subsec:segy}

O SEG-Y é um dos diversos formatos de arquivos existentes para o armazenamento
de dados sísmicos~\citep{Manual:segy}, sendo este o formato requisitado pela
Agência Nacional do Petróleo para que cópias de aquisições de dados sísmicos em
solo brasileiro lhe sejam enviadas~\citep{Manual:anp}.

O SEG-Y utiliza o formato EBCDIC (\emph{Extended Binary Coded Digital
Interchange Character}), proposto pela IBM e relativamente bem utilizado quando
o padrão SEG-Y foi desenvolvido. Apesar do EBCDIC ter sido amplamente
substituído pelo padrão ASCII (\emph{American Standard Code for Information
Interchange}), o SEG-Y ainda o utiliza por razões de compatibilidade.

O SEG-Y possui um cabeçalho de~\emph{exatamente} 3.200 bytes a ser preenchido
com o formato EBCDIC. Isso significa que mesmo que nenhuma informação
seja posta nesse cabeçalho é necessário que 3.200 bytes do arquivo sejam
alocados. Cada arquivo SEG-Y possui um cabeçalho com informações sobre a
aquisição, como o seu local, a distância entre dois tiros consecutivos etc.
Informações nesse cabeçalho são escritas em 40 linhas de 80 caracteres cada uma.
Como no formato EBCDIC cada caractere ocupa 8 bits (1 byte), tem-se $80 \times
40 \times 1 = 3.200$ bytes~\citep{Manual:segy}.

O próximo campo do arquivo SEG-Y é um cabeçalho binário de~\emph{exatamente} 400
bytes preenchidos com inteiros de 2 bytes ($200$ inteiros $\times~2$) ou de 4
bytes ($100$ inteiros $\times~4$). Há apenas um cabeçalho binário por arquivo
SEG-Y e ele contém dados referentes à disposição dos geofones, sistema de
medição de distâncias (metros, pés etc.), formato dos dados do arquivo (ponto
flutuante de 32 bits, inteiro de 16 ou 32 bits etc) dentre outras informações.
Seu preenchimento não é obrigatório, embora seja extremamente
recomendável~\citep{Manual:segy}.

Em seguida, um número opcional de cabeçalhos chamados de~\emph{Extended Textual
File Header} de tamanhos fixos de 3.200 bytes são apresentados. O objetivo de
tais cabeçalhos é proporcionar um espaço adicional para informações necessárias
sobre o arquivo SEG-Y de maneira flexível, mas bem estruturada. O tipo de
informação aqui registrada inclui dados sobre navegação, registro de atividades
não corriqueiras (condições adversas do clima, por exemplo), parâmetros de
aquisição etc.~\citep{Manual:segy}. O número de cabeçalhos depende da quantidade
de informação que se deseja reportar.

A partir de então, sempre será alternado um campo cabeçalho do
traço~(\emph{Trace Header}), de 240 bytes e os dados do traço~(\emph{Trace
Data}), de tamanho variável. Há exatamente um cabeçalho para cada~\emph{trace
data}. O cabeçalho possui atributos relacionados aos traços, utilizando inteiros
de dois ou quatro bytes. Algumas das informações que pode conter um cabeçalho
são identificação do tiro que gerou o traço, código de identificação do traço,
elevação do terreno em que o traço foi registrado, latitude e longitude do
posicionamento do geofone, data e hora da aquisição do traço, dentre outros.

Os campos~\emph{trace data} contém os valores obtidos pelos geofones no decorrer
da aquisição. Esses dados são armazenados em sequência, utilizando
números~\emph{float}. A representação~\emph{IBM single precision floating point
format} é a mais utilizada para este fim~\citep{tp:bitmask}.