\subsection{Razão de compressão}

\begin{figure}[hbtp]
\centering
\includegraphics[scale=1]{fig/res/consolidado_rc.png}
\caption[Razão de compressão consolidada]{Consolidação da razão de compressão envolvendo todas as metodologias}
\label{Figura:ConsolidadoRC}
\end{figure}
 

 A tabela~\ref{Tabela:RazaoCompressaoPrePosStack} sumariza os resultados obtidos pelo tipo de arquivo testado. Em dados pré-stack, o GARCH conseguiu uma razão de
 compressão melhor em $48\%$ dos testes, empatou em $4\%$ e foi inferior também em $48\%$. Nesse caso, a utilização ou não do GARCH, olhando as razões de compressão globais para pré-stack foi indiferente. Para dados pós-stack, o GARCH foi melhor em $46\%$, pior em $47\%$ e indiferente em $7\%$ dos testes. Dados cujas características são desconhecidas (marcados como~\emph{Não Disponível} na tabela~\ref{Tabela:RazaoCompressaoPrePosStack}) representam $53\%$, $2\%$ e $45\%$ de ganho, empate e perda, respectivamente.
 
\begin{center}
\begin{longtable}{cccc}
\toprule
\rowcolor{white}
\caption{Compilado da razão de compressão entre dados pré e pós-stack}
\label{Tabela:RazaoCompressaoPrePosStack} \\
\midrule
\rowcolor{white}
   \specialcell{Tipo de\\dado experimentos} & \specialcell{Melhor com\\GARCH} &
   Empate & \specialcell{Melhor sem\\GARCH} \\
\midrule
\endfirsthead
%\multicolumn{8}{c}%
%{\tablename\ \thetable\ -- \textit{Continuação da página anterior}} \\
\midrule
\rowcolor{white}
   \specialcell{Tipo de\\dado experimentos} & \specialcell{Melhor com\\GARCH} &
   Empate & \specialcell{Melhor sem\\GARCH} \\
\toprule
\endhead
\midrule \\ % \multicolumn{8}{r}{\textit{Continua na próxima página}} \\
\endfoot
\bottomrule
\endlastfoot
    Não Disponível & 53\%  & 2\%   & 45\% \\
    Pós-stack & 46\%  & 7\%   & 47\% \\
    Pré-stack & 48\%  & 4\%   & 48\% \\
\end{longtable}
\end{center}

 Já a tabela~\ref{Tabela:RazaoCompressaoMigradoNaoMigrado} compila os resultados de dados migrados e não migrados. Nos dados migrados, metade dos testes foi melhor sem a utilização do GARCH, em $7\%$ houve empate e no restante a utilização do GARCH foi mais vantajosa. Para dados não migrados, uma inversão ocorre, metade dos testes foi 
 favorável ao GARCH, em $4\%$ a utilização ou não do GARCH foi indiferente e em $46\%$ dos testes em dados não migrados, o GARCH não apresentou uma solução melhor. Em dados onde a informação de migração não foi encontrada, a utilização do GARCH foi favorável em $53\%$ dos testes, irrelevante em $2\%$ e desfavorável nos $45\%$ restantes.
 
 
 \begin{center}
\begin{longtable}{cccc}
\toprule
\rowcolor{white}
 \caption{Compilado da razão de compressão entre dados migrados e não migrados}
 \label{Tabela:RazaoCompressaoMigradoNaoMigrado} \\
\midrule
\rowcolor{white}
   \specialcell{Tipo de\\dado experimentos} & \specialcell{Melhor com\\GARCH} &
   Empate & \specialcell{Melhor sem\\GARCH} \\
\midrule
\endfirsthead
%\multicolumn{8}{c}%
%{\tablename\ \thetable\ -- \textit{Continuação da página anterior}} \\
\midrule
\rowcolor{white}
   \specialcell{Tipo de\\dado experimentos} & \specialcell{Melhor com\\GARCH} &
   Empate & \specialcell{Melhor sem\\GARCH} \\
\toprule
\endhead
\midrule \\ % \multicolumn{8}{r}{\textit{Continua na próxima página}} \\
\endfoot
\bottomrule
\endlastfoot
    Não Disponível & 53\%  & 2\%   & 45\% \\
    Não Migrado & 50\%  & 4\%   & 46\% \\
    Migrado & 43\%  & 7\%   & 50\% \\
\end{longtable}
\end{center}
 
 
Na tabela~\ref{Tabela:RazaoCompressaoRuidoEstimado}, são mostrados os resultados de acordo com o ruído estimado no dado. Em dados com ruído estimado em alto, a utilização do GARCH só foi superior em $29\%$ dos testes, em $9\%$ houve empate e inferior em $61\%$ dos testes. Em dados com ruído estimado em baixo, o GARCH foi superior em $51\%$ dos testes, em $2\%$ houve empate e nos $47\%$ restantes foi inferior. Em dados com ruído estimado em médio, o GARCH foi superior em $53\%$ dos casos de teste, empatou em $5\%$ e perdeu em $42\%$ dos testes. Esses resultados mostram que o GARCH conseguiu gerar conjuntos de resíduos mais propícios para a compressão do que os modelos AR, ARMA e ARIMA puramente para arquivos com ruído baixo ou médio.

 \begin{center}
\begin{longtable}{cccc}
\toprule
\rowcolor{white}
 \caption{Compilado da razão de compressão entre dados com relação ao ruído
 estimado do dado}\label{Tabela:RazaoCompressaoRuidoEstimado} \\
\midrule
\rowcolor{white}
   \specialcell{Ruido\\estimado\\no dado} & \specialcell{Melhor com\\GARCH} &
   Empate & \specialcell{Melhor sem\\GARCH} \\
\midrule
\endfirsthead
%\multicolumn{8}{c}%
%{\tablename\ \thetable\ -- \textit{Continuação da página anterior}} \\
\midrule
\rowcolor{white}
   \specialcell{Ruido\\estimado\\no dado} & \specialcell{Melhor com\\GARCH} &
   Empate & \specialcell{Melhor sem\\GARCH} \\
\toprule
\endhead
\midrule \\ % \multicolumn{8}{r}{\textit{Continua na próxima página}} \\
\endfoot
\bottomrule
\endlastfoot
    Alto & 29\%  & 9\%   & 61\% \\
	Baixo & 51\%  & 2\%   & 47\% \\
	Médio & 53\%  & 5\%   & 42\% \\
\end{longtable}
\end{center}

 
A tabela~\ref{Tabela:RazaoCompressaoSinteticoOuNaoSintetico} mostra a diferença
de desempenho das soluções sem e com GARCH em dados sintéticos. Para dados
sintéticos, o GARCH alcançou uma razão de compressão melhor em $47\%$ das vezes,
empatou em $3\%$ e foi superado em $50\%$ dos testes. Já para dados não
sintéticos, esses valores mudam para $49\%$, $5\%$ e $46\%$, respectivamente.

 \begin{center}
\begin{longtable}{cccc}
\toprule
\rowcolor{white}
 \caption{Compilado da razão de compressão entre dados sintéticos e
 não-sintéticos} \label{Tabela:RazaoCompressaoSinteticoOuNaoSintetico} \\
\midrule
\rowcolor{white}
   \specialcell{Tipo de\\dado experimentos} & \specialcell{Melhor com\\GARCH} &
   Empate & \specialcell{Melhor sem\\GARCH} \\
\midrule
\endfirsthead
%\multicolumn{8}{c}%
%{\tablename\ \thetable\ -- \textit{Continuação da página anterior}} \\
\midrule
\rowcolor{white}
   \specialcell{Tipo de\\dado experimentos} & \specialcell{Melhor com\\GARCH} &
   Empate & \specialcell{Melhor sem\\GARCH} \\
\toprule
\endhead
\midrule \\ % \multicolumn{8}{r}{\textit{Continua na próxima página}} \\
\endfoot
\bottomrule
\endlastfoot
    Não Sintético & 49\%  & 5\%   & 46\% \\
    Sintético & 47\%  & 3\%   & 50\% \\
\end{longtable}
\end{center}

\subsection{Evolução Entropia}

\begin{figure}[hbtp]
\centering
\includegraphics[scale=1]{fig/res/consolidado_ent.png}
\caption[Redução de entropia consolidada]{Consolidação da evolução da entropia
envolvendo todas as metodologias. Em $53\%$ dos testes, a utilização do GARCH
trouxe um resíduo com entropia menos elevada do que a não utilização do modelo.}
\label{Figura:ConsolidadoEE}
\end{figure}

\begin{figure}[hbtp]
\centering
\includegraphics[scale=0.9]{fig/res/consolidado_reducaoent.png}
\caption[Métrica de redução de entropia consolidada]{Medição da evolução da
entropia total envolvendo todas as metodologias.}
\label{Figura:ConsolidadoEEMetrica}
\end{figure}

A tabela~\ref{Tabela:EvolucaoEntropiaPrePosStack} apresenta a evolução da
entropia de dados pré e pós-stack levando em consideração todas as
metodologias de teste. Em dados pré-stack, a utilização do GARCH foi positiva em
$56\%$ dos testes, inferior em $20\%$ e irrelevante em quase um quarto
dos testes. Para dados pós-stack, houve vantagem utilizando o GARCH em $37\%$
dos testes e desvantagem em $26\%$; nos $37\%$ restantes a utilização do modelo
foi irrelevante. Para dados sem informação disponível, o uso do
GARCH apresentou ganhos em $63\%$, em $30\%$ houve piora na evolução da
entropia do resíduo em relação ao dado original e, com $7\%$ dos testes, não
houve diferença na utilização do modelo com GARCH.

\begin{center}
\begin{longtable}{cccc}
\toprule
\rowcolor{white}
\caption{Compilado da evolução da entropia entre dados pré e pós-stack}
\label{Tabela:EvolucaoEntropiaPrePosStack} \\
\midrule
\rowcolor{white}
   \specialcell{Tipo de\\dado experimentos} & \specialcell{Melhor sem\\GARCH} &
   Empate & \specialcell{Melhor com\\GARCH} \\
\midrule
\endfirsthead
%\multicolumn{8}{c}%
%{\tablename\ \thetable\ -- \textit{Continuação da página anterior}} \\
\midrule
\rowcolor{white}
   \specialcell{Tipo de\\dado experimentos} & \specialcell{Melhor sem\\GARCH} &
   Empate & \specialcell{Melhor com\\GARCH} \\
\toprule
\endhead
\midrule \\ % \multicolumn{8}{r}{\textit{Continua na próxima página}} \\
\endfoot
\bottomrule
\endlastfoot
    Não Disponível & 30\%  & 7\%   & 63\% \\
    Pós-stack & 26\%  & 37\%  & 37\% \\
    Pré-stack & 20\%  & 24\%  & 56\% \\
\end{longtable}
\end{center}

Já a tabela~\ref{Tabela:EvolucaoEntropiaMigradoNaoMigrado} compila os resultados
da evolução da entropia de dados migrados e não migrados. Em dados migrados,
$42\%$ dos testes apresentaram melhor desempenho com GARCH, $35\%$ não
apresentaram diferença e nos $23\%$ restantes a utilização do GARCH trouxe um
conjunto de resíduos com entropa superior a do dado original. Em dados não
migrados, mais da metade dos testes com a utilização do GARCH obteve um
conjunto de resíduos com entropia inferior, em $23\%$ houve empate e nos $21\%$
restantes a não utilização do modelo GARCH trouxe resíduos com entropia
superior. Para dados sem a informação de migrado ou não disponível, os índices
de ganho, empate e perda na utilização do GARCH foram, respectivamente, 63\%,
7\% e 30\%.

 \begin{center}
\begin{longtable}{cccc}
\toprule
\rowcolor{white}
 \caption{Compilado da evolução da entropia entre dados migrados e não
 migrados} \label{Tabela:EvolucaoEntropiaMigradoNaoMigrado} \\
\midrule
\rowcolor{white}
   \specialcell{Tipo de\\dado experimentos} & \specialcell{Melhor sem\\GARCH} &
   Empate & \specialcell{Melhor com\\GARCH} \\
\midrule
\endfirsthead
%\multicolumn{8}{c}%
%{\tablename\ \thetable\ -- \textit{Continuação da página anterior}} \\
\midrule
\rowcolor{white}
   \specialcell{Tipo de\\dado experimentos} & \specialcell{Melhor sem\\GARCH} &
   Empate & \specialcell{Melhor com\\GARCH} \\
\toprule
\endhead
\midrule \\ % \multicolumn{8}{r}{\textit{Continua na próxima página}} \\
\endfoot
\bottomrule
\endlastfoot
    Não Disponível   & 30\%  & 7\%   & 63\% \\
    Não Migrado & 21\%  & 23\%  & 56\% \\
    Migrado & 23\%  & 35\%  & 42\% \\
\end{longtable}
\end{center}

Na tabela~\ref{Tabela:EvolucaoEntropiaRuidoEstimado}, são mostrados os
resultados da evolução da entropia de acordo com o ruído estimado no dado. Em
dados com ruído estimado em alto, o GARCH foi inferior em $14\%$ dos testes,
houve empate em $55\%$ e ganho sem GARCH em $31\%$ dos testes realizados. Em
dados com ruído estimado em baixo, o GARCH foi superior em $55\%$ dos testes,
irrelevante em $31\%$ e perdeu nos $14\%$ restantes. Para dados com ruído
estimado em médio, a utilização do GARCH foi melhor em $59\%$ dos testes,
empatou em $9\%$ e foi vencida nos $32\%$ que sobram.

 \begin{center}
\begin{longtable}{cccc}
\toprule
\rowcolor{white}
 \caption{Compilado da evolução da entropia entre dados com relação ao ruído
 estimado do dado}\label{Tabela:EvolucaoEntropiaRuidoEstimado} \\
\midrule
\rowcolor{white}
   \specialcell{Ruido\\estimado\\no dado} & \specialcell{Melhor sem\\GARCH} &
   Empate & \specialcell{Melhor com\\GARCH} \\
\midrule
\endfirsthead
%\multicolumn{8}{c}%
%{\tablename\ \thetable\ -- \textit{Continuação da página anterior}} \\
\midrule
\rowcolor{white}
   \specialcell{Ruido\\estimado\\no dado} & \specialcell{Melhor sem\\GARCH} &
   Empate & \specialcell{Melhor com\\GARCH} \\
\toprule
\endhead
\midrule \\ % \multicolumn{8}{r}{\textit{Continua na próxima página}} \\
\endfoot
\bottomrule
\endlastfoot
    Alto  & 14\%  & 55\%  & 31\% \\
    Baixo & 14\%  & 31\%  & 55\% \\
    Médio & 32\%  & 9\%   & 59\% \\
\end{longtable}
\end{center}

A tabela~\ref{Tabela:EvolucaoEntropiaSinteticoOuNaoSintetico} mostra a
diferença de desempenho das soluções sem e com GARCH em dados sintéticos. Para dados
sintéticos, o GARCH alcançou uma entropia melhor em $61\%$ das vezes,
empatou em $9\%$ e foi superado em $30\%$ dos testes. Já para dados não
sintéticos, esses valores mudam para $49\%$, $33\%$ e $19\%$, respectivamente.

 \begin{center}
\begin{longtable}{cccc}
\toprule
\rowcolor{white}
 \caption{Compilado da evolução da entropia entre dados sintéticos e
 não-sintéticos} \label{Tabela:EvolucaoEntropiaSinteticoOuNaoSintetico} \\
\midrule
\rowcolor{white}
   \specialcell{Tipo de\\dado experimentos} & \specialcell{Melhor sem\\GARCH} &
   Empate & \specialcell{Melhor com\\GARCH} \\
\midrule
\endfirsthead
%\multicolumn{8}{c}%
%{\tablename\ \thetable\ -- \textit{Continuação da página anterior}} \\
\midrule
\rowcolor{white}
   \specialcell{Tipo de\\dado experimentos} & \specialcell{Melhor sem\\GARCH} &
   Empate & \specialcell{Melhor com\\GARCH} \\
\toprule
\endhead
\midrule \\ % \multicolumn{8}{r}{\textit{Continua na próxima página}} \\
\endfoot
\bottomrule
\endlastfoot
    Não Sintético & 19\%  & 33\%  & 49\% \\
    Sintético & 30\%  & 9\%   & 61\% \\
\end{longtable}
\end{center}

 \subsection{Evolução da Correlação}

\begin{figure}[hbtp]
\centering
\includegraphics[scale=1]{fig/res/consolidado_correlacao.png}
\caption[Razão de compressão consolidada]{Consolidação da razão de compressão envolvendo todas as metodologias}
\label{Figura:ConsolidadoRC}
\end{figure}

A tabela~\ref{Tabela:EvolucaoCorrelacaoPrePosStack} apresenta a evolução da
correlação de dados pré e pós-stack levando em consideração todas as
metodologias de teste. Em dados pré-stack, a utilização do GARCH foi positiva em
$6\%$ dos testes, inferior em $28\%$ e irrelevante em dois terços dos testes.
Para dados pós-stack, houve vantagem utilizando o GARCH em $9\%$ dos testes e
desvantagem em $20\%$; nos $71\%$ restantes a utilização do modelo foi
irrelevante. Para dados sem informação disponível, não houve testes em que o
GARCH tenha apresentado algum ganho, em $7\%$ houve piora na evolução da
correlação do resíduo em relação ao dado original e, com $93\%$ dos testes, não
houve diferença na utilização do modelo com GARCH.

\begin{center}
\begin{longtable}{cccc}
\toprule
\rowcolor{white}
\caption{Compilado da evolução da correlação entre dados pré e pós-stack}
\label{Tabela:EvolucaoCorrelacaoPrePosStack} \\
\midrule
\rowcolor{white}
   \specialcell{Tipo de\\dado experimentos} & \specialcell{Melhor com\\GARCH} &
   Empate & \specialcell{Melhor sem\\GARCH} \\
\midrule
\endfirsthead
%\multicolumn{8}{c}%
%{\tablename\ \thetable\ -- \textit{Continuação da página anterior}} \\
\midrule
\rowcolor{white}
   \specialcell{Tipo de\\dado experimentos} & \specialcell{Melhor com\\GARCH} &
   Empate & \specialcell{Melhor sem\\GARCH} \\
\toprule
\endhead
\midrule \\ % \multicolumn{8}{r}{\textit{Continua na próxima página}} \\
\endfoot
\bottomrule
\endlastfoot
    Não Disponível & 0\%   & 93\%  & 7\% \\
    Pós-stack & 9\%   & 71\%  & 20\% \\
    Pré-stack & 6\%   & 66\%  & 28\%
\end{longtable}
\end{center}

Já a tabela~\ref{Tabela:EvolucaoCorrelacaoMigradoNaoMigrado} compila os resultados de dados migrados e não
migrados. Em dados migrados, $43\%$ dos testes apresentaram melhor desempenho
com GARCH, $7\%$ não apresentaram diferença e na metade restante a utilização do
GARCH trouxe um conjunto de resíduos mais correlacionados do que a não
utilização dele. Em dados não migrados, entretanto, metade dos testes com a
utilização do GARCH obteve um conjunto de resíduos mais descorrelacionado, em
$4\%$ houve empate e nos $46\%$ restantes a não utilização do modelo GARCH
trouxe resíduos mais descorrelacionados. Para dados sem a informação de migrado
ou não disponível, os índices de ganho, empate e perda na utilização do GARCH
foram, respectivamente, 53\%, 2\% e 45\%.

 \begin{center}
\begin{longtable}{cccc}
\toprule
\rowcolor{white}
 \caption{Compilado da evolução da correlação entre dados migrados e não
 migrados} \label{Tabela:EvolucaoCorrelacaoMigradoNaoMigrado} \\
\midrule
\rowcolor{white}
   \specialcell{Tipo de\\dado experimentos} & \specialcell{Melhor com\\GARCH} &
   Empate & \specialcell{Melhor sem\\GARCH} \\
\midrule
\endfirsthead
%\multicolumn{8}{c}%
%{\tablename\ \thetable\ -- \textit{Continuação da página anterior}} \\
\midrule
\rowcolor{white}
   \specialcell{Tipo de\\dado experimentos} & \specialcell{Melhor com\\GARCH} &
   Empate & \specialcell{Melhor sem\\GARCH} \\
\toprule
\endhead
\midrule \\ % \multicolumn{8}{r}{\textit{Continua na próxima página}} \\
\endfoot
\bottomrule
\endlastfoot
    Não Disponível & 53\%  & 2\%   & 45\% \\
    Não Migrado & 50\%  & 4\%   & 46\% \\
    Migrado & 43\%  & 7\%   & 50\% \\
\end{longtable}
\end{center}

Na tabela~\ref{Tabela:EvolucaoCorrelacaoRuidoEstimado}, são mostrados os resultados
de acordo com o ruído estimado no dado. Em dados com ruído estimado em alto, o
GARCH foi inferior em $29\%$ dos testes, houve empate em $9\%$ e ganho sem GARCH
em $61\%$ dos testes realizados. Em dados com ruído estimado em baixo, o GARCH
foi superior em $51\%$ dos testes, irrelevante em $2\%$ e perdeu nos $47\%$
restantes. Para dados com ruído estimado em médio, a utilização do GARCH
foi melhor em $53\%$ dos testes, empatou em $5\%$ e foi vencida nos $42\%$ que
sobram.

 \begin{center}
\begin{longtable}{cccc}
\toprule
\rowcolor{white}
 \caption{Compilado da evolução da entropia entre dados com relação ao ruído
 estimado do dado}\label{Tabela:EvolucaoCorrelacaoRuidoEstimado} \\
\midrule
\rowcolor{white}
   \specialcell{Ruido\\estimado\\no dado} & \specialcell{Melhor com\\GARCH} &
   Empate & \specialcell{Melhor sem\\GARCH} \\
\midrule
\endfirsthead
%\multicolumn{8}{c}%
%{\tablename\ \thetable\ -- \textit{Continuação da página anterior}} \\
\midrule
\rowcolor{white}
   \specialcell{Ruido\\estimado\\no dado} & \specialcell{Melhor com\\GARCH} &
   Empate & \specialcell{Melhor sem\\GARCH} \\
\toprule
\endhead
\midrule \\ % \multicolumn{8}{r}{\textit{Continua na próxima página}} \\
\endfoot
\bottomrule
\endlastfoot
    Alto  & 29\%  & 9\%   & 61\% \\
    Baixo & 51\%  & 2\%   & 47\% \\
    Médio & 53\%  & 5\%   & 42\% \\
\end{longtable}
\end{center}

A tabela~\ref{Tabela:EvolucaoCorrelacaoSinteticoOuNaoSintetico} mostra a
diferença de desempenho da correlação das soluções sem e com GARCH em dados
sintéticos. Para dados sintéticos, o GARCH alcançou uma razão de compressão
melhor em $47\%$ das vezes, empatou em $3\%$ e foi superado em $50\%$ dos testes. Já para dados não
sintéticos, esses valores mudam para $49\%$, $5\%$ e $46\%$, respectivamente.

 \begin{center}
\begin{longtable}{cccc}
\toprule
\rowcolor{white}
 \caption{Compilado da evolução da correlação entre dados sintéticos e
 não-sintéticos} \label{Tabela:EvolucaoCorrelacaoSinteticoOuNaoSintetico} \\
\midrule
\rowcolor{white}
   \specialcell{Tipo de\\dado experimentos} & \specialcell{Melhor com\\GARCH} &
   Empate & \specialcell{Melhor sem\\GARCH} \\
\midrule
\endfirsthead
%\multicolumn{8}{c}%
%{\tablename\ \thetable\ -- \textit{Continuação da página anterior}} \\
\midrule
\rowcolor{white}
   \specialcell{Tipo de\\dado experimentos} & \specialcell{Melhor com\\GARCH} &
   Empate & \specialcell{Melhor sem\\GARCH} \\
\toprule
\endhead
\midrule \\ % \multicolumn{8}{r}{\textit{Continua na próxima página}} \\
\endfoot
\bottomrule
\endlastfoot
    Não Sintético & 7\%   & 68\%  & 25\% \\
    Sintético & 5\%   & 71\%  & 24\% \\
\end{longtable}
\end{center}

\subsection{Evolução Variância}

\begin{figure}[hbtp]
\centering
\includegraphics[scale=1]{fig/res/consolidado_var.png}
\caption[Variância consolidada]{Consolidação da evolução da variância envolvendo
todas as metodologias}
\label{Figura:ConsolidadoVAR}
\end{figure}

A tabela~\ref{Tabela:EvolucaoVarianciaPrePosStack} apresenta a evolução da
variância de dados pré e pós-stack levando em consideração todas as
metodologias de teste. Em dados pré-stack, a utilização do GARCH foi positiva em
$16\%$ dos testes, inferior em $80\%$ e irrelevante nos outros $4\%$. Para dados
pós-stack, houve vantagem utilizando o GARCH em $12\%$ dos testes e desvantagem
em $87\%$; nos $1\%$ restantes a utilização do modelo foi irrelevante. Para
dados sem informação disponível, houve ganho em $22\%$ dos testes e piora nos
$78\%$ restantes.

\begin{center}
\begin{longtable}{cccc}
\toprule
\rowcolor{white}
\caption{Compilado da evolução da variância entre dados pré e pós-stack}
\label{Tabela:EvolucaoVarianciaPrePosStack} \\
\midrule
\rowcolor{white}
   \specialcell{Tipo de\\dado experimentos} & \specialcell{Melhor com\\GARCH} &
   Empate & \specialcell{Melhor sem\\GARCH} \\
\midrule
\endfirsthead
%\multicolumn{8}{c}%
%{\tablename\ \thetable\ -- \textit{Continuação da página anterior}} \\
\midrule
\rowcolor{white}
   \specialcell{Tipo de\\dado experimentos} & \specialcell{Melhor com\\GARCH} &
   Empate & \specialcell{Melhor sem\\GARCH} \\
\toprule
\endhead
\midrule \\ % \multicolumn{8}{r}{\textit{Continua na próxima página}} \\
\endfoot
\bottomrule
\endlastfoot
    Não Disponível   & 22\%  & 0\%   & 78\% \\
    Pós-stack & 12\%  & 1\%   & 87\% \\
    Pré-stack & 16\%  & 4\%   & 80\% \\
\end{longtable}
\end{center}

Já a tabela~\ref{Tabela:EvolucaoVarianciaMigradoNaoMigrado} compila os
resultados de dados migrados e não migrados. Em dados migrados, $15\%$ dos
testes apresentaram melhor desempenho com GARCH, $1\%$ não apresentou
diferença e nos $81\%$ restantes a utilização do GARCH trouxe um conjunto de
resíduos com variância superior do que a não utilização dele. Em dados não
migrados, os valores são bem parecidos, com uma pequena variação de três pontos
percentuais a mais ao empate e a menos na utilização sem GARCH. Para dados sem a
informação de migrado ou não disponível, os índices de ganho, empate e perda na utilização do GARCH
foram, respectivamente, 78\%, 0\% e 22\%.

 \begin{center}
\begin{longtable}{cccc}
\toprule
\rowcolor{white}
 \caption{Compilado da evolução da correlação entre dados migrados e não
 migrados} \label{Tabela:EvolucaoVarianciaMigradoNaoMigrado} \\
\midrule
\rowcolor{white}
   \specialcell{Tipo de\\dado experimentos} & \specialcell{Melhor com\\GARCH} &
   Empate & \specialcell{Melhor sem\\GARCH} \\
\midrule
\endfirsthead
%\multicolumn{8}{c}%
%{\tablename\ \thetable\ -- \textit{Continuação da página anterior}} \\
\midrule
\rowcolor{white}
   \specialcell{Tipo de\\dado experimentos} & \specialcell{Melhor com\\GARCH} &
   Empate & \specialcell{Melhor sem\\GARCH} \\
\toprule
\endhead
\midrule \\ % \multicolumn{8}{r}{\textit{Continua na próxima página}} \\
\endfoot
\bottomrule
\endlastfoot
    Não Disponível & 22\%  & 0\%   & 78\% \\
    Não Migrado & 15\%  & 4\%   & 81\% \\
    Migrado   & 15\%  & 1\%   & 85\% \\
\end{longtable}
\end{center}

Na tabela~\ref{Tabela:EvolucaoVarianciaRuidoEstimado}, são mostrados os
resultados de acordo com o ruído estimado no dado. Em dados com ruído estimado em alto, o
GARCH foi inferior em $17\%$ dos testes, houve empate em $1\%$ e ganho sem GARCH
em $82\%$ dos testes realizados. Em dados com ruído estimado em baixo, o GARCH
foi superior em $17\%$ dos testes, irrelevante em $2\%$ e perdeu nos $81\%$
restantes. Para dados com ruído estimado em médio, a utilização do GARCH
foi melhor em $13\%$ dos testes, empatou em $4\%$ e foi vencida nos $83\%$ que
sobram.

 \begin{center}
\begin{longtable}{cccc}
\toprule
\rowcolor{white}
 \caption{Compilado da evolução da variância entre dados com relação ao ruído
 estimado do dado}\label{Tabela:EvolucaoVarianciaRuidoEstimado} \\
\midrule
\rowcolor{white}
   \specialcell{Ruido\\estimado\\no dado} & \specialcell{Melhor com\\GARCH} &
   Empate & \specialcell{Melhor sem\\GARCH} \\
\midrule
\endfirsthead
%\multicolumn{8}{c}%
%{\tablename\ \thetable\ -- \textit{Continuação da página anterior}} \\
\midrule
\rowcolor{white}
   \specialcell{Ruido\\estimado\\no dado} & \specialcell{Melhor com\\GARCH} &
   Empate & \specialcell{Melhor sem\\GARCH} \\
\toprule
\endhead
\midrule \\ % \multicolumn{8}{r}{\textit{Continua na próxima página}} \\
\endfoot
\bottomrule
\endlastfoot
    Alto  & 17\%  & 1\%   & 82\% \\
    Baixo & 17\%  & 2\%   & 81\% \\
    Médio & 13\%  & 4\%   & 83\% \\
\end{longtable}
\end{center}

A tabela~\ref{Tabela:EvolucaoVarianciaSinteticoOuNaoSintetico} mostra a
diferença de desempenho da correlação das soluções sem e com GARCH em dados
sintéticos. Para dados sintéticos, o GARCH alcançou uma evolução da variância
 melhor em $16\%$ das vezes, empatou em $4\%$ e foi superado em $80\%$
dos testes. Já para dados não sintéticos, esses valores mudam para $15\%$, $2\%$
e $83\%$, respectivamente.

 \begin{center}
\begin{longtable}{cccc}
\toprule
\rowcolor{white}
 \caption{Compilado da evolução da variância entre dados sintéticos e
 não-sintéticos} \label{Tabela:EvolucaoVarianciaSinteticoOuNaoSintetico} \\
\midrule
\rowcolor{white}
   \specialcell{Tipo de\\dado experimentos} & \specialcell{Melhor com\\GARCH} &
   Empate & \specialcell{Melhor sem\\GARCH} \\
\midrule
\endfirsthead
%\multicolumn{8}{c}%
%{\tablename\ \thetable\ -- \textit{Continuação da página anterior}} \\
\midrule
\rowcolor{white}
   \specialcell{Tipo de\\dado experimentos} & \specialcell{Melhor com\\GARCH} &
   Empate & \specialcell{Melhor sem\\GARCH} \\
\toprule
\endhead
\midrule \\ % \multicolumn{8}{r}{\textit{Continua na próxima página}} \\
\endfoot
\bottomrule
\endlastfoot
    Não Sintético & 15\%  & 2\%   & 83\% \\
    Sintético & 16\%  & 4\%   & 80\% \\
\end{longtable}
\end{center}