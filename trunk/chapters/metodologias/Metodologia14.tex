
\begin{center}
\begin{longtable}{cccccc}
\toprule
\rowcolor{white}
\caption[Metodologia XIV: comparativo de convergência de soluções]{Comparativo
   de quantidade de experimentos cujas soluções convergiram com e sem a
   utilização do GARCH na metodologia XIV} \label{Tab:convergenciaMet14} \\
\midrule
   Cenário & \specialcell{Total experimentos} & Convergiram &
   \specialcell{Não convergiram} & \% sucesso \\
\midrule
\endfirsthead
%\multicolumn{8}{c}%
%{\tablename\ \thetable\ -- \textit{Continuação da página anterior}} \\
\midrule
\rowcolor{white}
   Cenário & \specialcell{Total experimentos} & Convergiram &
   \specialcell{Não convergiram} & \% sucesso \\
\toprule
\endhead
\midrule \\ % \multicolumn{8}{r}{\textit{Continua na próxima página}} \\
\endfoot
\bottomrule
\endlastfoot
	Sem GARCH & 39 & 39 & 0 & 100\% \\
	Com GARCH & 39 & 39 & 0 & 100\% \\
\end{longtable}
\end{center}

%%%%%%%%%%%%%%%%%%%%%%%%%%%%%%%%%%%%%%%%%%%%%%%%%%%%%%%%%%%%%%%%%%%%%%%%%%%%%%%%%%%%%%%%%
\begin{center}
\begin{longtable}{cccccc}
\toprule
\rowcolor{white}
\caption[Metodologia XIV: Razão de compressão]{Razão de compressão dos
experimentos sem e com GARCH na Metodologia XIV.
Valores em bytes.} \label{Tab:razaocompressaoMet} \\
\midrule
Conjunto & \specialcell{Tamanho \\Original} & \specialcell{Tamanho
\\Comprimido\\Com GARCH} & \specialcell{Tamanho
\\Comprimido\\Sem GARCH} & \specialcell{Razão \\Compressão
\\Sem GARCH} & \specialcell{Razão \\Compressão
\\Com GARCH} \\
\midrule
\endfirsthead
%\multicolumn{8}{c}%
%{\tablename\ \thetable\ -- \textit{Continuação da página anterior}} \\
\midrule
\rowcolor{white}
Conjunto & \specialcell{Tamanho \\Original} & \specialcell{Tamanho
\\Comprimido\\Com GARCH} & \specialcell{Tamanho
\\Comprimido\\Sem GARCH} & \specialcell{Razão \\Compressão
\\Sem GARCH} & \specialcell{Razão \\Compressão
\\Com GARCH} \\
\toprule
\endhead
\midrule \\ % \multicolumn{8}{r}{\textit{Continua na próxima página}} \\
\endfoot
\bottomrule
\endlastfoot
    A1    & 1152000 & 867408 & 871169 & 1,33  & 1,32 \\
    A2    & 1152000 & 814667 & 815188 & 1,41  & 1,41 \\
    A3    & 1152000 & 850622 & 850531 & 1,35  & 1,35 \\
    B1    & 518592 & 200274 & 199890 & 2,59  & 2,59 \\
    B2    & 518592 & 200274 & 199890 & 2,59  & 2,59 \\
    B3    & 518592 & 200274 & 199890 & 2,59  & 2,59 \\
    C1    & 288192 & 261751 & 256221 & 1,10  & 1,12 \\
    C2    & 288192 & 259352 & 250990 & 1,11  & 1,15 \\
    C3    & 288192 & 263330 & 259909 & 1,09  & 1,11 \\
    D1    & 331200 & 279469 & 278812 & 1,19  & 1,19 \\
    D2    & 331200 & 261543 & 261879 & 1,27  & 1,26 \\
    D3    & 331200 & 251475 & 247505 & 1,32  & 1,34 \\
    E1    & 33792 & 32114 & 32177 & 1,05  & 1,05 \\
    E2    & 33792 & 32273 & 32355 & 1,05  & 1,04 \\
    E3    & 33792 & 32603 & 32618 & 1,04  & 1,04 \\
    F1    & 220992 & 184376 & 184207 & 1,20  & 1,20 \\
    F2    & 220992 & 137951 & 143971 & 1,60  & 1,53 \\
    F3    & 220992 & 154981 & 153210 & 1,43  & 1,44 \\
    G1    & 139392 & 117176 & 117203 & 1,19  & 1,19 \\
    G2    & 139392 & 106073 & 105981 & 1,31  & 1,32 \\
    G3    & 139392 & 118822 & 118831 & 1,17  & 1,17 \\
    H1    & 360192 & 329971 & 330046 & 1,09  & 1,09 \\
    H2    & 360192 & 293597 & 295533 & 1,23  & 1,22 \\
    H3    & 360192 & 299588 & 298978 & 1,20  & 1,20 \\
    I1    & 221184 & 171934 & 172390 & 1,29  & 1,28 \\
    I2    & 221184 & 134454 & 148855 & 1,65  & 1,49 \\
    I3    & 221184 & 182502 & 181543 & 1,21  & 1,22 \\
    J1    & 591936 & 389205 & 400442 & 1,52  & 1,48 \\
    J2    & 591936 & 386014 & 390683 & 1,53  & 1,52 \\
    J3    & 591936 & 386934 & 387096 & 1,53  & 1,53 \\
    K1    & 288000 & 194423 & 197368 & 1,48  & 1,46 \\
    K2    & 288000 & 198693 & 198943 & 1,45  & 1,45 \\
    K3    & 288000 & 193167 & 217639 & 1,49  & 1,32 \\
    L1    & 480192 & 400968 & 397611 & 1,20  & 1,21 \\
    L2    & 480192 & 408448 & 407100 & 1,18  & 1,18 \\
    L3    & 480192 & 414665 & 413640 & 1,16  & 1,16 \\
    L4    & 480192 & 412280 & 393637 & 1,16  & 1,22 \\
    L5    & 480192 & 378682 & 377245 & 1,27  & 1,27 \\
    L6    & 480192 & 377208 & 373274 & 1,27  & 1,29 \\
\end{longtable}
\end{center}

% \begin{figure}[!h]
% \centering
% \includegraphics[scale=1, angle=90]{fig/res/razaocompMetXIV00.png}
% \caption[Metodologia XIV: razão de compressão dos conjuntos A, B e C]{Gráfico
% com comparativo da razão de compressão dos conjuntos A, B e C sem e com GARCH na
% Metodologia XIV}
% \label{Figura:razaocompressaoABCMet14}
% \end{figure}
%  
% \begin{figure}[!h]
% \centering
% \includegraphics[scale=1, angle=90]{fig/res/razaocompMetXIV01.png}
% \caption[Metodologia XIV: razão de compressão dos conjuntos D, E e F]{Gráfico
% com comparativo da razão de compressão dos conjuntos D, E e F sem e com GARCH na
% Metodologia XIV}
% \label{Figura:razaocompressaoDEFMet14}
% \end{figure}
% 
% \begin{figure}[!h]
% \centering
% \includegraphics[scale=1, angle=90]{fig/res/razaocompMetXIV02.png}
% \caption[Metodologia XIV: razão de compressão dos conjuntos G, H e I]{Gráfico
% com comparativo da razão de compressão dos conjuntos G, H e I sem e com GARCH na
% Metodologia XIV}
% \label{Figura:razaocompressaoGHIMet14}
% \end{figure}
% 
% \begin{figure}[!h]
% \centering
% \includegraphics[scale=1, angle=90]{fig/res/razaocompMetXIV03.png}
% \caption[Metodologia XIV: razão de compressão dos conjuntos J, K e L]{Gráfico
% com comparativo da razão de compressão dos conjuntos J, K e L sem e com GARCH na
% Metodologia XIV}
% \label{Figura:razaocompressaoJKLMet14}
% \end{figure}

% \begin{figure}[!h]
% \centering
% \includegraphics[scale=0.9]{fig/res/razaocompMetXIV04.png}
% \caption[Metodologia XIV: razão de compressão]{Gráfico com comparativo da razão
% de compressão na Metodologia XIV}
% \label{Figura:razaocompressaoPizzaMet14}
% \end{figure}

\clearpage

\begin{center}
\begin{longtable}{cccc}
\toprule
\rowcolor{white}
\caption[Metodologia XIV: evolução da entropia]{Evolução da entropia do dado
original e do resíduo calculado na metodologia XIV}
\label{tab:EvolucaoEntropiaMet14}\\
\midrule
Conjunto & \specialcell{Entropia \\Inicial} & \specialcell{Entropia do
\\Resíduo sem GARC} & \specialcell{Entropia do
\\Resíduo com GARC}  \\
\midrule
\endfirsthead
%\multicolumn{8}{c}%
%{\tablename\ \thetable\ -- \textit{Continuação da página anterior}} \\
\midrule
\rowcolor{white}
Conjunto & \specialcell{Entropia \\Inicial} & \specialcell{Entropia do
\\Resíduo sem GARC} & \specialcell{Entropia do
\\Resíduo com GARC}  \\
\toprule
\endhead
\midrule \\ % \multicolumn{8}{r}{\textit{Continua na próxima página}} \\
\endfoot
\bottomrule 
\endlastfoot
    A1    & 11,30 & 11,14 & 11,14 \\
    A2    & 11,30 & 10,62 & 10,62 \\
    A3    & 11,27 & 10,77 & 10,77 \\
    B1    & 7,64  & 4,48  & 3,67 \\
    B2    & 7,64  & 4,48  & 3,67 \\
    B3    & 7,64  & 4,48  & 3,67 \\
    C1    & 12,34 & 12,27 & 12,34 \\
    C2    & 13,18 & 12,67 & 12,60 \\
    C3    & 13,17 & 12,74 & 12,76 \\
    D1    & 9,48  & 9,48  & 9,48 \\
    D2    & 12,38 & 11,39 & 11,35 \\
    D3    & 6,45  & 6,45  & 6,45 \\
    E1    & 10,80 & 10,80 & 10,80 \\
    E2    & 10,78 & 10,78 & 10,78 \\
    E3    & 10,80 & 10,80 & 10,80 \\
    F1    & 10,20 & 10,20 & 10,20 \\
    F2    & 8,20  & 7,74  & 7,52 \\
    F3    & 9,27  & 9,19  & 8,65 \\
    G1    & 12,03 & 11,71 & 11,71 \\
    G2    & 11,79 & 11,25 & 11,25 \\
    G3    & 12,06 & 11,82 & 11,81 \\
    H1    & 8,44  & 8,44  & 8,44 \\
    H2    & 12,29 & 12,24 & 12,24 \\
    H3    & 12,33 & 12,20 & 12,20 \\
    I1    & 8,14  & 8,01  & 7,92 \\
    I2    & 9,59  & 7,70  & 7,62 \\
    I3    & 8,15  & 8,15  & 8,13 \\
    J1    & 8,50  & 7,81  & 7,24 \\
    J2    & 8,52  & 7,94  & 7,34 \\
    J3    & 8,53  & 7,86  & 7,28 \\
    K1    & 10,94 & 9,99  & 9,96 \\
    K2    & 10,89 & 9,98  & 9,91 \\
    K3    & 10,87 & 9,99  & 10,46 \\
    L1    & 11,27 & 11,27 & 11,27 \\
    L2    & 11,08 & 11,08 & 11,08 \\
    L3    & 11,31 & 11,31 & 11,31 \\
    L4    & 12,80 & 12,32 & 11,94 \\
    L5    & 10,67 & 10,67 & 10,67 \\
    L6    & 11,58 & 11,58 & 11,58 \\

\end{longtable}
\end{center}

% \begin{figure}[!h]
% \centering
% \includegraphics[scale=0.8, angle=90]{fig/res/evolucaoentropiaMetXIV00.png} 
% \caption[Metodologia XIV: evolução da entropia nos conjuntos A, B e C]{Gráfico
% com comparativo da evolução da entropia dos conjuntos A, B e C sem e com GARCH na
% Metodologia XIV}
% \label{Figura:evolucaoentropiaABCMet14}
% \end{figure}
% 
% \begin{figure}[!h]
% \centering
% \includegraphics[scale=0.8, angle=90]{fig/res/evolucaoentropiaMetXIV01.png} 
% \caption[Metodologia XIV: evolução da entropia nos conjuntos D, E e F]{Gráfico
% com comparativo da evolução da entropia dos conjuntos D, E e F sem e com GARCH na
% Metodologia XIV}
% \label{Figura:evolucaoentropiaDEFMet14}
% \end{figure}
% 
% \begin{figure}[!h]
% \centering
% \includegraphics[scale=0.8, angle=90]{fig/res/evolucaoentropiaMetXIV02.png} 
% \caption[Metodologia XIV: evolução da entropia nos conjuntos G, H e I]{Gráfico
% com comparativo da evolução da entropia dos conjuntos G, H e I sem e com GARCH na
% Metodologia XIV}
% \label{Figura:evolucaoentropiaGHIMet14}
% \end{figure}
% 
% \begin{figure}[!h]
% \centering
% \includegraphics[scale=0.6]{fig/res/evolucaoentropiaMetXIV03.png} 
% \caption[Metodologia XIV: evolução da entropia nos conjuntos J e K]{Gráfico com
% comparativo da evolução da entropia dos conjuntos J e K sem e com GARCH na
% Metodologia XIV}
% \label{Figura:evolucaoentropiaJKMet14}
% \end{figure}
% 
% \begin{figure}[!h]
% \centering
% \includegraphics[scale=0.6]{fig/res/evolucaoentropiaMetXIV04.png} 
% \caption[Metodologia XIV: evolução da entropia nos conjuntos L]{Gráfico com
% comparativo da evolução da entropia dos conjuntos L sem e com GARCH na
% Metodologia XIV}
% \label{Figura:evolucaoentropiaLMet14}
% \end{figure}

% \begin{figure}[!h]
% \centering
% \includegraphics[scale=1]{fig/res/evolucaoentropiaMetXIV05.png} 
% \caption[Metodologia XIV: evolução da entropia]{Gráfico com comparativo da
% evolução da entopia na Metodologia XIV}
% \label{Figura:evolucaoentropiaPizzaMet14}
% \end{figure}

\clearpage

\begin{center}
\begin{longtable}{ccccc|cccc}
\toprule
\rowcolor{white}
\caption[Metodologia XIV: tempo de execução]{Tempo de execução (em segundos)
dos algoritmos sem e com GARCH na Metodologia XIV. Primeiro é exibido o tempo de
execução sem a utilização do modelo GARCH, depois com o modelo. Parâmetros
modelo se refere ao tempo gasto pelo algoritmo para o cálculo dos parâmetros do
modelo, Resíduo refere-se ao tempo gasto pelo modelo para calcular o resíduo do
modelo, Cod. Arit. refere-se ao tempo gasto pela codificação aritmética para
comprimir o resíduo.} \label{tab:EvolucaoEntropiaMet14}\\
\midrule
Conj & \specialcell{Parâmetros\\modelo} &
Resíduo & \specialcell{Cod.\\Arit.} & \specialcell{Tempo\\total} &
\specialcell{Parâmetros\\modelo} &
Resíduo & \specialcell{Cod.\\Arit.} & \specialcell{Tempo\\total} \\
\midrule
\endfirsthead 
%\multicolumn{8}{c}%
%{\tablename\ \thetable\ -- \textit{Continuação da página anterior}} \\
\midrule
\rowcolor{white}
Conj & \specialcell{Parâmetros\\modelo} &
Resíduo & \specialcell{Cod.\\Arit.} & \specialcell{Tempo\\total} &
\specialcell{Parâmetros\\modelo} &
Resíduo & \specialcell{Cod.\\Arit.} & \specialcell{Tempo\\total} \\
\toprule
\endhead
\midrule \\ % \multicolumn{8}{r}{\textit{Continua na próxima página}} \\
\endfoot
\bottomrule 
\endlastfoot
A1&162&$<1$&2&165&156&1&2&159\\
A2&119&1&9&129&535&1&2&538\\
A3&128&1&3&131&820&3&2&825\\
B1&114&1&1&116&323&$<1$&1&325\\
B2&107&$<1$&$<1$&108&329&$<1$&1&330\\
B3&109&$<1$&1&110&326&1&1&328\\
C1&48&1&3&52&119&1&2&122\\
C2&41&$<1$&1&43&116&1&2&119\\
C3&48&1&2&51&108&$<1$&1&109\\
D1&50&1&2&52&109&$<1$&1&110\\
D2&126&1&3&130&360&1&1&361\\
D3&40&$<1$&1&41&364&1&2&367\\
E1&9&$<1$&$<1$&9&32&$<1$&$<1$&32\\
E2&6&$<1$&$<1$&6&35&$<1$&$<1$&35\\
E3&7&$<1$&$<1$&7&41&$<1$&$<1$&41\\
F1&25&$<1$&1&26&87&1&2&91\\
F2&73&$<1$&$<1$&73&250&$<1$&$<1$&251\\
F3&47&$<1$&2&49&247&1&2&249\\
G1&18&$<1$&2&20&85&$<1$&1&86\\
G2&57&$<1$&$<1$&58&183&$<1$&1&184\\
G3&20&$<1$&2&22&67&1&2&70\\
H1&67&$<1$&3&70&214&1&1&216\\
H2&40&1&3&44&198&1&3&203\\
H3&35&1&3&39&133&$<1$&2&135\\
I1&36&$<1$&$<1$&37&250&1&1&252\\
I2&68&$<1$&$<1$&68&254&$<1$&$<1$&255\\
I3&15&$<1$&$<1$&15&35&$<1$&$<1$&36\\
J1&94&$<1$&1&95&405&$<1$&2&407\\
J2&59&$<1$&1&60&523&2&2&527\\
J3&106&$<1$&1&107&617&1&1&618\\
K1&27&$<1$&2&30&307&1&2&309\\
K2&72&$<1$&1&73&228&$<1$&1&229\\
K3&35&1&2&37&322&1&2&325\\
L1&185&$<1$&3&189&134&$<1$&2&137\\
L2&73&$<1$&1&74&182&$<1$&1&184\\
L3&60&1&2&63&173&1&1&175\\
L4&86&1&4&91&189&2&3&194\\
L5&71&$<1$&1&72&436&1&5&442\\
L6&169&1&4&174&161&$<1$&1&162\\
\end{longtable}
\end{center}

% \begin{figure}[!h]
% \centering
% \includegraphics[scale=1, angle=90]{fig/res/tempoexecMetXIV00.png} 
% \caption[Metodologia XIV: tempo de cálculo dos parâmetros dos modelos dos
% conjuntos A, B, C e D]{Gráfico com comparativo do tempo de cálculo dos
% parâmetros dos modelos dos conjuntos A, B, C e D sem e com GARCH na Metodologia
% XIV}
% \label{Figura:tempocalculoABCDMet14}
% \end{figure}
% 
% \begin{figure}[!h]
% \centering
% \includegraphics[scale=0.75]{fig/res/tempoexecMetXIV01.png} 
% \caption[Metodologia XIV: tempo de cálculo dos parâmetros dos modelos dos
% conjuntos E, F e G]{Gráfico com comparativo do tempo de cálculo dos
% parâmetros dos modelos dos conjuntos E, F e G sem e com GARCH na Metodologia
% XIV}
% \label{Figura:tempocalculoEFGMet14}
% \end{figure}
% 
% \begin{figure}[!h]
% \centering
% \includegraphics[scale=0.75]{fig/res/tempoexecMetXIV02.png} 
% \caption[Metodologia XIV: tempo de cálculo dos parâmetros dos modelos dos
% conjuntos H, I e J]{Gráfico com comparativo do tempo de cálculo dos
% parâmetros dos modelos dos conjuntos H, I e J sem e com GARCH na Metodologia
% XIV}
% \label{Figura:tempocalculoHIJMet14}
% \end{figure}
% 
% \begin{figure}[!h]
% \centering 
% \includegraphics[scale=1, angle=90]{fig/res/tempoexecMetXIV03.png} 
% \caption[Metodologia XIV: tempo de cálculo dos parâmetros dos modelos dos
% conjuntos K e L]{Gráfico com comparativo do tempo de cálculo dos
% parâmetros dos modelos dos conjuntos K e L sem e com GARCH na Metodologia XIV}
% \label{Figura:tempocalculoKLMet14}
% \end{figure}

% \begin{figure}[!h]
% \centering
% \includegraphics[scale=0.75]{fig/res/tempoexecMetXIV04.png} 
% \caption[Metodologia XIV: tempo total relativo gasto no cálculo dos
% parâmetros do modelo]{Gráfico com comparativo do tempo total relativo de cálculo
% dos parâmetros dos modelos sem e com GARCH na Metodologia XIV}
% \label{Figura:tempocalculoPizzaMet14}
% \end{figure}

\clearpage

\begin{center}
\begin{longtable}{ccccc|cccc}
\toprule
\rowcolor{white}
\caption[Metodologia XIV: evolução da autocorrelação]{Autocorrelação do dado
original e dos resíduos gerados sem e com a utilização do modelo GARCH na
Metodologia XIV} \label{tab:EvolucaoAutocorrelacaoMet14}\\
\midrule
Conjunto & \specialcell{Autocorrelação\\Inicial} & \specialcell{Autocorrelação\\Sem
GARCH} & \specialcell{Autocorrelação\\Com GARCH} \\
\midrule
\endfirsthead 
%\multicolumn{8}{c}%
%{\tablename\ \thetable\ -- \textit{Continuação da página anterior}} \\
\midrule
\rowcolor{white}
Conjunto & \specialcell{Autocorrelação\\Inicial} & \specialcell{Autocorrelação\\Sem
GARCH} & \specialcell{Autocorrelação\\Com GARCH} \\
\toprule
\endhead
\midrule \\ % \multicolumn{8}{r}{\textit{Continua na próxima página}} \\
\endfoot
\bottomrule 
\endlastfoot
A1    & 6     & 0     & 0 \\
A2    & 5     & 0     & 0 \\
A3    & 6     & 0     & 0 \\
B1    & 6     & 3     & 3 \\
B2    & 6     & 3     & 3 \\
B3    & 6     & 3     & 3 \\
C1    & 2     & 0     & 0 \\
C2    & 1     & 0     & 1 \\
C3    & 2     & 0     & 0 \\
D1    & 2     & 4     & 4 \\
D2    & 2     & 3     & 0 \\
D3    & 2     & 3     & 3 \\
E1    & 4     & 0     & 0 \\
E2    & 4     & 0     & 0 \\
E3    & 4     & 0     & 0 \\
F1    & 1     & 4     & 4 \\
F2    & 6     & 1     & 1 \\
F3    & 6     & 1     & 1 \\
G1    & 1     & 1     & 1 \\
G2    & 2     & 1     & 1 \\
G3    & 6     & 1     & 1 \\
H1    & 1     & 0     & 0 \\
H2    & 1     & 0     & 1 \\
H3    & 1     & 0     & 0 \\
I1    & 7     & 0     & 1 \\
I2    & 1     & 3     & 3 \\
I3    & 1     & 1     & 1 \\
J1    & 3     & 7     & 7 \\
J2    & 3     & 7     & 7 \\
J3    & 8     & 0     & 0 \\
K1    & 3     & 3     & 3 \\
K2    & 3     & 3     & 3 \\
K3    & 2     & 3     & 1 \\
L1    & 2     & 0     & 0 \\
L2    & 7     & 0     & 2 \\
L3    & 11    & 0     & 0 \\
L4    & 7     & 0     & 1 \\
L5    & 7     & 0     & 0 \\
L6    & 6     & 0     & 0 \\

\end{longtable}
\end{center}

% \begin{figure}[!h]
% \centering
% \includegraphics[scale=0.75]{fig/res/evolucaoautocorrMetXIV00.png} 
% \caption[Metodologia XIV: evolução da autocorrelação nos conjuntos A, B e
% C]{Gráfico com comparativo da autocorrelação do resíduo gerado sem e com a
% utilização do modelo GARCH em relação ao dado original nos conjuntos A, B e C na
% Metodologia XIV}
% \label{Figura:autocorrelacaoABCMet14}
% \end{figure}
% 
% \begin{figure}[!h]
% \centering
% \includegraphics[scale=0.69]{fig/res/evolucaoautocorrMetXIV01.png} 
% \caption[Metodologia XIV: evolução da autocorrelação nos conjuntos D, E e
% F]{Gráfico com comparativo da autocorrelação do resíduo gerado sem e com a
% utilização do modelo GARCH em relação ao dado original nos conjuntos D, E e F na
% Metodologia XIV}
% \label{Figura:autocorrelacaoDEFMet14}
% \end{figure}
% 
% \begin{figure}[!h]
% \centering
% \includegraphics[scale=0.69]{fig/res/evolucaoautocorrMetXIV02.png} 
% \caption[Metodologia XIV: evolução da autocorrelação nos conjuntos G, H e
% I]{Gráfico com comparativo da autocorrelação do resíduo gerado sem e com a
% utilização do modelo GARCH em relação ao dado original nos conjuntos G, H e I na
% Metodologia XIV}
% \label{Figura:autocorrelacaoGHIMet14}
% \end{figure}
% 
% \begin{figure}[!h]
% \centering
% \includegraphics[scale=0.69]{fig/res/evolucaoautocorrMetXIV03.png} 
% \caption[Metodologia XIV: evolução da autocorrelação nos conjuntos J e
% K]{Gráfico com comparativo da autocorrelação do resíduo gerado sem e com a
% utilização do modelo GARCH em relação ao dado original nos conjuntos J e K na
% Metodologia XIV}
% \label{Figura:autocorrelacaoJKMet14}
% \end{figure}
% 
% \begin{figure}[!h]
% \centering
% \includegraphics[scale=0.69]{fig/res/evolucaoautocorrMetXIV04.png} 
% \caption[Metodologia XIV: evolução da autocorrelação nos conjuntos L]{Gráfico
% com comparativo da autocorrelação do resíduo gerado sem e com a utilização do modelo GARCH em relação ao dado original nos conjuntos L na
% Metodologia XIV}
% \label{Figura:autocorrelacaoLMet14}
% \end{figure}

% \begin{figure}[!h]
% \centering
% \includegraphics[scale=0.75]{fig/res/evolucaoautocorrMetXIV05.png} 
% \caption[Metodologia XIV: tempo total relativo gasto no cálculo dos
% parâmetros do modelo]{Gráfico com comparativo da redução relativa total da
% autocorrelação do resíduo sem e com a utilização do modelo GARCH na
% Metodologia XIV}
% \label{Figura:tempocalculoPizzaMet14}
% \end{figure}

\clearpage

\begin{center}
\begin{longtable}{ccccccccc}
\toprule
\rowcolor{white}
\caption[Metodologia XIV: dados estatísticos]{Média e variância do dado original
comparadas às do resíduo calculado sem e com a utilização do modelo GARCH na
Metodologia XIV} \label{tab:DadosEstatisticosMet14}\\
\midrule
    Conjunto & \specialcell{Média\\Original} &
    \specialcell{Var.\\Original} & \specialcell{Média\\Sem\\GARCH} &
    \specialcell{Var.\\Sem\\GARCH} & \specialcell{Média\\Com\\GARCH}&
    \specialcell{Var.\\Com\\GARCH} \\

\midrule
\endfirsthead 
%\multicolumn{8}{c}%
%{\tablename\ \thetable\ -- \textit{Continuação da página anterior}} \\
\midrule
\rowcolor{white}
    Conjunto & \specialcell{Média\\Orig.} &
    \specialcell{Var.\\Orig.} & \specialcell{Média\\Sem\\GARCH} &
    \specialcell{Var.\\Sem\\GARCH} & \specialcell{Média\\Com\\GARCH}&
    \specialcell{Var.\\Com\\GARCH} \\

\toprule
\endhead
\midrule \\ % \multicolumn{8}{r}{\textit{Continua na próxima página}} \\
\endfoot
\bottomrule 
\endlastfoot
A1    & 3,0E+04 & 1,8E+07 & 0,5   & 3,7E+05 & 0,4   & 3,8E+05 \\
A2    & 3,2E+04 & 1,1E+07 & 0,4   & 3,0E+05 & -1,1  & 3,0E+05 \\
A3    & 3,1E+04 & 1,4E+07 & 0,4   & 3,2E+05 & -0,4  & 3,2E+05 \\
B1    & 2,8E+04 & 4,5E+05 & 0,5   & 6,5E+02 & 0,4   & 1,1E+03 \\
B2    & 2,8E+04 & 4,5E+05 & 0,5   & 6,5E+02 & 0,4   & 1,1E+03 \\
B3    & 2,8E+04 & 4,5E+05 & 0,5   & 6,5E+02 & 0,4   & 1,1E+03 \\
C1    & 3,3E+04 & 8,1E+07 & 0,3   & 2,0E+07 & 0,2   & 2,2E+07 \\
C2    & 3,3E+04 & 4,0E+07 & 0,9   & 1,3E+07 & 2,8   & 1,7E+07 \\
C3    & 3,3E+04 & 5,7E+07 & 0,6   & 1,1E+07 & 0,8   & 1,3E+07 \\
D1    & 3,7E+04 & 4,1E+07 & 0,2   & 3,6E+06 & 0,7   & 3,6E+06 \\
D2    & 3,3E+04 & 1,2E+07 & 0,6   & 1,8E+06 & 0,7   & 1,9E+06 \\
D3    & 3,1E+04 & 1,0E+07 & -4,4  & 1,1E+06 & 5,9   & 1,5E+06 \\
E1    & 2,9E+04 & 5,8E+07 & 0,6   & 5,1E+07 & -4,7  & 5,1E+07 \\
E2    & 3,0E+04 & 5,8E+07 & 0,5   & 5,2E+07 & 0,6   & 5,3E+07 \\
E3    & 3,0E+04 & 6,0E+07 & 0,9   & 5,5E+07 & 0,8   & 5,6E+07 \\
F1    & 3,8E+04 & 3,9E+07 & 0,5   & 5,6E+06 & 0,4   & 5,6E+06 \\
F2    & 2,3E+04 & 5,4E+06 & 0,4   & 5,1E+05 & 0,4   & 5,8E+05 \\
F3    & 2,6E+04 & 6,0E+06 & 1,9   & 4,0E+05 & 0,6   & 4,8E+05 \\
G1    & 3,3E+04 & 3,3E+07 & 2,9   & 2,6E+06 & 0,1   & 2,6E+06 \\
G2    & 3,8E+04 & 1,9E+07 & 5,5   & 1,2E+06 & 2,6   & 1,3E+06 \\
G3    & 2,9E+04 & 3,6E+07 & 3,5   & 2,7E+06 & 0,4   & 2,7E+06 \\
H1    & 3,1E+04 & 3,6E+07 & -2,7  & 3,5E+07 & 38,2  & 3,5E+07 \\
H2    & 3,4E+04 & 8,1E+06 & 0,3   & 5,9E+06 & -16,0 & 5,1E+06 \\
H3    & 3,2E+04 & 7,3E+06 & 0,6   & 4,9E+06 & 0,1   & 5,1E+06 \\
I1    & 3,6E+04 & 1,2E+07 & 0,7   & 1,3E+06 & 0,2   & 1,4E+06 \\
I2    & 2,9E+04 & 1,2E+06 & 1,4   & 1,8E+05 & 0,2   & 1,9E+05 \\
I3    & 3,1E+04 & 3,3E+07 & 0,3   & 6,9E+06 & 0,7   & 7,0E+06 \\
J1    & 3,7E+04 & 1,2E+06 & 0,8   & 4,6E+05 & 0,5   & 5,1E+05 \\
J2    & 3,5E+04 & 1,5E+06 & 0,5   & 5,9E+05 & 0,5   & 6,7E+05 \\
J3    & 3,3E+04 & 1,3E+06 & 1,3   & 5,0E+05 & 0,6   & 5,8E+05 \\
K1    & 3,9E+04 & 6,9E+06 & 0,5   & 4,3E+05 & 0,6   & 4,8E+05 \\
K2    & 4,0E+04 & 6,7E+06 & -0,1  & 4,1E+05 & 0,3   & 4,6E+05 \\
K3    & 3,6E+04 & 5,8E+06 & 0,6   & 4,3E+05 & -2,3  & 1,7E+06 \\
L1    & 3,4E+04 & 2,9E+07 & -4,7  & 5,3E+06 & 0,4   & 5,2E+06 \\
L2    & 3,1E+04 & 1,5E+07 & 0,7   & 5,2E+06 & 0,7   & 5,4E+06 \\
L3    & 3,5E+04 & 1,3E+07 & -0,3  & 7,5E+06 & 1,8   & 7,6E+06 \\
L4    & 3,7E+04 & 1,8E+07 & 8,3   & 4,8E+06 & 4,0   & 7,2E+06 \\
L5    & 3,1E+04 & 5,1E+07 & 0,6   & 1,6E+06 & -0,2  & 1,6E+06 \\
L6    & 3,2E+04 & 2,6E+07 & 1,0   & 1,2E+06 & 0,9   & 1,2E+06 \\
\end{longtable}
\end{center}

% \begin{figure}[!h]
% \centering
% \includegraphics[scale=0.69]{fig/res/estatisticasMetXIV03.png} 
% \caption[Metodologia XIV: Variância do conjunto A]{Gráfico com
% comparativo da variância original do dado e dos resíduos gerados pelos modelos
% sem e com GARCH do conjunto A na Metodologia XIV}
% \label{Figura:estatisticaAMet14}
% \end{figure}
% 
% \begin{figure}[!h]
% \centering
% \includegraphics[scale=0.69]{fig/res/estatisticasMetXIV00.png} 
% \caption[Metodologia XIV: Variância do conjunto B]{Gráfico com
% comparativo da variância original do dado e dos resíduos gerados pelos modelos
% sem e com GARCH do conjunto B na Metodologia XIV}
% \label{Figura:estatisticaBMet14}
% \end{figure}
% 
% \begin{figure}[!h]
% \centering
% \includegraphics[scale=0.69]{fig/res/estatisticasMetXIV01.png} 
% \caption[Metodologia XIV: Variância do conjunto C]{Gráfico com
% comparativo da variância original do dado e dos resíduos gerados pelos modelos
% sem e com GARCH do conjunto C na Metodologia XIV}
% \label{Figura:estatisticaCMet14}
% \end{figure}
% 
% \begin{figure}[!h]
% \centering
% \includegraphics[scale=0.69]{fig/res/estatisticasMetXIV02.png} 
% \caption[Metodologia XIV: Variância dos conjuntos D e E]{Gráfico com comparativo
% da variância original do dado e dos resíduos gerados pelos modelos sem e com
% GARCH dos conjuntos D e E na Metodologia XIV}
% \label{Figura:estatisticaDEMet14}
% \end{figure}
% 
% \begin{figure}[!h]
% \centering
% \includegraphics[scale=0.69]{fig/res/estatisticasMetXIV04.png} 
% \caption[Metodologia XIV: Variância do conjunto F]{Gráfico com
% comparativo da variância original do dado e dos resíduos gerados pelos modelos
% sem e com GARCH do conjunto F na Metodologia XIV}
% \label{Figura:estatisticaFMet14}
% \end{figure}
% 
% \begin{figure}[!h]
% \centering
% \includegraphics[scale=0.8, angle=90]{fig/res/estatisticasMetXIV05.png} 
% \caption[Metodologia XIV: Variância dos conjuntos G, H e I]{Gráfico com
% comparativo da variância original do dado e dos resíduos gerados pelos modelos
% sem e com GARCH dos conjuntos G, H e I na Metodologia XIV}
% \label{Figura:estatisticaGHIMet14}
% \end{figure}
% 
% \begin{figure}[!h]
% \centering
% \includegraphics[scale=0.8, angle=90]{fig/res/estatisticasMetXIV06.png} 
% \caption[Metodologia XIV: Variância dos conjuntos J e K]{Gráfico
% com comparativo da variância original do dado e dos resíduos gerados pelos modelos
% sem e com GARCH dos conjuntos J e K na Metodologia XIV}
% \label{Figura:estatisticaJKMet14}
% \end{figure}
% 
% \begin{figure}[!h]
% \centering
% \includegraphics[scale=0.8, angle=90]{fig/res/estatisticasMetXIV07.png} 
% \caption[Metodologia XIV: Variância do conjunto L]{Gráfico com
% comparativo da variância original do dado e dos resíduos gerados pelos modelos
% sem e com GARCH do conjunto L na Metodologia XIV}
% \label{Figura:estatisticaLMet14}
% \end{figure}

% \begin{figure}[!h]
% \centering
% \includegraphics[scale=0.65]{fig/res/estatisticasMetXIV08.png} 
% \caption[Metodologia XIV: redução relativa da variância]{Gráfico com comparativo
% da redução relativa total da variância do resíduo sem e com a utilização do modelo GARCH na
% Metodologia XIV}
% \label{Figura:estatisticaPizzaMet14}
% \end{figure}