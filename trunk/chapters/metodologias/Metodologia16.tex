
\begin{center}
\begin{longtable}{cccccc}
\toprule
\rowcolor{white}
\caption[Metodologia XVI: comparativo de convergência de soluções]{Comparativo
   de quantidade de experimentos cujas soluções convergiram com e sem a
   utilização do GARCH na metodologia XVI} \label{Tab:convergenciaMet16} \\
\midrule
   Cenário & \specialcell{Total experimentos} & Convergiram &
   \specialcell{Não convergiram} & \% sucesso \\
\midrule
\endfirsthead
%\multicolumn{8}{c}%
%{\tablename\ \thetable\ -- \textit{Continuação da página anterior}} \\
\midrule
\rowcolor{white}
   Cenário & \specialcell{Total experimentos} & Convergiram &
   \specialcell{Não convergiram} & \% sucesso \\
\toprule
\endhead
\midrule \\ % \multicolumn{8}{r}{\textit{Continua na próxima página}} \\
\endfoot
\bottomrule
\endlastfoot
	Sem GARCH & 39 & 39 & 0 & 100\% \\
	Com GARCH & 39 & 39 & 0 & 100\% \\
\end{longtable}
\end{center}

%%%%%%%%%%%%%%%%%%%%%%%%%%%%%%%%%%%%%%%%%%%%%%%%%%%%%%%%%%%%%%%%%%%%%%%%%%%%%%%%%%%%%%%%%
\begin{center}
\begin{longtable}{cccccc}
\toprule
\rowcolor{white}
\caption[Metodologia XVI: Razão de compressão]{Razão de compressão dos
experimentos sem e com GARCH na Metodologia XVI.
Valores em bytes.} \label{Tab:razaocompressaoMet} \\
\midrule
Conjunto & \specialcell{Tamanho \\Original} & \specialcell{Tamanho
\\Comprimido\\Com GARCH} & \specialcell{Tamanho
\\Comprimido\\Sem GARCH} & \specialcell{Razão \\Compressão
\\Sem GARCH} & \specialcell{Razão \\Compressão
\\Com GARCH} \\
\midrule
\endfirsthead
%\multicolumn{8}{c}%
%{\tablename\ \thetable\ -- \textit{Continuação da página anterior}} \\
\midrule
\rowcolor{white}
Conjunto & \specialcell{Tamanho \\Original} & \specialcell{Tamanho
\\Comprimido\\Com GARCH} & \specialcell{Tamanho
\\Comprimido\\Sem GARCH} & \specialcell{Razão \\Compressão
\\Sem GARCH} & \specialcell{Razão \\Compressão
\\Com GARCH} \\
\toprule
\endhead
\midrule \\ % \multicolumn{8}{r}{\textit{Continua na próxima página}} \\
\endfoot
\bottomrule
\endlastfoot
    A1    & 1152000 & 867408 & 879718 & 1,33  & 1,31 \\
    A2    & 1152000 & 814667 & 815188 & 1,41  & 1,41 \\
    A3    & 1152000 & 850622 & 850531 & 1,35  & 1,35 \\
    B1    & 518592 & 170570 & 186837 & 3,04  & 2,78 \\
    B2    & 518592 & 170570 & 186837 & 3,04  & 2,78 \\
    B3    & 518592 & 170570 & 186837 & 3,04  & 2,78 \\
    C1    & 288192 & 260660 & 256221 & 1,11  & 1,12 \\
    C2    & 288192 & 258546 & 250990 & 1,11  & 1,15 \\
    C3    & 288192 & 261423 & 259909 & 1,10  & 1,11 \\
    D1    & 331200 & 263553 & 283959 & 1,26  & 1,17 \\
    D2    & 331200 & 256716 & 264775 & 1,29  & 1,25 \\
    D3    & 331200 & 249241 & 245918 & 1,33  & 1,35 \\
    E1    & 33792 & 32114 & 32177 & 1,05  & 1,05 \\
    E2    & 33792 & 32273 & 32355 & 1,05  & 1,04 \\
    E3    & 33792 & 32603 & 32618 & 1,04  & 1,04 \\
    F1    & 220992 & 188102 & 190070 & 1,17  & 1,16 \\
    F2    & 220992 & 142583 & 140579 & 1,55  & 1,57 \\
    F3    & 220992 & 145529 & 142726 & 1,52  & 1,55 \\
    G1    & 139392 & 112252 & 104455 & 1,24  & 1,33 \\
    G2    & 139392 & 108524 & 103353 & 1,28  & 1,35 \\
    G3    & 139392 & 111444 & 106819 & 1,25  & 1,30 \\
    H1    & 360192 & 329971 & 330046 & 1,09  & 1,09 \\
    H2    & 360192 & 293597 & 297422 & 1,23  & 1,21 \\
    H3    & 360192 & 299588 & 300597 & 1,20  & 1,20 \\
    I1    & 221184 & 172092 & 171086 & 1,29  & 1,29 \\
    I2    & 221184 & 146823 & 146257 & 1,51  & 1,51 \\
    I3    & 221184 & 181616 & 195423 & 1,22  & 1,13 \\
    J1    & 591936 & 411132 & 387066 & 1,44  & 1,53 \\
    J2    & 591936 & 425522 & 389503 & 1,39  & 1,52 \\
    J3    & 591936 & 424626 & 415058 & 1,39  & 1,43 \\
    K1    & 288000 & 199116 & 200196 & 1,45  & 1,44 \\
    K2    & 288000 & 204687 & 198532 & 1,41  & 1,45 \\
    K3    & 288000 & 195925 & 198093 & 1,47  & 1,45 \\
    L1    & 480192 & 396839 & 404088 & 1,21  & 1,19 \\
    L2    & 480192 & 408448 & 407100 & 1,18  & 1,18 \\
    L3    & 480192 & 416375 & 415763 & 1,15  & 1,15 \\
    L4    & 480192 & 412280 & 393637 & 1,16  & 1,22 \\
    L5    & 480192 & 378389 & 377245 & 1,27  & 1,27 \\
    L6    & 480192 & 377208 & 373274 & 1,27  & 1,29 \\

\end{longtable}
\end{center}

% \begin{figure}[!h]
% \centering
% \includegraphics[scale=1, angle=90]{fig/res/razaocompMetXVI00.png}
% \caption[Metodologia XVI: razão de compressão dos conjuntos A, B e C]{Gráfico
% com comparativo da razão de compressão dos conjuntos A, B e C sem e com GARCH na
% Metodologia XVI}
% \label{Figura:razaocompressaoABCMet16}
% \end{figure}
%  
% \begin{figure}[!h]
% \centering
% \includegraphics[scale=1, angle=90]{fig/res/razaocompMetXVI01.png}
% \caption[Metodologia XVI: razão de compressão dos conjuntos D, E e F]{Gráfico
% com comparativo da razão de compressão dos conjuntos D, E e F sem e com GARCH na
% Metodologia XVI}
% \label{Figura:razaocompressaoDEFMet16}
% \end{figure}
% 
% \begin{figure}[!h]
% \centering
% \includegraphics[scale=1, angle=90]{fig/res/razaocompMetXVI02.png}
% \caption[Metodologia XVI: razão de compressão dos conjuntos G, H e I]{Gráfico
% com comparativo da razão de compressão dos conjuntos G, H e I sem e com GARCH na
% Metodologia XVI}
% \label{Figura:razaocompressaoGHIMet16}
% \end{figure}
% 
% \begin{figure}[!h]
% \centering
% \includegraphics[scale=1, angle=90]{fig/res/razaocompMetXVI03.png}
% \caption[Metodologia XVI: razão de compressão dos conjuntos J, K e L]{Gráfico
% com comparativo da razão de compressão dos conjuntos J, K e L sem e com GARCH na
% Metodologia XVI}
% \label{Figura:razaocompressaoJKLMet16}
% \end{figure}

% \begin{figure}[!h]
% \centering
% \includegraphics[scale=0.9]{fig/res/razaocompMetXVI04.png}
% \caption[Metodologia XVI: razão de compressão]{Gráfico com comparativo da razão
% de compressão na Metodologia XVI}
% \label{Figura:razaocompressaoPizzaMet16}
% \end{figure}

\clearpage

\begin{center}
\begin{longtable}{cccc}
\toprule
\rowcolor{white}
\caption[Metodologia XVI: evolução da entropia]{Evolução da entropia do dado
original e do resíduo calculado na metodologia XVI}
\label{tab:EvolucaoEntropiaMet16}\\
\midrule
Conjunto & \specialcell{Entropia \\Inicial} & \specialcell{Entropia do
\\Resíduo sem GARC} & \specialcell{Entropia do
\\Resíduo com GARC}  \\
\midrule
\endfirsthead
%\multicolumn{8}{c}%
%{\tablename\ \thetable\ -- \textit{Continuação da página anterior}} \\
\midrule
\rowcolor{white}
Conjunto & \specialcell{Entropia \\Inicial} & \specialcell{Entropia do
\\Resíduo sem GARC} & \specialcell{Entropia do
\\Resíduo com GARC}  \\
\toprule
\endhead
\midrule \\ % \multicolumn{8}{r}{\textit{Continua na próxima página}} \\
\endfoot
\bottomrule 
\endlastfoot
    A1    & 11,30 & 11,14 & 11,17 \\
    A2    & 11,30 & 10,62 & 10,62 \\
    A3    & 11,27 & 10,77 & 10,77 \\
    B1    & 3,98  & 3,98  & 3,29 \\
    B2    & 3,98  & 3,98  & 3,29 \\
    B3    & 3,98  & 3,98  & 3,29 \\
    C1    & 12,33 & 12,32 & 12,34 \\
    C2    & 12,67 & 12,67 & 12,60 \\
    C3    & 12,71 & 12,71 & 12,76 \\
    D1    & 9,93  & 9,90  & 9,68 \\
    D2    & 11,25 & 11,24 & 11,01 \\
    D3    & 8,29  & 8,29  & 6,56 \\
    E1    & 10,80 & 10,80 & 10,80 \\
    E2    & 10,78 & 10,78 & 10,78 \\
    E3    & 10,80 & 10,80 & 10,80 \\
    F1    & 10,81 & 10,81 & 10,59 \\
    F2    & 7,79  & 7,79  & 7,64 \\
    F3    & 9,19  & 9,19  & 8,38 \\
    G1    & 11,35 & 11,34 & 11,10 \\
    G2    & 10,98 & 10,97 & 10,54 \\
    G3    & 11,49 & 11,47 & 11,22 \\
    H1    & 8,44  & 8,44  & 8,44 \\
    H2    & 12,29 & 12,24 & 12,26 \\
    H3    & 12,33 & 12,20 & 12,27 \\
    I1    & 8,01  & 8,01  & 8,18 \\
    I2    & 7,39  & 7,11  & 7,14 \\
    I3    & 8,43  & 8,43  & 8,41 \\
    J1    & 10,00 & 10,00 & 7,49 \\
    J2    & 10,15 & 10,15 & 7,37 \\
    J3    & 7,86  & 7,86  & 7,22 \\
    K1    & 9,99  & 9,99  & 9,97 \\
    K2    & 9,98  & 9,98  & 9,85 \\
    K3    & 10,05 & 10,05 & 9,96 \\
    L1    & 12,38 & 12,38 & 12,33 \\
    L2    & 11,08 & 11,08 & 11,08 \\
    L3    & 12,98 & 12,97 & 12,98 \\
    L4    & 12,80 & 12,32 & 11,94 \\
    L5    & 11,52 & 11,52 & 10,67 \\
    L6    & 11,58 & 11,58 & 11,58 \\


\end{longtable}
\end{center}

% \begin{figure}[!h]
% \centering
% \includegraphics[scale=0.8, angle=90]{fig/res/evolucaoentropiaMetXVI00.png} 
% \caption[Metodologia XVI: evolução da entropia nos conjuntos A, B e C]{Gráfico
% com comparativo da evolução da entropia dos conjuntos A, B e C sem e com GARCH na
% Metodologia XVI}
% \label{Figura:evolucaoentropiaABCMet16}
% \end{figure}
% 
% \begin{figure}[!h]
% \centering
% \includegraphics[scale=0.8, angle=90]{fig/res/evolucaoentropiaMetXVI01.png} 
% \caption[Metodologia XVI: evolução da entropia nos conjuntos D, E e F]{Gráfico
% com comparativo da evolução da entropia dos conjuntos D, E e F sem e com GARCH na
% Metodologia XVI}
% \label{Figura:evolucaoentropiaDEFMet16}
% \end{figure}
% 
% \begin{figure}[!h]
% \centering
% \includegraphics[scale=0.8, angle=90]{fig/res/evolucaoentropiaMetXVI02.png} 
% \caption[Metodologia XVI: evolução da entropia nos conjuntos G, H e I]{Gráfico
% com comparativo da evolução da entropia dos conjuntos G, H e I sem e com GARCH na
% Metodologia XVI}
% \label{Figura:evolucaoentropiaGHIMet16}
% \end{figure}
% 
% \begin{figure}[!h]
% \centering
% \includegraphics[scale=0.6]{fig/res/evolucaoentropiaMetXVI03.png} 
% \caption[Metodologia XVI: evolução da entropia nos conjuntos J e K]{Gráfico com
% comparativo da evolução da entropia dos conjuntos J e K sem e com GARCH na
% Metodologia XVI}
% \label{Figura:evolucaoentropiaJKMet16}
% \end{figure}
% 
% \begin{figure}[!h]
% \centering
% \includegraphics[scale=0.6]{fig/res/evolucaoentropiaMetXVI04.png} 
% \caption[Metodologia XVI: evolução da entropia nos conjuntos L]{Gráfico com
% comparativo da evolução da entropia dos conjuntos L sem e com GARCH na
% Metodologia XVI}
% \label{Figura:evolucaoentropiaLMet16}
% \end{figure}

% \begin{figure}[!h]
% \centering
% \includegraphics[scale=1]{fig/res/evolucaoentropiaMetXVI05.png} 
% \caption[Metodologia XVI: evolução da entropia]{Gráfico com comparativo da
% evolução da entopia na Metodologia XVI}
% \label{Figura:evolucaoentropiaPizzaMet16}
% \end{figure}

\clearpage

\begin{center}
\begin{longtable}{ccccc|cccc}
\toprule
\rowcolor{white}
\caption[Metodologia XVI: tempo de execução]{Tempo de execução (em segundos)
dos algoritmos sem e com GARCH na Metodologia XVI. Primeiro é exibido o tempo de
execução sem a utilização do modelo GARCH, depois com o modelo. Parâmetros
modelo se refere ao tempo gasto pelo algoritmo para o cálculo dos parâmetros do
modelo, Resíduo refere-se ao tempo gasto pelo modelo para calcular o resíduo do
modelo, Cod. Arit. refere-se ao tempo gasto pela codificação aritmética para
comprimir o resíduo.} \label{tab:EvolucaoEntropiaMet16}\\
\midrule
Conj & \specialcell{Parâmetros\\modelo} &
Resíduo & \specialcell{Cod.\\Arit.} & \specialcell{Tempo\\total} &
\specialcell{Parâmetros\\modelo} &
Resíduo & \specialcell{Cod.\\Arit.} & \specialcell{Tempo\\total} \\
\midrule
\endfirsthead 
%\multicolumn{8}{c}%
%{\tablename\ \thetable\ -- \textit{Continuação da página anterior}} \\
\midrule
\rowcolor{white}
Conj & \specialcell{Parâmetros\\modelo} &
Resíduo & \specialcell{Cod.\\Arit.} & \specialcell{Tempo\\total} &
\specialcell{Parâmetros\\modelo} &
Resíduo & \specialcell{Cod.\\Arit.} & \specialcell{Tempo\\total} \\
\toprule
\endhead
\midrule \\ % \multicolumn{8}{r}{\textit{Continua na próxima página}} \\
\endfoot
\bottomrule 
\endlastfoot
A1&406&2&5&412&703&1&3&706\\
A2&124&2&8&135&774&1&6&782\\
A3&132&2&6&140&863&3&7&873\\
B1&77&$<1$&1&78&345&2&1&347\\
B2&76&1&1&77&343&$<1$&1&344\\
B3&77&$<1$&1&78&339&$<1$&1&340\\
C1&56&$<1$&1&57&123&$<1$&1&124\\
C2&57&$<1$&2&59&120&$<1$&1&121\\
C3&51&$<1$&2&52&114&$<1$&2&116\\
D1&45&$<1$&1&46&206&$<1$&1&207\\
D2&135&1&2&138&368&1&1&371\\
D3&125&$<1$&1&126&105&1&2&108\\
E1&8&$<1$&$<1$&8&34&$<1$&$<1$&34\\
E2&5&$<1$&$<1$&6&38&$<1$&$<1$&38\\
E3&10&$<1$&$<1$&10&40&$<1$&$<1$&41\\
F1&25&$<1$&1&26&131&$<1$&1&132\\
F2&74&$<1$&1&75&256&$<1$&$<1$&257\\
F3&25&$<1$&2&27&98&1&1&99\\
G1&55&$<1$&$<1$&55&142&1&1&144\\
G2&40&$<1$&$<1$&41&179&$<1$&1&181\\
G3&60&$<1$&$<1$&60&76&$<1$&$<1$&77\\
H1&71&1&3&75&219&$<1$&1&220\\
H2&39&1&3&43&407&$<1$&1&408\\
H3&37&$<1$&1&38&354&1&2&356\\
I1&30&$<1$&2&32&263&$<1$&1&264\\
I2&55&$<1$&$<1$&55&201&$<1$&$<1$&201\\
I3&11&$<1$&$<1$&12&111&$<1$&$<1$&112\\
J1&137&1&1&138&319&1&2&322\\
J2&237&$<1$&3&240&455&$<1$&1&456\\
J3&74&1&4&79&636&1&1&637\\
K1&30&1&2&33&335&$<1$&2&337\\
K2&31&$<1$&1&31&334&1&2&336\\
K3&96&1&2&98&259&$<1$&1&260\\
L1&197&$<1$&3&200&527&1&2&530\\
L2&75&1&4&80&189&2&4&195\\
L3&117&1&2&120&521&$<1$&1&523\\
L4&89&1&4&94&202&1&3&206\\
L5&72&$<1$&3&75&419&1&2&423\\
L6&181&1&5&186&193&1&4&199\\
\end{longtable}
\end{center}

% \begin{figure}[!h]
% \centering
% \includegraphics[scale=1, angle=90]{fig/res/tempoexecMetXVI00.png} 
% \caption[Metodologia XVI: tempo de cálculo dos parâmetros dos modelos dos
% conjuntos A, B, C e D]{Gráfico com comparativo do tempo de cálculo dos
% parâmetros dos modelos dos conjuntos A, B, C e D sem e com GARCH na Metodologia
% XVI}
% \label{Figura:tempocalculoABCDMet16}
% \end{figure}
% 
% \begin{figure}[!h]
% \centering
% \includegraphics[scale=0.75]{fig/res/tempoexecMetXVI01.png} 
% \caption[Metodologia XVI: tempo de cálculo dos parâmetros dos modelos dos
% conjuntos E, F e G]{Gráfico com comparativo do tempo de cálculo dos
% parâmetros dos modelos dos conjuntos E, F e G sem e com GARCH na Metodologia
% XVI}
% \label{Figura:tempocalculoEFGMet16}
% \end{figure}
% 
% \begin{figure}[!h]
% \centering
% \includegraphics[scale=0.75]{fig/res/tempoexecMetXVI02.png} 
% \caption[Metodologia XVI: tempo de cálculo dos parâmetros dos modelos dos
% conjuntos H, I e J]{Gráfico com comparativo do tempo de cálculo dos
% parâmetros dos modelos dos conjuntos H, I e J sem e com GARCH na Metodologia
% XVI}
% \label{Figura:tempocalculoHIJMet16}
% \end{figure}
% 
% \begin{figure}[!h]
% \centering 
% \includegraphics[scale=1, angle=90]{fig/res/tempoexecMetXVI03.png} 
% \caption[Metodologia XVI: tempo de cálculo dos parâmetros dos modelos dos
% conjuntos K e L]{Gráfico com comparativo do tempo de cálculo dos
% parâmetros dos modelos dos conjuntos K e L sem e com GARCH na Metodologia XVI}
% \label{Figura:tempocalculoKLMet16}
% \end{figure}

% \begin{figure}[!h]
% \centering
% \includegraphics[scale=0.75]{fig/res/tempoexecMetXVI04.png} 
% \caption[Metodologia XVI: tempo total relativo gasto no cálculo dos
% parâmetros do modelo]{Gráfico com comparativo do tempo total relativo de cálculo
% dos parâmetros dos modelos sem e com GARCH na Metodologia XVI}
% \label{Figura:tempocalculoPizzaMet16}
% \end{figure}

\clearpage

\begin{center}
\begin{longtable}{ccccc|cccc}
\toprule
\rowcolor{white}
\caption[Metodologia XVI: evolução da autocorrelação]{Autocorrelação do dado
original e dos resíduos gerados sem e com a utilização do modelo GARCH na
Metodologia XVI} \label{tab:EvolucaoAutocorrelacaoMet16}\\
\midrule
Conjunto & \specialcell{Autocorrelação\\Inicial} & \specialcell{Autocorrelação\\Sem
GARCH} & \specialcell{Autocorrelação\\Com GARCH} \\
\midrule
\endfirsthead 
%\multicolumn{8}{c}%
%{\tablename\ \thetable\ -- \textit{Continuação da página anterior}} \\
\midrule
\rowcolor{white}
Conjunto & \specialcell{Autocorrelação\\Inicial} & \specialcell{Autocorrelação\\Sem
GARCH} & \specialcell{Autocorrelação\\Com GARCH} \\
\toprule
\endhead
\midrule \\ % \multicolumn{8}{r}{\textit{Continua na próxima página}} \\
\endfoot
\bottomrule 
\endlastfoot
A1    & 6     & 0     & 0 \\
A2    & 5     & 0     & 0 \\
A3    & 6     & 0     & 0 \\
B1    & 6     & 0     & 3 \\
B2    & 6     & 0     & 3 \\
B3    & 6     & 0     & 3 \\
C1    & 2     & 0     & 0 \\
C2    & 1     & 0     & 1 \\
C3    & 2     & 0     & 0 \\
D1    & 2     & 0     & 21 \\
D2    & 2     & 0     & 12 \\
D3    & 2     & 0     & 0 \\
E1    & 4     & 0     & 0 \\
E2    & 4     & 0     & 0 \\
E3    & 4     & 0     & 0 \\
F1    & 1     & 0     & 14 \\
F2    & 6     & 0     & 0 \\
F3    & 6     & 0     & 0 \\
G1    & 1     & 0     & 51 \\
G2    & 2     & 0     & 6 \\
G3    & 6     & 0     & 49 \\
H1    & 1     & 0     & 0 \\
H2    & 1     & 0     & 0 \\
H3    & 1     & 0     & 0 \\
I1    & 7     & 0     & 0 \\
I2    & 1     & 0     & 0 \\
I3    & 1     & 0     & 0 \\
J1    & 3     & 0     & 0 \\
J2    & 3     & 0     & 0 \\
J3    & 8     & 0     & 0 \\
K1    & 3     & 0     & 3 \\
K2    & 3     & 0     & 3 \\
K3    & 2     & 0     & 3 \\
L1    & 2     & 0     & 5 \\
L2    & 7     & 0     & 2 \\
L3    & 11    & 0     & 0 \\
L4    & 7     & 0     & 1 \\
L5    & 7     & 0     & 0 \\
L6    & 6     & 0     & 0 \\

\end{longtable}
\end{center}

% \begin{figure}[!h]
% \centering
% \includegraphics[scale=0.75]{fig/res/evolucaoautocorrMetXVI00.png} 
% \caption[Metodologia XVI: evolução da autocorrelação nos conjuntos A, B e
% C]{Gráfico com comparativo da autocorrelação do resíduo gerado sem e com a
% utilização do modelo GARCH em relação ao dado original nos conjuntos A, B e C na
% Metodologia XVI}
% \label{Figura:autocorrelacaoABCMet16}
% \end{figure}
% 
% \begin{figure}[!h]
% \centering
% \includegraphics[scale=0.69]{fig/res/evolucaoautocorrMetXVI01.png} 
% \caption[Metodologia XVI: evolução da autocorrelação nos conjuntos D, E e
% F]{Gráfico com comparativo da autocorrelação do resíduo gerado sem e com a
% utilização do modelo GARCH em relação ao dado original nos conjuntos D, E e F na
% Metodologia XVI}
% \label{Figura:autocorrelacaoDEFMet16}
% \end{figure}
% 
% \begin{figure}[!h]
% \centering
% \includegraphics[scale=0.69]{fig/res/evolucaoautocorrMetXVI02.png} 
% \caption[Metodologia XVI: evolução da autocorrelação nos conjuntos G, H e
% I]{Gráfico com comparativo da autocorrelação do resíduo gerado sem e com a
% utilização do modelo GARCH em relação ao dado original nos conjuntos G, H e I na
% Metodologia XVI}
% \label{Figura:autocorrelacaoGHIMet16}
% \end{figure}
% 
% \begin{figure}[!h]
% \centering
% \includegraphics[scale=0.69]{fig/res/evolucaoautocorrMetXVI03.png} 
% \caption[Metodologia XVI: evolução da autocorrelação nos conjuntos J e
% K]{Gráfico com comparativo da autocorrelação do resíduo gerado sem e com a
% utilização do modelo GARCH em relação ao dado original nos conjuntos J e K na
% Metodologia XVI}
% \label{Figura:autocorrelacaoJKMet16}
% \end{figure}
% 
% \begin{figure}[!h]
% \centering
% \includegraphics[scale=0.69]{fig/res/evolucaoautocorrMetXVI04.png} 
% \caption[Metodologia XVI: evolução da autocorrelação nos conjuntos L]{Gráfico
% com comparativo da autocorrelação do resíduo gerado sem e com a utilização do modelo GARCH em relação ao dado original nos conjuntos L na
% Metodologia XVI}
% \label{Figura:autocorrelacaoLMet16}
% \end{figure}
% 
% \begin{figure}[!h]
% \centering
% \includegraphics[scale=0.75]{fig/res/evolucaoautocorrMetXVI05.png} 
% \caption[Metodologia XVI: tempo total relativo gasto no cálculo dos
% parâmetros do modelo]{Gráfico com comparativo da redução relativa total da
% autocorrelação do resíduo sem e com a utilização do modelo GARCH na
% Metodologia XVI}
% \label{Figura:tempocalculoPizzaMet16}
% \end{figure}

\clearpage

\begin{center}
\begin{longtable}{ccccccccc}
\toprule
\rowcolor{white}
\caption[Metodologia XVI: dados estatísticos]{Média e variância do dado original
comparadas às do resíduo calculado sem e com a utilização do modelo GARCH na
Metodologia XVI} \label{tab:DadosEstatisticosMet16}\\
\midrule
    Conjunto & \specialcell{Média\\Original} &
    \specialcell{Var.\\Original} & \specialcell{Média\\Sem\\GARCH} &
    \specialcell{Var.\\Sem\\GARCH} & \specialcell{Média\\Com\\GARCH}&
    \specialcell{Var.\\Com\\GARCH} \\

\midrule
\endfirsthead 
%\multicolumn{8}{c}%
%{\tablename\ \thetable\ -- \textit{Continuação da página anterior}} \\
\midrule
\rowcolor{white}
    Conjunto & \specialcell{Média\\Orig.} &
    \specialcell{Var.\\Orig.} & \specialcell{Média\\Sem\\GARCH} &
    \specialcell{Var.\\Sem\\GARCH} & \specialcell{Média\\Com\\GARCH}&
    \specialcell{Var.\\Com\\GARCH} \\

\toprule
\endhead
\midrule \\ % \multicolumn{8}{r}{\textit{Continua na próxima página}} \\
\endfoot
\bottomrule 
\endlastfoot
A1    & 3,0E+04 & 1,8E+07 & 0,5   & 3,7E+05 & 0,5   & 4,2E+05 \\
A2    & 3,2E+04 & 1,1E+07 & 0,4   & 3,0E+05 & -1,1  & 3,0E+05 \\
A3    & 3,1E+04 & 1,4E+07 & 0,4   & 3,2E+05 & -0,4  & 3,2E+05 \\
B1    & 2,8E+04 & 4,5E+05 & 0,5   & 3,8E+01 & 0,5   & 2,6E+02 \\
B2    & 2,8E+04 & 4,5E+05 & 0,5   & 3,8E+01 & 0,5   & 2,6E+02 \\
B3    & 2,8E+04 & 4,5E+05 & 0,5   & 3,8E+01 & 0,5   & 2,6E+02 \\
C1    & 3,3E+04 & 8,1E+07 & 0,6   & 1,9E+07 & 0,2   & 2,2E+07 \\
C2    & 3,3E+04 & 4,0E+07 & 0,2   & 1,3E+07 & 2,8   & 1,7E+07 \\
C3    & 3,3E+04 & 5,7E+07 & 0,5   & 1,1E+07 & 0,8   & 1,3E+07 \\
D1    & 3,7E+04 & 4,1E+07 & 0,7   & 4,0E+06 & 1,1   & 1,2E+07 \\
D2    & 3,3E+04 & 1,2E+07 & 1,6   & 1,6E+06 & 0,5   & 1,3E+07 \\
D3    & 3,1E+04 & 1,0E+07 & -1,2  & 1,1E+06 & -1,0  & 1,3E+06 \\
E1    & 2,9E+04 & 5,8E+07 & 0,6   & 5,1E+07 & -4,7  & 5,1E+07 \\
E2    & 3,0E+04 & 5,8E+07 & 0,5   & 5,2E+07 & 0,6   & 5,3E+07 \\
E3    & 3,0E+04 & 6,0E+07 & 0,9   & 5,5E+07 & 0,8   & 5,6E+07 \\
F1    & 3,8E+04 & 3,9E+07 & 0,3   & 5,9E+06 & -3,1  & 8,2E+06 \\
F2    & 2,3E+04 & 5,4E+06 & 0,5   & 5,7E+05 & 1,4   & 5,4E+05 \\
F3    & 2,6E+04 & 6,0E+06 & 0,4   & 3,6E+05 & 4,3   & 2,6E+05 \\
G1    & 3,3E+04 & 3,3E+07 & 1,1   & 9,8E+05 & -0,8  & 1,1E+06 \\
G2    & 3,8E+04 & 1,9E+07 & 1,3   & 5,8E+05 & 2,7   & 4,7E+05 \\
G3    & 2,9E+04 & 3,6E+07 & -14,7 & 1,5E+06 & -0,9  & 3,8E+06 \\
H1    & 3,1E+04 & 3,6E+07 & -2,7  & 3,5E+07 & 38,2  & 3,5E+07 \\
H2    & 3,4E+04 & 8,1E+06 & 0,3   & 5,9E+06 & -23,3 & 4,7E+06 \\
H3    & 3,2E+04 & 7,3E+06 & 0,6   & 4,9E+06 & -0,4  & 5,1E+06 \\
I1    & 3,6E+04 & 1,2E+07 & 0,3   & 1,8E+06 & 0,2   & 3,5E+06 \\
I2    & 2,9E+04 & 1,2E+06 & 0,5   & 2,3E+05 & 0,1   & 2,8E+05 \\
I3    & 3,1E+04 & 3,3E+07 & 0,5   & 6,6E+06 & 0,4   & 3,8E+07 \\
J1    & 3,7E+04 & 1,2E+06 & 2,9   & 4,5E+05 & -1,2  & 9,4E+05 \\
J2    & 3,5E+04 & 1,5E+06 & -36,1 & 5,8E+05 & 0,3   & 7,4E+05 \\
J3    & 3,3E+04 & 1,3E+06 & 0,5   & 5,2E+05 & 0,0   & 5,8E+05 \\
K1    & 3,9E+04 & 6,9E+06 & 0,5   & 5,6E+05 & 1,1   & 4,4E+05 \\
K2    & 4,0E+04 & 6,7E+06 & 0,4   & 5,0E+05 & 2,5   & 3,8E+05 \\
K3    & 3,6E+04 & 5,8E+06 & 3,8   & 3,9E+05 & 0,2   & 4,8E+05 \\
L1    & 3,4E+04 & 2,9E+07 & 192,1 & 5,3E+06 & -9,6  & 5,7E+06 \\
L2    & 3,1E+04 & 1,5E+07 & 0,7   & 5,2E+06 & 0,7   & 5,4E+06 \\
L3    & 3,5E+04 & 1,3E+07 & 14,7  & 7,5E+06 & 12,0  & 7,6E+06 \\
L4    & 3,7E+04 & 1,8E+07 & 8,3   & 4,8E+06 & 4,0   & 7,2E+06 \\
L5    & 3,1E+04 & 5,1E+07 & -4,7  & 1,6E+06 & -0,2  & 1,6E+06 \\
L6    & 3,2E+04 & 2,6E+07 & 1,0   & 1,2E+06 & 0,9   & 1,2E+06 \\
\end{longtable}
\end{center}

% \begin{figure}[!h]
% \centering
% \includegraphics[scale=0.69]{fig/res/estatisticasMetXVI03.png} 
% \caption[Metodologia XVI: Variância do conjunto A]{Gráfico com
% comparativo da variância original do dado e dos resíduos gerados pelos modelos
% sem e com GARCH do conjunto A na Metodologia XVI}
% \label{Figura:estatisticaAMet16}
% \end{figure}
% 
% \begin{figure}[!h]
% \centering
% \includegraphics[scale=0.69]{fig/res/estatisticasMetXVI00.png} 
% \caption[Metodologia XVI: Variância do conjunto B]{Gráfico com
% comparativo da variância original do dado e dos resíduos gerados pelos modelos
% sem e com GARCH do conjunto B na Metodologia XVI}
% \label{Figura:estatisticaBMet16}
% \end{figure}
% 
% \begin{figure}[!h]
% \centering
% \includegraphics[scale=0.69]{fig/res/estatisticasMetXVI01.png} 
% \caption[Metodologia XVI: Variância do conjunto C]{Gráfico com
% comparativo da variância original do dado e dos resíduos gerados pelos modelos
% sem e com GARCH do conjunto C na Metodologia XVI}
% \label{Figura:estatisticaCMet16}
% \end{figure}
% 
% \begin{figure}[!h]
% \centering
% \includegraphics[scale=0.69]{fig/res/estatisticasMetXVI02.png} 
% \caption[Metodologia XVI: Variância dos conjuntos D e E]{Gráfico com comparativo
% da variância original do dado e dos resíduos gerados pelos modelos sem e com
% GARCH dos conjuntos D e E na Metodologia XVI}
% \label{Figura:estatisticaDEMet16}
% \end{figure}
% 
% \begin{figure}[!h]
% \centering
% \includegraphics[scale=0.69]{fig/res/estatisticasMetXVI04.png} 
% \caption[Metodologia XVI: Variância do conjunto F]{Gráfico com
% comparativo da variância original do dado e dos resíduos gerados pelos modelos
% sem e com GARCH do conjunto F na Metodologia XVI}
% \label{Figura:estatisticaFMet16}
% \end{figure}
% 
% \begin{figure}[!h]
% \centering
% \includegraphics[scale=0.8, angle=90]{fig/res/estatisticasMetXVI05.png} 
% \caption[Metodologia XVI: Variância dos conjuntos G, H e I]{Gráfico com
% comparativo da variância original do dado e dos resíduos gerados pelos modelos
% sem e com GARCH dos conjuntos G, H e I na Metodologia XVI}
% \label{Figura:estatisticaGHIMet16}
% \end{figure}
% 
% \begin{figure}[!h]
% \centering
% \includegraphics[scale=0.8, angle=90]{fig/res/estatisticasMetXVI06.png} 
% \caption[Metodologia XVI: Variância dos conjuntos J e K]{Gráfico
% com comparativo da variância original do dado e dos resíduos gerados pelos modelos
% sem e com GARCH dos conjuntos J e K na Metodologia XVI}
% \label{Figura:estatisticaJKMet16}
% \end{figure}
% 
% \begin{figure}[!h]
% \centering
% \includegraphics[scale=0.8, angle=90]{fig/res/estatisticasMetXVI07.png} 
% \caption[Metodologia XVI: Variância do conjunto L]{Gráfico com
% comparativo da variância original do dado e dos resíduos gerados pelos modelos
% sem e com GARCH do conjunto L na Metodologia XVI}
% \label{Figura:estatisticaLMet16}
% \end{figure}

% \begin{figure}[!h]
% \centering
% \includegraphics[scale=0.65]{fig/res/estatisticasMetXVI08.png} 
% \caption[Metodologia XVI: redução relativa da variância]{Gráfico com comparativo
% da redução relativa total da variância do resíduo sem e com a utilização do modelo GARCH na
% Metodologia XVI}
% \label{Figura:estatisticaPizzaMet16}
% \end{figure}