
\begin{center}
\begin{longtable}{cccccc}
\toprule
\rowcolor{white}
\caption[Metodologia XVIII: comparativo de convergência de soluções]{Comparativo
   de quantidade de experimentos cujas soluções convergiram com e sem a
   utilização do GARCH na metodologia XVIII} \label{Tab:convergenciaMet18} \\
\midrule
   Cenário & \specialcell{Total experimentos} & Convergiram &
   \specialcell{Não convergiram} & \% sucesso \\
\midrule
\endfirsthead
%\multicolumn{8}{c}%
%{\tablename\ \thetable\ -- \textit{Continuação da página anterior}} \\
\midrule
\rowcolor{white}
   Cenário & \specialcell{Total experimentos} & Convergiram &
   \specialcell{Não convergiram} & \% sucesso \\
\toprule
\endhead
\midrule \\ % \multicolumn{8}{r}{\textit{Continua na próxima página}} \\
\endfoot
\bottomrule
\endlastfoot
	Sem GARCH & 39 & 39 & 0 & 100\% \\
	Com GARCH & 39 & 38 & 1 & 97,44\% \\
\end{longtable}
\end{center}

%%%%%%%%%%%%%%%%%%%%%%%%%%%%%%%%%%%%%%%%%%%%%%%%%%%%%%%%%%%%%%%%%%%%%%%%%%%%%%%%%%%%%%%%%
\begin{center}
\begin{longtable}{cccccc}
\toprule
\rowcolor{white}
\caption[Metodologia XVIII: Razão de compressão]{Razão de compressão dos
experimentos sem e com GARCH na Metodologia XVIII.
Valores em bytes.} \label{Tab:razaocompressaoMet} \\
\midrule
Conjunto & \specialcell{Tamanho \\Original} & \specialcell{Tamanho
\\Comprimido\\Com GARCH} & \specialcell{Tamanho
\\Comprimido\\Sem GARCH} & \specialcell{Razão \\Compressão
\\Sem GARCH} & \specialcell{Razão \\Compressão
\\Com GARCH} \\
\midrule
\endfirsthead
%\multicolumn{8}{c}%
%{\tablename\ \thetable\ -- \textit{Continuação da página anterior}} \\
\midrule
\rowcolor{white}
Conjunto & \specialcell{Tamanho \\Original} & \specialcell{Tamanho
\\Comprimido\\Com GARCH} & \specialcell{Tamanho
\\Comprimido\\Sem GARCH} & \specialcell{Razão \\Compressão
\\Sem GARCH} & \specialcell{Razão \\Compressão
\\Com GARCH} \\
\toprule
\endhead
\midrule \\ % \multicolumn{8}{r}{\textit{Continua na próxima página}} \\
\endfoot
\bottomrule
\endlastfoot
    A1    & 1152000 & 938716 & 946768 & 1,23  & 1,22 \\
    A2    & 1152000 & 882390 & 880373 & 1,31  & 1,31 \\
    A3    & 1152000 & 922165 & 922605 & 1,25  & 1,25 \\
    B1    & 518592 & 166690 & 201716 & 3,11  & 2,57 \\
    B2    & 518592 & 166690 & 201716 & 3,11  & 2,57 \\
    B3    & 518592 & 166690 & 201716 & 3,11  & 2,57 \\
    C1    & 288192 & 279763 & 274233 & 1,03  & 1,05 \\
    C2    & 288192 & 277364 & 269002 & 1,04  & 1,07 \\
    C3    & 288192 & 281342 & 277921 & 1,02  & 1,04 \\
    D1    & 331200 & 300169 & 299512 & 1,10  & 1,11 \\
    D2    & 331200 & 282243 & 282579 & 1,17  & 1,17 \\
    D3    & 331200 & 272175 & 268205 & 1,22  & 1,23 \\
    E1    & 33792 & 34226 & 34289 & 0,99  & 0,99 \\
    E2    & 33792 & 34385 & 34467 & 0,98  & 0,98 \\
    E3    & 33792 & 34715 & 34730 & 0,97  & 0,97 \\
    F1    & 220992 & 198188 & 198019 & 1,12  & 1,12 \\
    F2    & 220992 & 140753 & 157684 & 1,57  & 1,40 \\
    F3    & 220992 & 140759 & 137809 & 1,57  & 1,60 \\
    G1    & 139392 & 125888 & 125915 & 1,11  & 1,11 \\
    G2    & 139392 & 114785 & 114693 & 1,21  & 1,22 \\
    G3    & 139392 & 127534 & 127543 & 1,09  & 1,09 \\
    H1    & 360192 & 352483 & 352558 & 1,02  & 1,02 \\
    H2    & 360192 & 316109 & 318045 & 1,14  & 1,13 \\
    H3    & 360192 & 322100 & 321490 & 1,12  & 1,12 \\
    I1    & 221184 & 185758 & 186214 & 1,19  & 1,19 \\
    I2    & 221184 & 147492 & 144920 & 1,50  & 1,53 \\
    I3    & 221184 & 196326 & 195367 & 1,13  & 1,13 \\
    J1    & 591936 & 337779 & 352846 & 1,75  & 1,68 \\
    J2    & 591936 & 359470 & 388265 & 1,65  & 1,52 \\
    J3    & 591936 & 373108 & 465505 & 1,59  & 1,27 \\
    K1    & 288000 & 216968 & 217569 & 1,33  & 1,32 \\
    K2    & 288000 & 212118 & 220684 & 1,36  & 1,31 \\
    K3    & 288000 & 212083 & 213744 & 1,36  & 1,35 \\
    L1    & 480192 & 430980 & 427623 & 1,11  & 1,12 \\
    L2    & 480192 & 438460 & 437112 & 1,10  & 1,10 \\
    L3    & 480192 & 444677 & 443652 & 1,08  & 1,08 \\
    L4    & 480192 & 422552 & 427004 & 1,14  & 1,12 \\
    L5    & 480192 & 409320 & 408783 & 1,17  & 1,17 \\
    L6    & 480192 & 407696 & 405967 & 1,18  & 1,18 \\
\end{longtable}
\end{center}

% \begin{figure}[!h]
% \centering
% \includegraphics[scale=1, angle=90]{fig/res/razaocompMetXVIII00.png}
% \caption[Metodologia XVIII: razão de compressão dos conjuntos A, B e C]{Gráfico
% com comparativo da razão de compressão dos conjuntos A, B e C sem e com GARCH na
% Metodologia XVIII}
% \label{Figura:razaocompressaoABCMet18}
% \end{figure}
%  
% \begin{figure}[!h]
% \centering
% \includegraphics[scale=1, angle=90]{fig/res/razaocompMetXVIII01.png}
% \caption[Metodologia XVIII: razão de compressão dos conjuntos D, E e F]{Gráfico
% com comparativo da razão de compressão dos conjuntos D, E e F sem e com GARCH na
% Metodologia XVIII}
% \label{Figura:razaocompressaoDEFMet18}
% \end{figure}
% 
% \begin{figure}[!h]
% \centering
% \includegraphics[scale=1, angle=90]{fig/res/razaocompMetXVIII02.png}
% \caption[Metodologia XVIII: razão de compressão dos conjuntos G, H e I]{Gráfico
% com comparativo da razão de compressão dos conjuntos G, H e I sem e com GARCH na
% Metodologia XVIII}
% \label{Figura:razaocompressaoGHIMet18}
% \end{figure}
% 
% \begin{figure}[!h]
% \centering
% \includegraphics[scale=1, angle=90]{fig/res/razaocompMetXVIII03.png}
% \caption[Metodologia XVIII: razão de compressão dos conjuntos J, K e L]{Gráfico
% com comparativo da razão de compressão dos conjuntos J, K e L sem e com GARCH na
% Metodologia XVIII}
% \label{Figura:razaocompressaoJKLMet18}
% \end{figure}

% \begin{figure}[!h]
% \centering
% \includegraphics[scale=0.9]{fig/res/razaocompMetXVIII04.png}
% \caption[Metodologia XVIII: razão de compressão]{Gráfico com comparativo da razão
% de compressão na Metodologia XVIII}
% \label{Figura:razaocompressaoPizzaMet18}
% \end{figure}

\clearpage

\begin{center}
\begin{longtable}{cccc}
\toprule
\rowcolor{white}
\caption[Metodologia XVIII: evolução da entropia]{Evolução da entropia do dado
original e do resíduo calculado na metodologia XVIII}
\label{tab:EvolucaoEntropiaMet18}\\
\midrule
Conjunto & \specialcell{Entropia \\Inicial} & \specialcell{Entropia do
\\Resíduo sem GARC} & \specialcell{Entropia do
\\Resíduo com GARC}  \\
\midrule
\endfirsthead
%\multicolumn{8}{c}%
%{\tablename\ \thetable\ -- \textit{Continuação da página anterior}} \\
\midrule
\rowcolor{white}
Conjunto & \specialcell{Entropia \\Inicial} & \specialcell{Entropia do
\\Resíduo sem GARC} & \specialcell{Entropia do
\\Resíduo com GARC}  \\
\toprule
\endhead
\midrule \\ % \multicolumn{8}{r}{\textit{Continua na próxima página}} \\
\endfoot
\bottomrule 
\endlastfoot
    A1    & 11,30 & 11,06 & 11,05 \\
    A2    & 11,30 & 10,57 & 10,57 \\
    A3    & 11,27 & 10,73 & 10,73 \\
    B1    & 7,64  & 3,14  & 2,76 \\
    B2    & 7,64  & 3,14  & 2,76 \\
    B3    & 7,64  & 3,14  & 2,76 \\
    C1    & 12,34 & 12,27 & 12,34 \\
    C2    & 13,18 & 12,67 & 12,60 \\
    C3    & 13,17 & 12,74 & 12,76 \\
    D1    & 9,48  & 9,48  & 9,48 \\
    D2    & 12,38 & 11,39 & 11,35 \\
    D3    & 6,45  & 6,45  & 6,45 \\
    E1    & 10,80 & 10,80 & 10,80 \\
    E2    & 10,78 & 10,78 & 10,78 \\
    E3    & 10,80 & 10,80 & 10,80 \\
    F1    & 10,20 & 10,20 & 10,20 \\
    F2    & 8,20  & 7,14  & 7,32 \\
    F3    & 9,27  & 7,40  & 7,24 \\
    G1    & 12,03 & 11,71 & 11,71 \\
    G2    & 11,79 & 11,25 & 11,25 \\
    G3    & 12,06 & 11,82 & 11,81 \\
    H1    & 8,44  & 8,44  & 8,44 \\
    H2    & 12,29 & 12,24 & 12,24 \\
    H3    & 12,33 & 12,20 & 12,20 \\
    I1    & 8,14  & 8,01  & 7,92 \\
    I2    & 9,59  & 7,37  & 6,58 \\
    I3    & 8,15  & 8,15  & 8,13 \\
    J1    & 8,50  & 7,25  & 6,27 \\
    J2    & 8,52  & 7,34  & 6,45 \\
    J3    & 8,53  & 7,17  & 8,53 \\
    K1    & 10,94 & 9,88  & 9,82 \\
    K2    & 10,89 & 9,88  & 9,64 \\
    K3    & 10,87 & 9,79  & 9,74 \\
    L1    & 11,27 & 11,27 & 11,27 \\
    L2    & 11,08 & 11,08 & 11,08 \\
    L3    & 11,31 & 11,31 & 11,31 \\
    L4    & 12,80 & 12,20 & 11,93 \\
    L5    & 10,67 & 10,67 & 10,67 \\
    L6    & 11,58 & 11,58 & 11,58 \\


\end{longtable}
\end{center}

% \begin{figure}[!h]
% \centering
% \includegraphics[scale=0.8, angle=90]{fig/res/evolucaoentropiaMetXVIII00.png} 
% \caption[Metodologia XVIII: evolução da entropia nos conjuntos A, B e C]{Gráfico
% com comparativo da evolução da entropia dos conjuntos A, B e C sem e com GARCH na
% Metodologia XVIII}
% \label{Figura:evolucaoentropiaABCMet18}
% \end{figure}
% 
% \begin{figure}[!h]
% \centering
% \includegraphics[scale=0.8, angle=90]{fig/res/evolucaoentropiaMetXVIII01.png} 
% \caption[Metodologia XVIII: evolução da entropia nos conjuntos D, E e F]{Gráfico
% com comparativo da evolução da entropia dos conjuntos D, E e F sem e com GARCH na
% Metodologia XVIII}
% \label{Figura:evolucaoentropiaDEFMet18}
% \end{figure}
% 
% \begin{figure}[!h]
% \centering
% \includegraphics[scale=0.8, angle=90]{fig/res/evolucaoentropiaMetXVIII02.png} 
% \caption[Metodologia XVIII: evolução da entropia nos conjuntos G, H e I]{Gráfico
% com comparativo da evolução da entropia dos conjuntos G, H e I sem e com GARCH na
% Metodologia XVIII}
% \label{Figura:evolucaoentropiaGHIMet18}
% \end{figure}
% 
% \begin{figure}[!h]
% \centering
% \includegraphics[scale=0.6]{fig/res/evolucaoentropiaMetXVIII03.png} 
% \caption[Metodologia XVIII: evolução da entropia nos conjuntos J e K]{Gráfico com
% comparativo da evolução da entropia dos conjuntos J e K sem e com GARCH na
% Metodologia XVIII}
% \label{Figura:evolucaoentropiaJKMet18}
% \end{figure}
% 
% \begin{figure}[!h]
% \centering
% \includegraphics[scale=0.6]{fig/res/evolucaoentropiaMetXVIII04.png} 
% \caption[Metodologia XVIII: evolução da entropia nos conjuntos L]{Gráfico com
% comparativo da evolução da entropia dos conjuntos L sem e com GARCH na
% Metodologia XVIII}
% \label{Figura:evolucaoentropiaLMet18}
% \end{figure}

% \begin{figure}[!h]
% \centering
% \includegraphics[scale=1]{fig/res/evolucaoentropiaMetXVIII05.png} 
% \caption[Metodologia XVIII: evolução da entropia]{Gráfico com comparativo da
% evolução da entopia na Metodologia XVIII}
% \label{Figura:evolucaoentropiaPizzaMet18}
% \end{figure}

\clearpage

\begin{center}
\begin{longtable}{ccccc|cccc}
\toprule
\rowcolor{white}
\caption[Metodologia XVIII: tempo de execução]{Tempo de execução (em segundos)
dos algoritmos sem e com GARCH na Metodologia XVIII. Primeiro é exibido o tempo de
execução sem a utilização do modelo GARCH, depois com o modelo. Parâmetros
modelo se refere ao tempo gasto pelo algoritmo para o cálculo dos parâmetros do
modelo, Resíduo refere-se ao tempo gasto pelo modelo para calcular o resíduo do
modelo, Cod. Arit. refere-se ao tempo gasto pela codificação aritmética para
comprimir o resíduo.} \label{tab:EvolucaoEntropiaMet18}\\
\midrule
Conj & \specialcell{Parâmetros\\modelo} &
Resíduo & \specialcell{Cod.\\Arit.} & \specialcell{Tempo\\total} &
\specialcell{Parâmetros\\modelo} &
Resíduo & \specialcell{Cod.\\Arit.} & \specialcell{Tempo\\total} \\
\midrule
\endfirsthead 
%\multicolumn{8}{c}%
%{\tablename\ \thetable\ -- \textit{Continuação da página anterior}} \\
\midrule
\rowcolor{white}
Conj & \specialcell{Parâmetros\\modelo} &
Resíduo & \specialcell{Cod.\\Arit.} & \specialcell{Tempo\\total} &
\specialcell{Parâmetros\\modelo} &
Resíduo & \specialcell{Cod.\\Arit.} & \specialcell{Tempo\\total} \\
\toprule
\endhead
\midrule \\ % \multicolumn{8}{r}{\textit{Continua na próxima página}} \\
\endfoot
\bottomrule 
\endlastfoot
A1&184&$<1$&2&186&214&1&2&217\\
A2&165&$<1$&2&167&521&1&2&524\\
A3&165&$<1$&1&167&235&1&1&237\\
B1&55&$<1$&$<1$&56&215&$<1$&$<1$&216\\
B2&47&$<1$&$<1$&48&223&$<1$&$<1$&224\\
B3&48&$<1$&$<1$&48&216&$<1$&$<1$&216\\
C1&13&$<1$&1&13&36&$<1$&1&36\\
C2&15&$<1$&$<1$&15&41&$<1$&$<1$&42\\
C3&19&$<1$&$<1$&20&34&$<1$&$<1$&35\\
D1&19&$<1$&1&20&33&$<1$&1&34\\
D2&50&$<1$&1&51&131&$<1$&1&131\\
D3&15&$<1$&2&17&125&$<1$&$<1$&126\\
E1&3&$<1$&$<1$&3&18&$<1$&$<1$&18\\
E2&3&$<1$&$<1$&3&19&$<1$&$<1$&19\\
E3&4&$<1$&$<1$&4&20&$<1$&$<1$&20\\
F1&14&$<1$&1&14&28&$<1$&1&29\\
F2&28&$<1$&$<1$&28&99&$<1$&$<1$&100\\
F3&22&$<1$&$<1$&23&138&$<1$&$<1$&139\\
G1&6&$<1$&$<1$&7&28&$<1$&$<1$&29\\
G2&33&$<1$&$<1$&34&72&$<1$&$<1$&72\\
G3&10&$<1$&1&12&20&$<1$&$<1$&21\\
H1&27&$<1$&1&28&79&$<1$&1&80\\
H2&17&$<1$&1&18&60&$<1$&1&60\\
H3&9&$<1$&1&10&47&$<1$&1&48\\
I1&16&$<1$&$<1$&17&90&$<1$&$<1$&91\\
I2&21&$<1$&$<1$&21&62&$<1$&$<1$&62\\
I3&6&$<1$&$<1$&7&25&$<1$&$<1$&26\\
J1&135&$<1$&1&136&485&$<1$&$<1$&486\\
J2&97&$<1$&$<1$&98&348&$<1$&$<1$&349\\
J3&268&$<1$&1&268&$<1$&$<1$&2&2\\
K1&15&$<1$&$<1$&16&62&$<1$&$<1$&63\\
K2&87&$<1$&$<1$&88&220&$<1$&$<1$&220\\
K3&42&$<1$&$<1$&42&77&$<1$&$<1$&78\\
L1&59&$<1$&1&60&47&$<1$&1&48\\
L2&17&$<1$&1&18&61&$<1$&1&62\\
L3&14&$<1$&1&15&57&$<1$&1&59\\
L4&25&$<1$&1&26&98&$<1$&1&99\\
L5&96&$<1$&1&97&93&$<1$&1&94\\
L6&86&$<1$&1&87&83&$<1$&1&84\\
\end{longtable}
\end{center}

% \begin{figure}[!h]
% \centering
% \includegraphics[scale=1, angle=90]{fig/res/tempoexecMetXVIII00.png} 
% \caption[Metodologia XVIII: tempo de cálculo dos parâmetros dos modelos dos
% conjuntos A, B, C e D]{Gráfico com comparativo do tempo de cálculo dos
% parâmetros dos modelos dos conjuntos A, B, C e D sem e com GARCH na Metodologia
% XVIII}
% \label{Figura:tempocalculoABCDMet18}
% \end{figure}
% 
% \begin{figure}[!h]
% \centering
% \includegraphics[scale=0.75]{fig/res/tempoexecMetXVIII01.png} 
% \caption[Metodologia XVIII: tempo de cálculo dos parâmetros dos modelos dos
% conjuntos E, F e G]{Gráfico com comparativo do tempo de cálculo dos
% parâmetros dos modelos dos conjuntos E, F e G sem e com GARCH na Metodologia
% XVIII}
% \label{Figura:tempocalculoEFGMet18}
% \end{figure}
% 
% \begin{figure}[!h]
% \centering
% \includegraphics[scale=0.75]{fig/res/tempoexecMetXVIII02.png} 
% \caption[Metodologia XVIII: tempo de cálculo dos parâmetros dos modelos dos
% conjuntos H, I e J]{Gráfico com comparativo do tempo de cálculo dos
% parâmetros dos modelos dos conjuntos H, I e J sem e com GARCH na Metodologia
% XVIII}
% \label{Figura:tempocalculoHIJMet18}
% \end{figure}
% 
% \begin{figure}[!h]
% \centering 
% \includegraphics[scale=1, angle=90]{fig/res/tempoexecMetXVIII03.png} 
% \caption[Metodologia XVIII: tempo de cálculo dos parâmetros dos modelos dos
% conjuntos K e L]{Gráfico com comparativo do tempo de cálculo dos
% parâmetros dos modelos dos conjuntos K e L sem e com GARCH na Metodologia XVIII}
% \label{Figura:tempocalculoKLMet18}
% \end{figure}

% \begin{figure}[!h]
% \centering
% \includegraphics[scale=0.75]{fig/res/tempoexecMetXVIII04.png} 
% \caption[Metodologia XVIII: tempo total relativo gasto no cálculo dos
% parâmetros do modelo]{Gráfico com comparativo do tempo total relativo de cálculo
% dos parâmetros dos modelos sem e com GARCH na Metodologia XVIII}
% \label{Figura:tempocalculoPizzaMet18}
% \end{figure}

\clearpage

\begin{center}
\begin{longtable}{ccccc|cccc}
\toprule
\rowcolor{white}
\caption[Metodologia XVIII: evolução da autocorrelação]{Autocorrelação do dado
original e dos resíduos gerados sem e com a utilização do modelo GARCH na
Metodologia XVIII} \label{tab:EvolucaoAutocorrelacaoMet18}\\
\midrule
Conjunto & \specialcell{Autocorrelação\\Inicial} & \specialcell{Autocorrelação\\Sem
GARCH} & \specialcell{Autocorrelação\\Com GARCH} \\
\midrule
\endfirsthead 
%\multicolumn{8}{c}%
%{\tablename\ \thetable\ -- \textit{Continuação da página anterior}} \\
\midrule
\rowcolor{white}
Conjunto & \specialcell{Autocorrelação\\Inicial} & \specialcell{Autocorrelação\\Sem
GARCH} & \specialcell{Autocorrelação\\Com GARCH} \\
\toprule
\endhead
\midrule \\ % \multicolumn{8}{r}{\textit{Continua na próxima página}} \\
\endfoot
\bottomrule 
\endlastfoot
A1    & 6     & 0     & 1 \\
A2    & 5     & 0     & 0 \\
A3    & 6     & 0     & 0 \\
B1    & 6     & 0     & 3 \\
B2    & 6     & 0     & 3 \\
B3    & 6     & 0     & 3 \\
C1    & 2     & 0     & 0 \\
C2    & 1     & 0     & 1 \\
C3    & 2     & 0     & 0 \\
D1    & 2     & 4     & 4 \\
D2    & 2     & 3     & 0 \\
D3    & 2     & 3     & 3 \\
E1    & 4     & 0     & 0 \\
E2    & 4     & 0     & 0 \\
E3    & 4     & 0     & 0 \\
F1    & 1     & 4     & 4 \\
F2    & 6     & 0     & 6 \\
F3    & 6     & 0     & 0 \\
G1    & 1     & 1     & 1 \\
G2    & 2     & 1     & 1 \\
G3    & 6     & 1     & 1 \\
H1    & 1     & 0     & 0 \\
H2    & 1     & 0     & 1 \\
H3    & 1     & 0     & 0 \\
I1    & 7     & 0     & 1 \\
I2    & 1     & 0     & 0 \\
I3    & 1     & 1     & 1 \\
J1    & 3     & 0     & 7 \\
J2    & 3     & 3     & 7 \\
J3    & 8     & 2     & 8 \\
K1    & 3     & 0     & 0 \\
K2    & 3     & 0     & 2 \\
K3    & 2     & 0     & 1 \\
L1    & 2     & 0     & 0 \\
L2    & 7     & 0     & 2 \\
L3    & 11    & 0     & 0 \\
L4    & 7     & 0     & 1 \\
L5    & 7     & 0     & 0 \\
L6    & 6     & 0     & 0 \\


\end{longtable}
\end{center}

% \begin{figure}[!h]
% \centering
% \includegraphics[scale=0.75]{fig/res/evolucaoautocorrMetXVIII00.png} 
% \caption[Metodologia XVIII: evolução da autocorrelação nos conjuntos A, B e
% C]{Gráfico com comparativo da autocorrelação do resíduo gerado sem e com a
% utilização do modelo GARCH em relação ao dado original nos conjuntos A, B e C na
% Metodologia XVIII}
% \label{Figura:autocorrelacaoABCMet18}
% \end{figure}
% 
% \begin{figure}[!h]
% \centering
% \includegraphics[scale=0.69]{fig/res/evolucaoautocorrMetXVIII01.png} 
% \caption[Metodologia XVIII: evolução da autocorrelação nos conjuntos D, E e
% F]{Gráfico com comparativo da autocorrelação do resíduo gerado sem e com a
% utilização do modelo GARCH em relação ao dado original nos conjuntos D, E e F na
% Metodologia XVIII}
% \label{Figura:autocorrelacaoDEFMet18}
% \end{figure}
% 
% \begin{figure}[!h]
% \centering
% \includegraphics[scale=0.69]{fig/res/evolucaoautocorrMetXVIII02.png} 
% \caption[Metodologia XVIII: evolução da autocorrelação nos conjuntos G, H e
% I]{Gráfico com comparativo da autocorrelação do resíduo gerado sem e com a
% utilização do modelo GARCH em relação ao dado original nos conjuntos G, H e I na
% Metodologia XVIII}
% \label{Figura:autocorrelacaoGHIMet18}
% \end{figure}
% 
% \begin{figure}[!h]
% \centering
% \includegraphics[scale=0.69]{fig/res/evolucaoautocorrMetXVIII03.png} 
% \caption[Metodologia XVIII: evolução da autocorrelação nos conjuntos J e
% K]{Gráfico com comparativo da autocorrelação do resíduo gerado sem e com a
% utilização do modelo GARCH em relação ao dado original nos conjuntos J e K na
% Metodologia XVIII}
% \label{Figura:autocorrelacaoJKMet18}
% \end{figure}
% 
% \begin{figure}[!h]
% \centering
% \includegraphics[scale=0.69]{fig/res/evolucaoautocorrMetXVIII04.png} 
% \caption[Metodologia XVIII: evolução da autocorrelação nos conjuntos L]{Gráfico
% com comparativo da autocorrelação do resíduo gerado sem e com a utilização do modelo GARCH em relação ao dado original nos conjuntos L na
% Metodologia XVIII}
% \label{Figura:autocorrelacaoLMet18}
% \end{figure}

% \begin{figure}[!h]
% \centering
% \includegraphics[scale=0.75]{fig/res/evolucaoautocorrMetXVIII05.png} 
% \caption[Metodologia XVIII: tempo total relativo gasto no cálculo dos
% parâmetros do modelo]{Gráfico com comparativo da redução relativa total da
% autocorrelação do resíduo sem e com a utilização do modelo GARCH na
% Metodologia XVIII}
% \label{Figura:tempocalculoPizzaMet18}
% \end{figure}

\clearpage

\begin{center}
\begin{longtable}{ccccccccc}
\toprule
\rowcolor{white}
\caption[Metodologia XVIII: dados estatísticos]{Média e variância do dado original
comparadas às do resíduo calculado sem e com a utilização do modelo GARCH na
Metodologia XVIII} \label{tab:DadosEstatisticosMet18}\\
\midrule
    Conjunto & \specialcell{Média\\Original} &
    \specialcell{Var.\\Original} & \specialcell{Média\\Sem\\GARCH} &
    \specialcell{Var.\\Sem\\GARCH} & \specialcell{Média\\Com\\GARCH}&
    \specialcell{Var.\\Com\\GARCH} \\

\midrule
\endfirsthead 
%\multicolumn{8}{c}%
%{\tablename\ \thetable\ -- \textit{Continuação da página anterior}} \\
\midrule
\rowcolor{white}
    Conjunto & \specialcell{Média\\Orig.} &
    \specialcell{Var.\\Orig.} & \specialcell{Média\\Sem\\GARCH} &
    \specialcell{Var.\\Sem\\GARCH} & \specialcell{Média\\Com\\GARCH}&
    \specialcell{Var.\\Com\\GARCH} \\

\toprule
\endhead
\midrule \\ % \multicolumn{8}{r}{\textit{Continua na próxima página}} \\
\endfoot
\bottomrule 
\endlastfoot
A1    & 3,0E+04 & 1,8E+07 & 0,1   & 3,5E+05 & -0,3  & 3,7E+05 \\
A2    & 3,2E+04 & 1,1E+07 & 0,7   & 2,8E+05 & -0,8  & 2,8E+05 \\
A3    & 3,1E+04 & 1,4E+07 & 0,5   & 3,0E+05 & -1,4  & 3,1E+05 \\
B1    & 2,8E+04 & 4,5E+05 & 0,4   & 1,1E+01 & 0,4   & 1,0E+02 \\
B2    & 2,8E+04 & 4,5E+05 & 0,4   & 1,1E+01 & 0,4   & 1,0E+02 \\
B3    & 2,8E+04 & 4,5E+05 & 0,4   & 1,1E+01 & 0,4   & 1,0E+02 \\
C1    & 3,3E+04 & 8,1E+07 & 0,3   & 2,0E+07 & 0,2   & 2,2E+07 \\
C2    & 3,3E+04 & 4,0E+07 & 0,9   & 1,3E+07 & 2,8   & 1,7E+07 \\
C3    & 3,3E+04 & 5,7E+07 & 0,6   & 1,1E+07 & 0,8   & 1,3E+07 \\
D1    & 3,7E+04 & 4,1E+07 & 0,2   & 3,6E+06 & 0,7   & 3,6E+06 \\
D2    & 3,3E+04 & 1,2E+07 & 0,6   & 1,8E+06 & 0,7   & 1,9E+06 \\
D3    & 3,1E+04 & 1,0E+07 & -4,4  & 1,1E+06 & 5,9   & 1,5E+06 \\
E1    & 2,9E+04 & 5,8E+07 & 0,6   & 5,1E+07 & -4,7  & 5,1E+07 \\
E2    & 3,0E+04 & 5,8E+07 & 0,5   & 5,2E+07 & 0,6   & 5,3E+07 \\
E3    & 3,0E+04 & 6,0E+07 & 0,9   & 5,5E+07 & 0,8   & 5,6E+07 \\
F1    & 3,8E+04 & 3,9E+07 & 0,5   & 5,6E+06 & 0,4   & 5,6E+06 \\
F2    & 2,3E+04 & 5,4E+06 & 1,1   & 2,3E+05 & 0,3   & 3,7E+05 \\
F3    & 2,6E+04 & 6,0E+06 & 0,4   & 7,0E+04 & 0,6   & 8,3E+04 \\
G1    & 3,3E+04 & 3,3E+07 & 2,9   & 2,6E+06 & 0,1   & 2,6E+06 \\
G2    & 3,8E+04 & 1,9E+07 & 5,5   & 1,2E+06 & 2,6   & 1,3E+06 \\
G3    & 2,9E+04 & 3,6E+07 & 3,5   & 2,7E+06 & 0,4   & 2,7E+06 \\
H1    & 3,1E+04 & 3,6E+07 & -2,7  & 3,5E+07 & 38,2  & 3,5E+07 \\
H2    & 3,4E+04 & 8,1E+06 & 0,3   & 5,9E+06 & -16,0 & 5,1E+06 \\
H3    & 3,2E+04 & 7,3E+06 & 0,6   & 4,9E+06 & 0,1   & 5,1E+06 \\
I1    & 3,6E+04 & 1,2E+07 & 0,7   & 1,3E+06 & 0,2   & 1,4E+06 \\
I2    & 2,9E+04 & 1,2E+06 & 1,4   & 8,3E+04 & 0,0   & 1,5E+05 \\
I3    & 3,1E+04 & 3,3E+07 & 0,3   & 6,9E+06 & 0,7   & 7,0E+06 \\
J1    & 3,7E+04 & 1,2E+06 & 0,5   & 1,1E+04 & 0,5   & 6,1E+04 \\
J2    & 3,5E+04 & 1,5E+06 & 0,5   & 4,6E+04 & 0,5   & 9,1E+04 \\
J3    & 3,3E+04 & 1,3E+06 & 3,3E+04 & 1,3E+06 & 3,3E+04 & 1,3E+06 \\
K1    & 3,9E+04 & 6,9E+06 & 0,5   & 2,5E+05 & 1,3   & 2,6E+05 \\
K2    & 4,0E+04 & 6,7E+06 & 0,6   & 1,6E+05 & 0,2   & 4,2E+05 \\
K3    & 3,6E+04 & 5,8E+06 & 1,1   & 3,1E+05 & 0,5   & 3,5E+05 \\
L1    & 3,4E+04 & 2,9E+07 & -4,7  & 5,3E+06 & 0,4   & 5,2E+06 \\
L2    & 3,1E+04 & 1,5E+07 & 0,7   & 5,2E+06 & 0,7   & 5,4E+06 \\
L3    & 3,5E+04 & 1,3E+07 & -0,3  & 7,5E+06 & 1,8   & 7,6E+06 \\
L4    & 3,7E+04 & 1,8E+07 & 0,2   & 3,7E+06 & 3,3   & 6,9E+06 \\
L5    & 3,1E+04 & 5,1E+07 & 0,6   & 1,3E+06 & 0,2   & 1,3E+06 \\
L6    & 3,2E+04 & 2,6E+07 & -0,7  & 1,1E+06 & 0,1   & 1,1E+06 \\
\end{longtable}
\end{center}

% \begin{figure}[!h]
% \centering
% \includegraphics[scale=0.69]{fig/res/estatisticasMetXVIII03.png} 
% \caption[Metodologia XVIII: Variância do conjunto A]{Gráfico com
% comparativo da variância original do dado e dos resíduos gerados pelos modelos
% sem e com GARCH do conjunto A na Metodologia XVIII}
% \label{Figura:estatisticaAMet18}
% \end{figure}
% 
% \begin{figure}[!h]
% \centering
% \includegraphics[scale=0.69]{fig/res/estatisticasMetXVIII00.png} 
% \caption[Metodologia XVIII: Variância do conjunto B]{Gráfico com
% comparativo da variância original do dado e dos resíduos gerados pelos modelos
% sem e com GARCH do conjunto B na Metodologia XVIII}
% \label{Figura:estatisticaBMet18}
% \end{figure}
% 
% \begin{figure}[!h]
% \centering
% \includegraphics[scale=0.69]{fig/res/estatisticasMetXVIII01.png} 
% \caption[Metodologia XVIII: Variância do conjunto C]{Gráfico com
% comparativo da variância original do dado e dos resíduos gerados pelos modelos
% sem e com GARCH do conjunto C na Metodologia XVIII}
% \label{Figura:estatisticaCMet18}
% \end{figure}
% 
% \begin{figure}[!h]
% \centering
% \includegraphics[scale=0.69]{fig/res/estatisticasMetXVIII02.png} 
% \caption[Metodologia XVIII: Variância dos conjuntos D e E]{Gráfico com comparativo
% da variância original do dado e dos resíduos gerados pelos modelos sem e com
% GARCH dos conjuntos D e E na Metodologia XVIII}
% \label{Figura:estatisticaDEMet18}
% \end{figure}
% 
% \begin{figure}[!h]
% \centering
% \includegraphics[scale=0.69]{fig/res/estatisticasMetXVIII04.png} 
% \caption[Metodologia XVIII: Variância do conjunto F]{Gráfico com
% comparativo da variância original do dado e dos resíduos gerados pelos modelos
% sem e com GARCH do conjunto F na Metodologia XVIII}
% \label{Figura:estatisticaFMet18}
% \end{figure}
% 
% \begin{figure}[!h]
% \centering
% \includegraphics[scale=0.8, angle=90]{fig/res/estatisticasMetXVIII05.png} 
% \caption[Metodologia XVIII: Variância dos conjuntos G, H e I]{Gráfico com
% comparativo da variância original do dado e dos resíduos gerados pelos modelos
% sem e com GARCH dos conjuntos G, H e I na Metodologia XVIII}
% \label{Figura:estatisticaGHIMet18}
% \end{figure}
% 
% \begin{figure}[!h]
% \centering
% \includegraphics[scale=0.8, angle=90]{fig/res/estatisticasMetXVIII06.png} 
% \caption[Metodologia XVIII: Variância dos conjuntos J e K]{Gráfico
% com comparativo da variância original do dado e dos resíduos gerados pelos modelos
% sem e com GARCH dos conjuntos J e K na Metodologia XVIII}
% \label{Figura:estatisticaJKMet18}
% \end{figure}
% 
% \begin{figure}[!h]
% \centering
% \includegraphics[scale=0.8, angle=90]{fig/res/estatisticasMetXVIII07.png} 
% \caption[Metodologia XVIII: Variância do conjunto L]{Gráfico com
% comparativo da variância original do dado e dos resíduos gerados pelos modelos
% sem e com GARCH do conjunto L na Metodologia XVIII}
% \label{Figura:estatisticaLMet18}
% \end{figure}

% \begin{figure}[!h]
% \centering
% \includegraphics[scale=0.65]{fig/res/estatisticasMetXVIII08.png} 
% \caption[Metodologia XVIII: redução relativa da variância]{Gráfico com comparativo
% da redução relativa total da variância do resíduo sem e com a utilização do modelo GARCH na
% Metodologia XVIII}
% \label{Figura:estatisticaPizzaMet18}
% \end{figure}