
\begin{center}
\begin{longtable}{cccccc}
\toprule
\rowcolor{white}
\caption[Metodologia X: comparativo de convergência de soluções]{Comparativo
   de quantidade de experimentos cujas soluções convergiram com e sem a
   utilização do GARCH na metodologia X} \label{Tab:convergenciaMet10} \\
\midrule
   Cenário & \specialcell{Total experimentos} & Convergiram &
   \specialcell{Não convergiram} & \% sucesso \\
\midrule
\endfirsthead
%\multicolumn{8}{c}%
%{\tablename\ \thetable\ -- \textit{Continuação da página anterior}} \\
\midrule
\rowcolor{white}
   Cenário & \specialcell{Total experimentos} & Convergiram &
   \specialcell{Não convergiram} & \% sucesso \\
\toprule
\endhead
\midrule \\ % \multicolumn{8}{r}{\textit{Continua na próxima página}} \\
\endfoot
\bottomrule
\endlastfoot
	Sem GARCH & 39 & 39 & 0 & 100\% \\
	Com GARCH & 39 & 39 & 0 & 100\% \\
\end{longtable}
\end{center}

%%%%%%%%%%%%%%%%%%%%%%%%%%%%%%%%%%%%%%%%%%%%%%%%%%%%%%%%%%%%%%%%%%%%%%%%%%%%%%%%%%%%%%%%%
\begin{center}
\begin{longtable}{cccccc}
\toprule
\rowcolor{white}
\caption[Metodologia X: Razão de compressão]{Razão de compressão dos
experimentos sem e com GARCH na Metodologia X.
Valores em bytes.} \label{Tab:razaocompressaoMet} \\
\midrule
Conjunto & \specialcell{Tamanho \\Original} & \specialcell{Tamanho
\\Comprimido\\Com GARCH} & \specialcell{Tamanho
\\Comprimido\\Sem GARCH} & \specialcell{Razão \\Compressão
\\Sem GARCH} & \specialcell{Razão \\Compressão
\\Com GARCH} \\
\midrule
\endfirsthead
%\multicolumn{8}{c}%
%{\tablename\ \thetable\ -- \textit{Continuação da página anterior}} \\
\midrule
\rowcolor{white}
Conjunto & \specialcell{Tamanho \\Original} & \specialcell{Tamanho
\\Comprimido\\Com GARCH} & \specialcell{Tamanho
\\Comprimido\\Sem GARCH} & \specialcell{Razão \\Compressão
\\Sem GARCH} & \specialcell{Razão \\Compressão
\\Com GARCH} \\
\toprule
\endhead
\midrule \\ % \multicolumn{8}{r}{\textit{Continua na próxima página}} \\
\endfoot
\bottomrule
\endlastfoot
A1    & 1152000 & 1003400 & 1003400 & 1,15  & 1,15 \\
A2    & 1152000 & 966743 & 966739 & 1,19  & 1,19 \\
A3    & 1152000 & 985836 & 985839 & 1,17  & 1,17 \\
B1    & 518592 & 354038 & 354074 & 1,46  & 1,46 \\
B2    & 518592 & 354038 & 354074 & 1,46  & 1,46 \\
B3    & 518592 & 354038 & 354074 & 1,46  & 1,46 \\
C1    & 288192 & 282095 & 282096 & 1,02  & 1,02 \\
C2    & 288192 & 268248 & 268248 & 1,07  & 1,07 \\
C3    & 288192 & 278128 & 278128 & 1,04  & 1,04 \\
D1    & 331200 & 291097 & 291097 & 1,14  & 1,14 \\
D2    & 331200 & 278600 & 278601 & 1,19  & 1,19 \\
D3    & 331200 & 258797 & 258799 & 1,28  & 1,28 \\
E1    & 33792 & 31771 & 31771 & 1,06  & 1,06 \\
E2    & 33792 & 31928 & 31928 & 1,06  & 1,06 \\
E3    & 33792 & 32051 & 32051 & 1,05  & 1,05 \\
F1    & 220992 & 195331 & 195331 & 1,13  & 1,13 \\
F2    & 220992 & 153419 & 153426 & 1,44  & 1,44 \\
F3    & 220992 & 165199 & 165199 & 1,34  & 1,34 \\
G1    & 139392 & 125367 & 125367 & 1,11  & 1,11 \\
G2    & 139392 & 120337 & 120339 & 1,16  & 1,16 \\
G3    & 139392 & 126437 & 126437 & 1,10  & 1,10 \\
H1    & 360192 & 328186 & 328187 & 1,10  & 1,10 \\
H2    & 360192 & 297864 & 297864 & 1,21  & 1,21 \\
H3    & 360192 & 297359 & 297359 & 1,21  & 1,21 \\
I1    & 221184 & 180127 & 180127 & 1,23  & 1,23 \\
I2    & 221184 & 152397 & 152393 & 1,45  & 1,45 \\
I3    & 221184 & 184878 & 184879 & 1,20  & 1,20 \\
J1    & 591936 & 392522 & 392528 & 1,51  & 1,51 \\
J2    & 591936 & 389230 & 389230 & 1,52  & 1,52 \\
J3    & 591936 & 397656 & 397704 & 1,49  & 1,49 \\
K1    & 288000 & 220927 & 220927 & 1,30  & 1,30 \\
K2    & 288000 & 216235 & 216248 & 1,33  & 1,33 \\
K3    & 288000 & 215473 & 215473 & 1,34  & 1,34 \\
L1    & 480192 & 429382 & 429384 & 1,12  & 1,12 \\
L2    & 480192 & 408630 & 408632 & 1,18  & 1,18 \\
L3    & 480192 & 421849 & 421850 & 1,14  & 1,14 \\
L4    & 480192 & 419455 & 419455 & 1,14  & 1,14 \\
L5    & 480192 & 444338 & 444340 & 1,08  & 1,08 \\
L6    & 480192 & 430385 & 430384 & 1,12  & 1,12 \\
\end{longtable}
\end{center}

% \begin{figure}[!h]
% \centering
% \includegraphics[scale=1, angle=90]{fig/res/razaocompMetX00.png}
% \caption[Metodologia X: razão de compressão dos conjuntos A, B e C]{Gráfico
% com comparativo da razão de compressão dos conjuntos A, B e C sem e com GARCH na
% Metodologia X}
% \label{Figura:razaocompressaoABCMet10}
% \end{figure}
%  
% \begin{figure}[!h]
% \centering
% \includegraphics[scale=1, angle=90]{fig/res/razaocompMetX01.png}
% \caption[Metodologia X: razão de compressão dos conjuntos D, E e F]{Gráfico
% com comparativo da razão de compressão dos conjuntos D, E e F sem e com GARCH na
% Metodologia X}
% \label{Figura:razaocompressaoDEFMet10}
% \end{figure}
% 
% \begin{figure}[!h]
% \centering
% \includegraphics[scale=1, angle=90]{fig/res/razaocompMetX02.png}
% \caption[Metodologia X: razão de compressão dos conjuntos G, H e I]{Gráfico
% com comparativo da razão de compressão dos conjuntos G, H e I sem e com GARCH na
% Metodologia X}
% \label{Figura:razaocompressaoGHIMet10}
% \end{figure}
% 
% \begin{figure}[!h]
% \centering
% \includegraphics[scale=1, angle=90]{fig/res/razaocompMetX03.png}
% \caption[Metodologia X: razão de compressão dos conjuntos J, K e L]{Gráfico
% com comparativo da razão de compressão dos conjuntos J, K e L sem e com GARCH na
% Metodologia X}
% \label{Figura:razaocompressaoJKLMet10}
% \end{figure}

% \begin{figure}[!h]
% \centering
% \includegraphics[scale=0.9]{fig/res/razaocompMetX04.png}
% \caption[Metodologia X: razão de compressão]{Gráfico com comparativo da razão
% de compressão na Metodologia X}
% \label{Figura:razaocompressaoPizzaMet10}
% \end{figure}

\clearpage

\begin{center}
\begin{longtable}{cccc}
\toprule
\rowcolor{white}
\caption[Metodologia X: evolução da entropia]{Evolução da entropia do dado
original e do resíduo calculado na metodologia X}
\label{tab:EvolucaoEntropiaMet10}\\
\midrule
Conjunto & \specialcell{Entropia \\Inicial} & \specialcell{Entropia do
\\Resíduo sem GARC} & \specialcell{Entropia do
\\Resíduo com GARC}  \\
\midrule
\endfirsthead
%\multicolumn{8}{c}%
%{\tablename\ \thetable\ -- \textit{Continuação da página anterior}} \\
\midrule
\rowcolor{white}
Conjunto & \specialcell{Entropia \\Inicial} & \specialcell{Entropia do
\\Resíduo sem GARC} & \specialcell{Entropia do
\\Resíduo com GARC}  \\
\toprule
\endhead
\midrule \\ % \multicolumn{8}{r}{\textit{Continua na próxima página}} \\
\endfoot
\bottomrule 
\endlastfoot
A1    & 11,30 & 11,30 & 11,30 \\
A2    & 11,30 & 11,30 & 11,30 \\
A3    & 11,27 & 11,27 & 11,27 \\
B1    & 7,64  & 7,18  & 7,19 \\
B2    & 7,64  & 7,18  & 7,19 \\
B3    & 7,64  & 7,18  & 7,19 \\
C1    & 12,34 & 12,33 & 12,33 \\
C2    & 13,18 & 13,04 & 13,04 \\
C3    & 13,17 & 13,06 & 13,05 \\
D1    & 9,48  & 9,48  & 9,48 \\
D2    & 12,38 & 12,16 & 12,16 \\
D3    & 6,45  & 6,45  & 6,45 \\
E1    & 10,80 & 10,80 & 10,80 \\
E2    & 10,78 & 10,78 & 10,78 \\
E3    & 10,80 & 10,80 & 10,80 \\
F1    & 10,20 & 10,18 & 10,18 \\
F2    & 8,20  & 7,98  & 7,98 \\
F3    & 9,27  & 9,10  & 9,10 \\
G1    & 12,03 & 11,97 & 11,98 \\
G2    & 11,79 & 11,68 & 11,70 \\
G3    & 12,06 & 12,00 & 12,01 \\
H1    & 8,44  & 8,44  & 8,44 \\
H2    & 12,29 & 12,17 & 12,16 \\
H3    & 12,33 & 12,15 & 12,15 \\
I1    & 8,14  & 8,11  & 8,11 \\
I2    & 9,59  & 9,20  & 9,20 \\
I3    & 8,15  & 8,13  & 8,09 \\
J1    & 8,50  & 8,13  & 8,13 \\
J2    & 8,52  & 8,16  & 8,16 \\
J3    & 8,53  & 8,16  & 8,16 \\
K1    & 10,94 & 10,65 & 10,65 \\
K2    & 10,89 & 10,60 & 10,60 \\
K3    & 10,87 & 10,58 & 10,58 \\
L1    & 11,27 & 11,27 & 11,27 \\
L2    & 11,08 & 11,08 & 11,08 \\
L3    & 11,31 & 11,31 & 11,31 \\
L4    & 12,80 & 12,80 & 12,80 \\
L5    & 10,67 & 10,67 & 10,67 \\
L6    & 11,58 & 11,58 & 11,58 \\

\end{longtable}
\end{center}

% \begin{figure}[!h]
% \centering
% \includegraphics[scale=0.8, angle=90]{fig/res/evolucaoentropiaMetX00.png} 
% \caption[Metodologia X: evolução da entropia nos conjuntos A, B e C]{Gráfico
% com comparativo da evolução da entropia dos conjuntos A, B e C sem e com GARCH na
% Metodologia X}
% \label{Figura:evolucaoentropiaABCMet10}
% \end{figure}
% 
% \begin{figure}[!h]
% \centering
% \includegraphics[scale=0.8, angle=90]{fig/res/evolucaoentropiaMetX01.png} 
% \caption[Metodologia X: evolução da entropia nos conjuntos D, E e F]{Gráfico
% com comparativo da evolução da entropia dos conjuntos D, E e F sem e com GARCH na
% Metodologia X}
% \label{Figura:evolucaoentropiaDEFMet10}
% \end{figure}
% 
% \begin{figure}[!h]
% \centering
% \includegraphics[scale=0.8, angle=90]{fig/res/evolucaoentropiaMetX02.png} 
% \caption[Metodologia X: evolução da entropia nos conjuntos G, H e I]{Gráfico
% com comparativo da evolução da entropia dos conjuntos G, H e I sem e com GARCH na
% Metodologia X}
% \label{Figura:evolucaoentropiaGHIMet10}
% \end{figure}
% 
% \begin{figure}[!h]
% \centering
% \includegraphics[scale=0.6]{fig/res/evolucaoentropiaMetX03.png} 
% \caption[Metodologia X: evolução da entropia nos conjuntos J e K]{Gráfico com
% comparativo da evolução da entropia dos conjuntos J e K sem e com GARCH na
% Metodologia X}
% \label{Figura:evolucaoentropiaJKMet10}
% \end{figure}
% 
% \begin{figure}[!h]
% \centering
% \includegraphics[scale=0.6]{fig/res/evolucaoentropiaMetX04.png} 
% \caption[Metodologia X: evolução da entropia nos conjuntos L]{Gráfico com
% comparativo da evolução da entropia dos conjuntos L sem e com GARCH na
% Metodologia X}
% \label{Figura:evolucaoentropiaLMet10}
% \end{figure}

% \begin{figure}[!h]
% \centering
% \includegraphics[scale=1]{fig/res/evolucaoentropiaMetX05.png} 
% \caption[Metodologia X: evolução da entropia]{Gráfico com comparativo da
% evolução da entopia na Metodologia X}
% \label{Figura:evolucaoentropiaPizzaMet10}
% \end{figure}

\clearpage

\begin{center}
\begin{longtable}{ccccc|cccc}
\toprule
\rowcolor{white}
\caption[Metodologia X: tempo de execução]{Tempo de execução (em segundos)
dos algoritmos sem e com GARCH na Metodologia X. Primeiro é exibido o tempo de
execução sem a utilização do modelo GARCH, depois com o modelo. Parâmetros
modelo se refere ao tempo gasto pelo algoritmo para o cálculo dos parâmetros do
modelo, Resíduo refere-se ao tempo gasto pelo modelo para calcular o resíduo do
modelo, Cod. Arit. refere-se ao tempo gasto pela codificação aritmética para
comprimir o resíduo.} \label{tab:EvolucaoEntropiaMet10}\\
\midrule
Conj & \specialcell{Parâmetros\\modelo} &
Resíduo & \specialcell{Cod.\\Arit.} & \specialcell{Tempo\\total} &
\specialcell{Parâmetros\\modelo} &
Resíduo & \specialcell{Cod.\\Arit.} & \specialcell{Tempo\\total} \\
\midrule
\endfirsthead 
%\multicolumn{8}{c}%
%{\tablename\ \thetable\ -- \textit{Continuação da página anterior}} \\
\midrule
\rowcolor{white}
Conj & \specialcell{Parâmetros\\modelo} &
Resíduo & \specialcell{Cod.\\Arit.} & \specialcell{Tempo\\total} &
\specialcell{Parâmetros\\modelo} &
Resíduo & \specialcell{Cod.\\Arit.} & \specialcell{Tempo\\total} \\
\toprule
\endhead
\midrule \\ % \multicolumn{8}{r}{\textit{Continua na próxima página}} \\
\endfoot
\bottomrule 
\endlastfoot
A1&7&1&9&17&434&2&13&449\\
A2&6&$<1$&3&10&62&1&4&66\\
A3&6&1&3&10&50&1&3&54\\
B1&3&$<1$&1&4&155&$<1$&1&156\\
B2&10&1&2&12&200&$<1$&1&201\\
B3&10&$<1$&1&12&346&$<1$&1&348\\
C1&11&$<1$&1&13&32&$<1$&2&34\\
C2&2&$<1$&3&5&12&$<1$&4&16\\
C3&3&$<1$&1&4&40&$<1$&1&41\\
D1&3&1&5&8&13&1&5&19\\
D2&2&$<1$&4&7&35&1&1&38\\
D3&5&$<1$&1&6&29&1&1&30\\
E1&1&$<1$&$<1$&1&6&$<1$&1&7\\
E2&7&1&1&9&4&$<1$&$<1$&5\\
E3&1&$<1$&$<1$&1&2&$<1$&$<1$&3\\
F1&7&1&2&9&11&1&2&13\\
F2&5&$<1$&$<1$&6&18&1&1&20\\
F3&3&$<1$&1&4&20&1&2&23\\
G1&1&$<1$&1&2&22&1&2&25\\
G2&3&$<1$&2&6&19&1&2&21\\
G3&10&1&1&11&18&$<1$&1&19\\
H1&7&$<1$&1&9&37&1&3&41\\
H2&7&1&5&12&33&$<1$&4&38\\
H3&3&$<1$&4&7&35&$<1$&1&37\\
I1&3&$<1$&1&3&27&$<1$&1&28\\
I2&11&$<1$&$<1$&12&85&1&2&88\\
I3&1&$<1$&1&2&17&$<1$&1&18\\
J1&3&$<1$&1&4&106&$<1$&1&107\\
J2&10&1&1&12&170&$<1$&1&170\\
J3&11&$<1$&1&13&215&1&1&217\\
K1&6&1&3&9&11&1&3&14\\
K2&2&$<1$&1&3&28&$<1$&1&29\\
K3&7&$<1$&1&8&12&1&2&14\\
L1&7&1&5&13&44&$<1$&2&46\\
L2&4&$<1$&4&8&19&1&2&23\\
L3&3&$<1$&6&10&45&1&3&49\\
L4&4&1&7&12&16&$<1$&2&18\\
L5&8&1&4&13&71&$<1$&2&73\\
L6&10&1&2&13&156&$<1$&2&158\\
\end{longtable}
\end{center}

% \begin{figure}[!h]
% \centering
% \includegraphics[scale=1, angle=90]{fig/res/tempoexecMetX00.png} 
% \caption[Metodologia X: tempo de cálculo dos parâmetros dos modelos dos
% conjuntos A, B, C e D]{Gráfico com comparativo do tempo de cálculo dos
% parâmetros dos modelos dos conjuntos A, B, C e D sem e com GARCH na Metodologia
% X}
% \label{Figura:tempocalculoABCDMet10}
% \end{figure}
% 
% \begin{figure}[!h]
% \centering
% \includegraphics[scale=0.75]{fig/res/tempoexecMetX01.png} 
% \caption[Metodologia X: tempo de cálculo dos parâmetros dos modelos dos
% conjuntos E, F e G]{Gráfico com comparativo do tempo de cálculo dos
% parâmetros dos modelos dos conjuntos E, F e G sem e com GARCH na Metodologia
% X}
% \label{Figura:tempocalculoEFGMet10}
% \end{figure}
% 
% \begin{figure}[!h]
% \centering
% \includegraphics[scale=0.75]{fig/res/tempoexecMetX02.png} 
% \caption[Metodologia X: tempo de cálculo dos parâmetros dos modelos dos
% conjuntos H, I e J]{Gráfico com comparativo do tempo de cálculo dos
% parâmetros dos modelos dos conjuntos H, I e J sem e com GARCH na Metodologia
% X}
% \label{Figura:tempocalculoHIJMet10}
% \end{figure}
% 
% \begin{figure}[!h]
% \centering 
% \includegraphics[scale=1, angle=90]{fig/res/tempoexecMetX03.png} 
% \caption[Metodologia X: tempo de cálculo dos parâmetros dos modelos dos
% conjuntos K e L]{Gráfico com comparativo do tempo de cálculo dos
% parâmetros dos modelos dos conjuntos K e L sem e com GARCH na Metodologia X}
% \label{Figura:tempocalculoKLMet10}
% \end{figure}

% \begin{figure}[!h]
% \centering
% \includegraphics[scale=0.75]{fig/res/tempoexecMetX04.png} 
% \caption[Metodologia X: tempo total relativo gasto no cálculo dos
% parâmetros do modelo]{Gráfico com comparativo do tempo total relativo de cálculo
% dos parâmetros dos modelos sem e com GARCH na Metodologia X}
% \label{Figura:tempocalculoPizzaMet10}
% \end{figure}

\clearpage

\begin{center}
\begin{longtable}{ccccc|cccc}
\toprule
\rowcolor{white}
\caption[Metodologia X: evolução da autocorrelação]{Autocorrelação do dado
original e dos resíduos gerados sem e com a utilização do modelo GARCH na
Metodologia X} \label{tab:EvolucaoAutocorrelacaoMet10}\\
\midrule
Conjunto & \specialcell{Autocorrelação\\Inicial} & \specialcell{Autocorrelação\\Sem
GARCH} & \specialcell{Autocorrelação\\Com GARCH} \\
\midrule
\endfirsthead 
%\multicolumn{8}{c}%
%{\tablename\ \thetable\ -- \textit{Continuação da página anterior}} \\
\midrule
\rowcolor{white}
Conjunto & \specialcell{Autocorrelação\\Inicial} & \specialcell{Autocorrelação\\Sem
GARCH} & \specialcell{Autocorrelação\\Com GARCH} \\
\toprule
\endhead
\midrule \\ % \multicolumn{8}{r}{\textit{Continua na próxima página}} \\
\endfoot
\bottomrule 
\endlastfoot
A1    & 6     & 5     & 5 \\
A2    & 5     & 5     & 5 \\
A3    & 6     & 6     & 6 \\
B1    & 6     & 6     & 6 \\
B2    & 6     & 6     & 6 \\
B3    & 6     & 6     & 6 \\
C1    & 2     & 1     & 1 \\
C2    & 1     & 1     & 1 \\
C3    & 2     & 2     & 2 \\
D1    & 2     & 2     & 2 \\
D2    & 2     & 1     & 1 \\
D3    & 2     & 2     & 2 \\
E1    & 4     & 0     & 0 \\
E2    & 4     & 0     & 0 \\
E3    & 4     & 0     & 0 \\
F1    & 1     & 1     & 1 \\
F2    & 6     & 6     & 6 \\
F3    & 6     & 6     & 6 \\
G1    & 1     & 1     & 1 \\
G2    & 2     & 6     & 6 \\
G3    & 6     & 1     & 1 \\
H1    & 1     & 1     & 1 \\
H2    & 1     & 1     & 1 \\
H3    & 1     & 1     & 1 \\
I1    & 7     & 6     & 6 \\
I2    & 1     & 1     & 1 \\
I3    & 1     & 1     & 1 \\
J1    & 3     & 8     & 8 \\
J2    & 3     & 8     & 8 \\
J3    & 8     & 7     & 7 \\
K1    & 3     & 3     & 3 \\
K2    & 3     & 2     & 2 \\
K3    & 2     & 2     & 2 \\
L1    & 2     & 1     & 1 \\
L2    & 7     & 3     & 3 \\
L3    & 11    & 8     & 8 \\
L4    & 7     & 6     & 6 \\
L5    & 7     & 7     & 7 \\
L6    & 6     & 6     & 6 \\
\end{longtable}
\end{center}

% \begin{figure}[!h]
% \centering
% \includegraphics[scale=0.75]{fig/res/evolucaoautocorrMetX00.png} 
% \caption[Metodologia X: evolução da autocorrelação nos conjuntos A, B e
% C]{Gráfico com comparativo da autocorrelação do resíduo gerado sem e com a
% utilização do modelo GARCH em relação ao dado original nos conjuntos A, B e C na
% Metodologia X}
% \label{Figura:autocorrelacaoABCMet10}
% \end{figure}
% 
% \begin{figure}[!h]
% \centering
% \includegraphics[scale=0.69]{fig/res/evolucaoautocorrMetX01.png} 
% \caption[Metodologia X: evolução da autocorrelação nos conjuntos D, E e
% F]{Gráfico com comparativo da autocorrelação do resíduo gerado sem e com a
% utilização do modelo GARCH em relação ao dado original nos conjuntos D, E e F na
% Metodologia X}
% \label{Figura:autocorrelacaoDEFMet10}
% \end{figure}
% 
% \begin{figure}[!h]
% \centering
% \includegraphics[scale=0.69]{fig/res/evolucaoautocorrMetX02.png} 
% \caption[Metodologia X: evolução da autocorrelação nos conjuntos G, H e
% I]{Gráfico com comparativo da autocorrelação do resíduo gerado sem e com a
% utilização do modelo GARCH em relação ao dado original nos conjuntos G, H e I na
% Metodologia X}
% \label{Figura:autocorrelacaoGHIMet10}
% \end{figure}
% 
% \begin{figure}[!h]
% \centering
% \includegraphics[scale=0.69]{fig/res/evolucaoautocorrMetX03.png} 
% \caption[Metodologia X: evolução da autocorrelação nos conjuntos J e
% K]{Gráfico com comparativo da autocorrelação do resíduo gerado sem e com a
% utilização do modelo GARCH em relação ao dado original nos conjuntos J e K na
% Metodologia X}
% \label{Figura:autocorrelacaoJKMet10}
% \end{figure}
% 
% \begin{figure}[!h]
% \centering
% \includegraphics[scale=0.69]{fig/res/evolucaoautocorrMetX04.png} 
% \caption[Metodologia X: evolução da autocorrelação nos conjuntos L]{Gráfico
% com comparativo da autocorrelação do resíduo gerado sem e com a utilização do modelo GARCH em relação ao dado original nos conjuntos L na
% Metodologia X}
% \label{Figura:autocorrelacaoLMet10}
% \end{figure}

% \begin{figure}[!h]
% \centering
% \includegraphics[scale=0.75]{fig/res/evolucaoautocorrMetX05.png} 
% \caption[Metodologia X: tempo total relativo gasto no cálculo dos
% parâmetros do modelo]{Gráfico com comparativo da redução relativa total da
% autocorrelação do resíduo sem e com a utilização do modelo GARCH na
% Metodologia X}
% \label{Figura:tempocalculoPizzaMet10}
% \end{figure}

\clearpage

\begin{center}
\begin{longtable}{ccccccccc}
\toprule
\rowcolor{white}
\caption[Metodologia X: dados estatísticos]{Média e variância do dado original
comparadas às do resíduo calculado sem e com a utilização do modelo GARCH na
Metodologia X} \label{tab:DadosEstatisticosMet10}\\
\midrule
    Conjunto & \specialcell{Média\\Original} &
    \specialcell{Var.\\Original} & \specialcell{Média\\Sem\\GARCH} &
    \specialcell{Var.\\Sem\\GARCH} & \specialcell{Média\\Com\\GARCH}&
    \specialcell{Var.\\Com\\GARCH} \\

\midrule
\endfirsthead 
%\multicolumn{8}{c}%
%{\tablename\ \thetable\ -- \textit{Continuação da página anterior}} \\
\midrule
\rowcolor{white}
    Conjunto & \specialcell{Média\\Orig.} &
    \specialcell{Var.\\Orig.} & \specialcell{Média\\Sem\\GARCH} &
    \specialcell{Var.\\Sem\\GARCH} & \specialcell{Média\\Com\\GARCH}&
    \specialcell{Var.\\Com\\GARCH} \\

\toprule
\endhead
\midrule \\ % \multicolumn{8}{r}{\textit{Continua na próxima página}} \\
\endfoot
\bottomrule 
\endlastfoot
A1    & 3,0E+04 & 1,8E+07 & 0,4   & 9,1E+06 & -66,6 & 9,1E+06 \\
A2    & 3,2E+04 & 1,1E+07 & 0,5   & 5,3E+06 & -25,9 & 5,3E+06 \\
A3    & 3,1E+04 & 1,4E+07 & 0,5   & 7,1E+06 & -25,4 & 7,1E+06 \\
B1    & 2,8E+04 & 4,5E+05 & 0,5   & 2,2E+05 & 0,2   & 2,2E+05 \\
B2    & 2,8E+04 & 4,5E+05 & 0,5   & 2,2E+05 & 0,2   & 2,2E+05 \\
B3    & 2,8E+04 & 4,5E+05 & 0,5   & 2,2E+05 & 0,2   & 2,2E+05 \\
C1    & 3,3E+04 & 8,1E+07 & 0,3   & 4,7E+07 & -34,6 & 4,7E+07 \\
C2    & 3,3E+04 & 4,0E+07 & 0,3   & 2,4E+07 & 0,3   & 2,4E+07 \\
C3    & 3,3E+04 & 5,7E+07 & 0,3   & 3,2E+07 & -34,9 & 3,2E+07 \\
D1    & 3,7E+04 & 4,1E+07 & 0,4   & 2,3E+07 & 0,4   & 2,3E+07 \\
D2    & 3,3E+04 & 1,2E+07 & 0,4   & 6,8E+06 & -21,7 & 6,8E+06 \\
D3    & 3,1E+04 & 1,0E+07 & 0,4   & 5,7E+06 & -2,5  & 5,7E+06 \\
E1    & 2,9E+04 & 5,8E+07 & -0,5  & 4,8E+07 & -0,6  & 4,8E+07 \\
E2    & 3,0E+04 & 5,8E+07 & -0,5  & 4,8E+07 & -0,5  & 4,8E+07 \\
E3    & 3,0E+04 & 6,0E+07 & -0,5  & 5,0E+07 & -0,5  & 5,0E+07 \\
F1    & 3,8E+04 & 3,9E+07 & 0,2   & 2,3E+07 & 0,2   & 2,3E+07 \\
F2    & 2,3E+04 & 5,4E+06 & 0,3   & 3,0E+06 & 5,9   & 3,0E+06 \\
F3    & 2,6E+04 & 6,0E+06 & 0,3   & 3,3E+06 & -8,8  & 3,3E+06 \\
G1    & 3,3E+04 & 3,3E+07 & 0,1   & 1,9E+07 & 28,2  & 1,9E+07 \\
G2    & 3,8E+04 & 1,9E+07 & 0,1   & 1,1E+07 & -19,6 & 1,1E+07 \\
G3    & 2,9E+04 & 3,6E+07 & 0,2   & 2,0E+07 & 27,0  & 2,0E+07 \\
H1    & 3,1E+04 & 3,6E+07 & 0,5   & 3,7E+07 & 10,7  & 3,7E+07 \\
H2    & 3,4E+04 & 8,1E+06 & 0,4   & 5,4E+06 & -6,5  & 5,4E+06 \\
H3    & 3,2E+04 & 7,3E+06 & 0,4   & 4,8E+06 & 20,3  & 4,8E+06 \\
I1    & 3,6E+04 & 1,2E+07 & 0,3   & 6,4E+06 & 73,2  & 6,4E+06 \\
I2    & 2,9E+04 & 1,2E+06 & 0,3   & 7,0E+05 & 2,2   & 7,0E+05 \\
I3    & 3,1E+04 & 3,3E+07 & 0,3   & 1,9E+07 & -12,8 & 1,9E+07 \\
J1    & 3,7E+04 & 1,2E+06 & 0,4   & 8,4E+05 & 0,2   & 8,4E+05 \\
J2    & 3,5E+04 & 1,5E+06 & 0,4   & 1,0E+06 & 0,5   & 1,0E+06 \\
J3    & 3,3E+04 & 1,3E+06 & 0,4   & 8,8E+05 & 0,1   & 8,8E+05 \\
K1    & 3,9E+04 & 6,9E+06 & 0,3   & 3,7E+06 & 0,3   & 3,7E+06 \\
K2    & 4,0E+04 & 6,7E+06 & 0,3   & 3,5E+06 & 2,1   & 3,5E+06 \\
K3    & 3,6E+04 & 5,8E+06 & 0,3   & 3,1E+06 & 0,3   & 3,1E+06 \\
L1    & 3,4E+04 & 2,9E+07 & 0,4   & 1,7E+07 & -52,9 & 1,7E+07 \\
L2    & 3,1E+04 & 1,5E+07 & 0,4   & 9,2E+06 & 0,4   & 9,2E+06 \\
L3    & 3,5E+04 & 1,3E+07 & 0,4   & 9,0E+06 & -17,2 & 9,0E+06 \\
L4    & 3,7E+04 & 1,8E+07 & 0,4   & 1,0E+07 & 0,4   & 1,0E+07 \\
L5    & 3,1E+04 & 5,1E+07 & 0,4   & 2,5E+07 & 16,4  & 2,5E+07 \\
L6    & 3,2E+04 & 2,6E+07 & 0,4   & 1,3E+07 & -56,2 & 1,3E+07 \\
\end{longtable}
\end{center}

% \begin{figure}[!h]
% \centering
% \includegraphics[scale=0.69]{fig/res/estatisticasMetX03.png} 
% \caption[Metodologia X: Variância do conjunto A]{Gráfico com
% comparativo da variância original do dado e dos resíduos gerados pelos modelos
% sem e com GARCH do conjunto A na Metodologia X}
% \label{Figura:estatisticaAMet10}
% \end{figure}
% 
% \begin{figure}[!h]
% \centering
% \includegraphics[scale=0.69]{fig/res/estatisticasMetX00.png} 
% \caption[Metodologia X: Variância do conjunto B]{Gráfico com
% comparativo da variância original do dado e dos resíduos gerados pelos modelos
% sem e com GARCH do conjunto B na Metodologia X}
% \label{Figura:estatisticaBMet10}
% \end{figure}
% 
% \begin{figure}[!h]
% \centering
% \includegraphics[scale=0.69]{fig/res/estatisticasMetX01.png} 
% \caption[Metodologia X: Variância do conjunto C]{Gráfico com
% comparativo da variância original do dado e dos resíduos gerados pelos modelos
% sem e com GARCH do conjunto C na Metodologia X}
% \label{Figura:estatisticaCMet10}
% \end{figure}
% 
% \begin{figure}[!h]
% \centering
% \includegraphics[scale=0.69]{fig/res/estatisticasMetX02.png} 
% \caption[Metodologia X: Variância dos conjuntos D e E]{Gráfico com comparativo
% da variância original do dado e dos resíduos gerados pelos modelos sem e com
% GARCH dos conjuntos D e E na Metodologia X}
% \label{Figura:estatisticaDEMet10}
% \end{figure}
% 
% \begin{figure}[!h]
% \centering
% \includegraphics[scale=0.69]{fig/res/estatisticasMetX04.png} 
% \caption[Metodologia X: Variância do conjunto F]{Gráfico com
% comparativo da variância original do dado e dos resíduos gerados pelos modelos
% sem e com GARCH do conjunto F na Metodologia X}
% \label{Figura:estatisticaFMet10}
% \end{figure}
% 
% \begin{figure}[!h]
% \centering
% \includegraphics[scale=0.8, angle=90]{fig/res/estatisticasMetX05.png} 
% \caption[Metodologia X: Variância dos conjuntos G, H e I]{Gráfico com
% comparativo da variância original do dado e dos resíduos gerados pelos modelos
% sem e com GARCH dos conjuntos G, H e I na Metodologia X}
% \label{Figura:estatisticaGHIMet10}
% \end{figure}
% 
% \begin{figure}[!h]
% \centering
% \includegraphics[scale=0.8, angle=90]{fig/res/estatisticasMetX06.png} 
% \caption[Metodologia X: Variância dos conjuntos J e K]{Gráfico
% com comparativo da variância original do dado e dos resíduos gerados pelos modelos
% sem e com GARCH dos conjuntos J e K na Metodologia X}
% \label{Figura:estatisticaJKMet10}
% \end{figure}
% 
% \begin{figure}[!h]
% \centering
% \includegraphics[scale=0.8, angle=90]{fig/res/estatisticasMetX07.png} 
% \caption[Metodologia X: Variância do conjunto L]{Gráfico com
% comparativo da variância original do dado e dos resíduos gerados pelos modelos
% sem e com GARCH do conjunto L na Metodologia X}
% \label{Figura:estatisticaLMet10}
% \end{figure}

% \begin{figure}[!h]
% \centering
% \includegraphics[scale=0.65]{fig/res/estatisticasMetX08.png} 
% \caption[Metodologia X: redução relativa da variância]{Gráfico com comparativo
% da redução relativa total da variância do resíduo sem e com a utilização do modelo GARCH na
% Metodologia X}
% \label{Figura:estatisticaPizzaMet10}
% \end{figure}