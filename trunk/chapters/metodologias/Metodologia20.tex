
\begin{center}
\begin{longtable}{cccccc}
\toprule
\rowcolor{white}
\caption[Metodologia XX: comparativo de convergência de soluções]{Comparativo
   de quantidade de experimentos cujas soluções convergiram com e sem a
   utilização do GARCH na metodologia XX} \label{Tab:convergenciaMet20} \\
\midrule
   Cenário & \specialcell{Total experimentos} & Convergiram &
   \specialcell{Não convergiram} & \% sucesso \\
\midrule
\endfirsthead
%\multicolumn{8}{c}%
%{\tablename\ \thetable\ -- \textit{Continuação da página anterior}} \\
\midrule
\rowcolor{white}
   Cenário & \specialcell{Total experimentos} & Convergiram &
   \specialcell{Não convergiram} & \% sucesso \\
\toprule
\endhead
\midrule \\ % \multicolumn{8}{r}{\textit{Continua na próxima página}} \\
\endfoot
\bottomrule
\endlastfoot
	Sem GARCH & 39 & 39 & 0 & 100\% \\
	Com GARCH & 39 & 39 & 0 & 100\% \\
\end{longtable}
\end{center}

%%%%%%%%%%%%%%%%%%%%%%%%%%%%%%%%%%%%%%%%%%%%%%%%%%%%%%%%%%%%%%%%%%%%%%%%%%%%%%%%%%%%%%%%%
\begin{center}
\begin{longtable}{cccccc}
\toprule
\rowcolor{white}
\caption[Metodologia XX: Razão de compressão]{Razão de compressão dos
experimentos sem e com GARCH na Metodologia XX.
Valores em bytes.} \label{Tab:razaocompressaoMet} \\
\midrule
Conjunto & \specialcell{Tamanho \\Original} & \specialcell{Tamanho
\\Comprimido\\Com GARCH} & \specialcell{Tamanho
\\Comprimido\\Sem GARCH} & \specialcell{Razão \\Compressão
\\Sem GARCH} & \specialcell{Razão \\Compressão
\\Com GARCH} \\
\midrule
\endfirsthead
%\multicolumn{8}{c}%
%{\tablename\ \thetable\ -- \textit{Continuação da página anterior}} \\
\midrule
\rowcolor{white}
Conjunto & \specialcell{Tamanho \\Original} & \specialcell{Tamanho
\\Comprimido\\Com GARCH} & \specialcell{Tamanho
\\Comprimido\\Sem GARCH} & \specialcell{Razão \\Compressão
\\Sem GARCH} & \specialcell{Razão \\Compressão
\\Com GARCH} \\
\toprule
\endhead
\midrule \\ % \multicolumn{8}{r}{\textit{Continua na próxima página}} \\
\endfoot
\bottomrule
\endlastfoot
A1    & 1152000 & 894659 & 894658 & 1,29  & 1,29 \\
A2    & 1152000 & 842363 & 842365 & 1,37  & 1,37 \\
A3    & 1152000 & 835116 & 835118 & 1,38  & 1,38 \\
B1    & 518592 & 270019 & 270032 & 1,92  & 1,92 \\
B2    & 518592 & 270019 & 270032 & 1,92  & 1,92 \\
B3    & 518592 & 270019 & 270032 & 1,92  & 1,92 \\
C1    & 288192 & 272073 & 272073 & 1,06  & 1,06 \\
C2    & 288192 & 250794 & 250794 & 1,15  & 1,15 \\
C3    & 288192 & 260687 & 260687 & 1,11  & 1,11 \\
D1    & 331200 & 285849 & 285850 & 1,16  & 1,16 \\
D2    & 331200 & 268149 & 268149 & 1,24  & 1,24 \\
D3    & 331200 & 252800 & 252796 & 1,31  & 1,31 \\
E1    & 33792 & 33898 & 33898 & 1,00  & 1,00 \\
E2    & 33792 & 34144 & 34144 & 0,99  & 0,99 \\
E3    & 33792 & 34257 & 34257 & 0,99  & 0,99 \\
F1    & 220992 & 196083 & 196083 & 1,13  & 1,13 \\
F2    & 220992 & 153687 & 153666 & 1,44  & 1,44 \\
F3    & 220992 & 165398 & 165398 & 1,34  & 1,34 \\
G1    & 139392 & 125410 & 125410 & 1,11  & 1,11 \\
G2    & 139392 & 113641 & 113640 & 1,23  & 1,23 \\
G3    & 139392 & 126183 & 126185 & 1,10  & 1,10 \\
H1    & 360192 & 345887 & 345887 & 1,04  & 1,04 \\
H2    & 360192 & 293436 & 293436 & 1,23  & 1,23 \\
H3    & 360192 & 298347 & 298347 & 1,21  & 1,21 \\
I1    & 221184 & 173238 & 173240 & 1,28  & 1,28 \\
I2    & 221184 & 150229 & 150227 & 1,47  & 1,47 \\
I3    & 221184 & 183145 & 183140 & 1,21  & 1,21 \\
J1    & 591936 & 388351 & 388351 & 1,52  & 1,52 \\
J2    & 591936 & 387994 & 387994 & 1,53  & 1,53 \\
J3    & 591936 & 387265 & 387265 & 1,53  & 1,53 \\
K1    & 288000 & 216996 & 217037 & 1,33  & 1,33 \\
K2    & 288000 & 197211 & 197204 & 1,46  & 1,46 \\
K3    & 288000 & 200198 & 200199 & 1,44  & 1,44 \\
L1    & 480192 & 407548 & 407548 & 1,18  & 1,18 \\
L2    & 480192 & 406557 & 406565 & 1,18  & 1,18 \\
L3    & 480192 & 421521 & 421522 & 1,14  & 1,14 \\
L4    & 480192 & 403713 & 403702 & 1,19  & 1,19 \\
L5    & 480192 & 388986 & 388980 & 1,23  & 1,23 \\
L6    & 480192 & 394957 & 394957 & 1,22  & 1,22 \\
\end{longtable}
\end{center}

% \begin{figure}[!h]
% \centering
% \includegraphics[scale=1, angle=90]{fig/res/razaocompMetXX00.png}
% \caption[Metodologia XX razão de compressão dos conjuntos A, B e C]{Gráfico
% com comparativo da razão de compressão dos conjuntos A, B e C sem e com GARCH na
% Metodologia XX}
% \label{Figura:razaocompressaoABCMet20}
% \end{figure}
%  
% \begin{figure}[!h]
% \centering
% \includegraphics[scale=1, angle=90]{fig/res/razaocompMetXX01.png}
% \caption[Metodologia XX: razão de compressão dos conjuntos D, E e F]{Gráfico
% com comparativo da razão de compressão dos conjuntos D, E e F sem e com GARCH na
% Metodologia XX}
% \label{Figura:razaocompressaoDEFMet20}
% \end{figure}
% 
% \begin{figure}[!h]
% \centering
% \includegraphics[scale=1, angle=90]{fig/res/razaocompMetXX02.png}
% \caption[Metodologia XX: razão de compressão dos conjuntos G, H e I]{Gráfico
% com comparativo da razão de compressão dos conjuntos G, H e I sem e com GARCH na
% Metodologia XX}
% \label{Figura:razaocompressaoGHIMet20}
% \end{figure}
% 
% \begin{figure}[!h]
% \centering
% \includegraphics[scale=1, angle=90]{fig/res/razaocompMetXX03.png}
% \caption[Metodologia XX: razão de compressão dos conjuntos J, K e L]{Gráfico
% com comparativo da razão de compressão dos conjuntos J, K e L sem e com GARCH na
% Metodologia XX}
% \label{Figura:razaocompressaoJKLMet20}
% \end{figure}
% 
% \begin{figure}[!h]
% \centering
% \includegraphics[scale=0.9]{fig/res/razaocompMetXX04.png}
% \caption[Metodologia XX: razão de compressão]{Gráfico com comparativo da razão
% de compressão na Metodologia XX}
% \label{Figura:razaocompressaoPizzaMet20}
% \end{figure}

\clearpage

\begin{center}
\begin{longtable}{cccc}
\toprule
\rowcolor{white}
\caption[Metodologia XX: evolução da entropia]{Evolução da entropia do dado
original e do resíduo calculado na metodologia XX}
\label{tab:EvolucaoEntropiaMet20}\\
\midrule
Conjunto & \specialcell{Entropia \\Inicial} & \specialcell{Entropia do
\\Resíduo sem GARC} & \specialcell{Entropia do
\\Resíduo com GARC}  \\
\midrule
\endfirsthead
%\multicolumn{8}{c}%
%{\tablename\ \thetable\ -- \textit{Continuação da página anterior}} \\
\midrule
\rowcolor{white}
Conjunto & \specialcell{Entropia \\Inicial} & \specialcell{Entropia do
\\Resíduo sem GARC} & \specialcell{Entropia do
\\Resíduo com GARC}  \\
\toprule
\endhead
\midrule \\ % \multicolumn{8}{r}{\textit{Continua na próxima página}} \\
\endfoot
\bottomrule 
\endlastfoot
A1    & 11,30 & 11,30 & 11,30 \\
A2    & 11,30 & 11,02 & 11,02 \\
A3    & 11,27 & 11,20 & 11,20 \\
B1    & 7,64  & 5,00  & 5,00 \\
B2    & 7,64  & 5,00  & 5,00 \\
B3    & 7,64  & 5,00  & 5,00 \\
C1    & 12,34 & 12,22 & 12,22 \\
C2    & 13,18 & 12,82 & 12,82 \\
C3    & 13,17 & 12,83 & 12,82 \\
D1    & 9,48  & 9,48  & 9,48 \\
D2    & 12,38 & 11,76 & 11,76 \\
D3    & 6,45  & 6,33  & 6,33 \\
E1    & 10,80 & 10,80 & 10,80 \\
E2    & 10,78 & 10,78 & 10,78 \\
E3    & 10,80 & 10,80 & 10,80 \\
F1    & 10,20 & 10,12 & 10,13 \\
F2    & 8,20  & 7,75  & 7,76 \\
F3    & 9,27  & 8,91  & 8,92 \\
G1    & 12,03 & 11,81 & 11,81 \\
G2    & 11,79 & 11,46 & 11,45 \\
G3    & 12,06 & 11,87 & 11,86 \\
H1    & 8,44  & 8,44  & 8,44 \\
H2    & 12,29 & 12,27 & 12,26 \\
H3    & 12,33 & 12,20 & 12,19 \\
I1    & 8,14  & 7,94  & 7,94 \\
I2    & 9,59  & 8,49  & 8,49 \\
I3    & 8,15  & 8,05  & 8,03 \\
J1    & 8,50  & 7,75  & 7,75 \\
J2    & 8,52  & 7,84  & 7,84 \\
J3    & 8,53  & 7,77  & 7,77 \\
K1    & 10,94 & 10,37 & 10,37 \\
K2    & 10,89 & 10,33 & 10,32 \\
K3    & 10,87 & 10,33 & 10,33 \\
L1    & 11,27 & 11,27 & 11,27 \\
L2    & 11,08 & 11,08 & 11,08 \\
L3    & 11,31 & 11,31 & 11,31 \\
L4    & 12,80 & 12,11 & 12,11 \\
L5    & 10,67 & 10,67 & 10,67 \\
L6    & 11,58 & 11,58 & 11,58 \\


\end{longtable}
\end{center}

% \begin{figure}[!h]
% \centering
% \includegraphics[scale=0.8, angle=90]{fig/res/evolucaoentropiaMetXX00.png} 
% \caption[Metodologia XX: evolução da entropia nos conjuntos A, B e C]{Gráfico
% com comparativo da evolução da entropia dos conjuntos A, B e C sem e com GARCH na
% Metodologia XX}
% \label{Figura:evolucaoentropiaABCMet20}
% \end{figure}
% 
% \begin{figure}[!h]
% \centering
% \includegraphics[scale=0.8, angle=90]{fig/res/evolucaoentropiaMetXX01.png} 
% \caption[Metodologia XX: evolução da entropia nos conjuntos D, E e F]{Gráfico
% com comparativo da evolução da entropia dos conjuntos D, E e F sem e com GARCH na
% Metodologia XX}
% \label{Figura:evolucaoentropiaDEFMet20}
% \end{figure}
% 
% \begin{figure}[!h]
% \centering
% \includegraphics[scale=0.8, angle=90]{fig/res/evolucaoentropiaMetXX02.png} 
% \caption[Metodologia XX: evolução da entropia nos conjuntos G, H e I]{Gráfico
% com comparativo da evolução da entropia dos conjuntos G, H e I sem e com GARCH na
% Metodologia XX}
% \label{Figura:evolucaoentropiaGHIMet20}
% \end{figure}
% 
% \begin{figure}[!h]
% \centering
% \includegraphics[scale=0.6]{fig/res/evolucaoentropiaMetXX03.png} 
% \caption[Metodologia XX: evolução da entropia nos conjuntos J e K]{Gráfico com
% comparativo da evolução da entropia dos conjuntos J e K sem e com GARCH na
% Metodologia XX}
% \label{Figura:evolucaoentropiaJKMet20}
% \end{figure}
% 
% \begin{figure}[!h]
% \centering
% \includegraphics[scale=0.6]{fig/res/evolucaoentropiaMetXX04.png} 
% \caption[Metodologia XX: evolução da entropia nos conjuntos L]{Gráfico com
% comparativo da evolução da entropia dos conjuntos L sem e com GARCH na
% Metodologia XX}
% \label{Figura:evolucaoentropiaLMet20}
% \end{figure}

% \begin{figure}[!h]
% \centering
% \includegraphics[scale=1]{fig/res/evolucaoentropiaMetXX05.png} 
% \caption[Metodologia XX: evolução da entropia]{Gráfico com comparativo da
% evolução da entopia na Metodologia XX}
% \label{Figura:evolucaoentropiaPizzaMet20}
% \end{figure}

\clearpage

\begin{center}
\begin{longtable}{ccccc|cccc}
\toprule
\rowcolor{white}
\caption[Metodologia XX: tempo de execução]{Tempo de execução (em segundos)
dos algoritmos sem e com GARCH na Metodologia XX. Primeiro é exibido o tempo de
execução sem a utilização do modelo GARCH, depois com o modelo. Parâmetros
modelo se refere ao tempo gasto pelo algoritmo para o cálculo dos parâmetros do
modelo, Resíduo refere-se ao tempo gasto pelo modelo para calcular o resíduo do
modelo, Cod. Arit. refere-se ao tempo gasto pela codificação aritmética para
comprimir o resíduo.} \label{tab:EvolucaoEntropiaMet20}\\
\midrule
Conj & \specialcell{Parâmetros\\modelo} &
Resíduo & \specialcell{Cod.\\Arit.} & \specialcell{Tempo\\total} &
\specialcell{Parâmetros\\modelo} &
Resíduo & \specialcell{Cod.\\Arit.} & \specialcell{Tempo\\total} \\
\midrule
\endfirsthead 
%\multicolumn{8}{c}%
%{\tablename\ \thetable\ -- \textit{Continuação da página anterior}} \\
\midrule
\rowcolor{white}
Conj & \specialcell{Parâmetros\\modelo} &
Resíduo & \specialcell{Cod.\\Arit.} & \specialcell{Tempo\\total} &
\specialcell{Parâmetros\\modelo} &
Resíduo & \specialcell{Cod.\\Arit.} & \specialcell{Tempo\\total} \\
\toprule
\endhead
\midrule \\ % \multicolumn{8}{r}{\textit{Continua na próxima página}} \\
\endfoot
\bottomrule 
\endlastfoot
A1&31&$<1$&3&35&47&1&3&51\\
A2&29&$<1$&3&32&40&1&3&44\\
A3&56&$<1$&3&59&43&1&3&46\\
B1&59&1&2&62&217&1&2&220\\
B2&65&$<1$&1&66&221&$<1$&1&222\\
B3&55&$<1$&2&58&219&$<1$&1&220\\
C1&6&$<1$&1&7&23&$<1$&4&27\\
C2&12&$<1$&1&13&25&1&1&26\\
C3&15&$<1$&1&16&26&$<1$&1&28\\
D1&18&$<1$&1&19&92&1&2&95\\
D2&43&$<1$&3&46&42&1&3&47\\
D3&26&$<1$&1&27&45&$<1$&1&46\\
E1&1&$<1$&$<1$&2&6&$<1$&$<1$&6\\
E2&2&$<1$&$<1$&2&5&$<1$&$<1$&5\\
E3&1&$<1$&$<1$&1&11&$<1$&$<1$&12\\
F1&17&$<1$&3&20&27&1&1&29\\
F2&29&$<1$&$<1$&29&116&1&1&118\\
F3&10&$<1$&2&12&38&1&2&41\\
G1&4&$<1$&1&5&13&$<1$&1&14\\
G2&15&$<1$&1&16&22&1&2&24\\
G3&5&$<1$&1&6&22&$<1$&1&23\\
H1&14&$<1$&2&16&37&1&1&38\\
H2&19&1&4&23&34&$<1$&1&35\\
H3&14&$<1$&1&15&36&1&3&40\\
I1&25&$<1$&2&28&36&$<1$&1&37\\
I2&11&$<1$&$<1$&12&79&1&1&81\\
I3&3&$<1$&1&4&11&$<1$&1&11\\
J1&82&$<1$&1&83&112&$<1$&1&113\\
J2&40&1&2&43&188&1&2&191\\
J3&65&1&1&67&198&2&2&202\\
K1&27&1&2&30&127&$<1$&1&128\\
K2&27&$<1$&1&28&130&1&2&133\\
K3&41&$<1$&1&42&125&$<1$&1&126\\
L1&28&$<1$&2&29&41&$<1$&2&43\\
L2&8&1&3&12&79&1&2&82\\
L3&23&1&2&27&39&$<1$&2&41\\
L4&28&$<1$&1&29&157&1&4&162\\
L5&36&1&6&42&48&1&5&54\\
L6&53&$<1$&3&56&28&1&2&31\\
\end{longtable}
\end{center}

% \begin{figure}[!h]
% \centering
% \includegraphics[scale=1, angle=90]{fig/res/tempoexecMetXX00.png} 
% \caption[Metodologia XX: tempo de cálculo dos parâmetros dos modelos dos
% conjuntos A, B, C e D]{Gráfico com comparativo do tempo de cálculo dos
% parâmetros dos modelos dos conjuntos A, B, C e D sem e com GARCH na Metodologia
% XX}
% \label{Figura:tempocalculoABCDMet20}
% \end{figure}
% 
% \begin{figure}[!h]
% \centering
% \includegraphics[scale=0.75]{fig/res/tempoexecMetXX01.png} 
% \caption[Metodologia XX: tempo de cálculo dos parâmetros dos modelos dos
% conjuntos E, F e G]{Gráfico com comparativo do tempo de cálculo dos
% parâmetros dos modelos dos conjuntos E, F e G sem e com GARCH na Metodologia
% XX}
% \label{Figura:tempocalculoEFGMet20}
% \end{figure}
% 
% \begin{figure}[!h]
% \centering
% \includegraphics[scale=0.75]{fig/res/tempoexecMetXX02.png} 
% \caption[Metodologia XX: tempo de cálculo dos parâmetros dos modelos dos
% conjuntos H, I e J]{Gráfico com comparativo do tempo de cálculo dos
% parâmetros dos modelos dos conjuntos H, I e J sem e com GARCH na Metodologia
% XX}
% \label{Figura:tempocalculoHIJMet20}
% \end{figure}
% 
% \begin{figure}[!h]
% \centering 
% \includegraphics[scale=1, angle=90]{fig/res/tempoexecMetXX03.png} 
% \caption[Metodologia XX: tempo de cálculo dos parâmetros dos modelos dos
% conjuntos K e L]{Gráfico com comparativo do tempo de cálculo dos
% parâmetros dos modelos dos conjuntos K e L sem e com GARCH na Metodologia XX}
% \label{Figura:tempocalculoKLMet20}
% \end{figure}

% \begin{figure}[!h]
% \centering
% \includegraphics[scale=0.75]{fig/res/tempoexecMetXX04.png} 
% \caption[Metodologia XX: tempo total relativo gasto no cálculo dos
% parâmetros do modelo]{Gráfico com comparativo do tempo total relativo de cálculo
% dos parâmetros dos modelos sem e com GARCH na Metodologia XX}
% \label{Figura:tempocalculoPizzaMet20}
% \end{figure}

\clearpage

\begin{center}
\begin{longtable}{ccccc|cccc}
\toprule
\rowcolor{white}
\caption[Metodologia XX: evolução da autocorrelação]{Autocorrelação do dado
original e dos resíduos gerados sem e com a utilização do modelo GARCH na
Metodologia XX} \label{tab:EvolucaoAutocorrelacaoMet20}\\
\midrule
Conjunto & \specialcell{Autocorrelação\\Inicial} & \specialcell{Autocorrelação\\Sem
GARCH} & \specialcell{Autocorrelação\\Com GARCH} \\
\midrule
\endfirsthead 
%\multicolumn{8}{c}%
%{\tablename\ \thetable\ -- \textit{Continuação da página anterior}} \\
\midrule
\rowcolor{white}
Conjunto & \specialcell{Autocorrelação\\Inicial} & \specialcell{Autocorrelação\\Sem
GARCH} & \specialcell{Autocorrelação\\Com GARCH} \\
\toprule
\endhead
\midrule \\ % \multicolumn{8}{r}{\textit{Continua na próxima página}} \\
\endfoot
\bottomrule 
\endlastfoot
A1    & 6     & 3     & 3 \\
A2    & 5     & 3     & 3 \\
A3    & 6     & 3     & 3 \\
B1    & 6     & 4     & 4 \\
B2    & 6     & 4     & 4 \\
B3    & 6     & 4     & 4 \\
C1    & 2     & 5     & 5 \\
C2    & 1     & 3     & 3 \\
C3    & 2     & 5     & 5 \\
D1    & 2     & 1     & 1 \\
D2    & 2     & 4     & 4 \\
D3    & 2     & 3     & 3 \\
E1    & 4     & 0     & 0 \\
E2    & 4     & 0     & 0 \\
E3    & 4     & 0     & 0 \\
F1    & 1     & 4     & 4 \\
F2    & 6     & 1     & 1 \\
F3    & 6     & 1     & 1 \\
G1    & 1     & 1     & 1 \\
G2    & 2     & 1     & 1 \\
G3    & 6     & 1     & 1 \\
H1    & 1     & 1     & 1 \\
H2    & 1     & 0     & 0 \\
H3    & 1     & 0     & 0 \\
I1    & 7     & 1     & 1 \\
I2    & 1     & 4     & 4 \\
I3    & 1     & 4     & 4 \\
J1    & 3     & 7     & 7 \\
J2    & 3     & 7     & 7 \\
J3    & 8     & 7     & 7 \\
K1    & 3     & 1     & 1 \\
K2    & 3     & 1     & 1 \\
K3    & 2     & 1     & 1 \\
L1    & 2     & 5     & 5 \\
L2    & 7     & 0     & 0 \\
L3    & 11    & 4     & 4 \\
L4    & 7     & 1     & 1 \\
L5    & 7     & 3     & 3 \\
L6    & 6     & 1     & 1 \\


\end{longtable}
\end{center}

% \begin{figure}[!h]
% \centering
% \includegraphics[scale=0.75]{fig/res/evolucaoautocorrMetXX00.png} 
% \caption[Metodologia XX: evolução da autocorrelação nos conjuntos A, B e
% C]{Gráfico com comparativo da autocorrelação do resíduo gerado sem e com a
% utilização do modelo GARCH em relação ao dado original nos conjuntos A, B e C na
% Metodologia XX}
% \label{Figura:autocorrelacaoABCMet20}
% \end{figure}
% 
% \begin{figure}[!h]
% \centering
% \includegraphics[scale=0.69]{fig/res/evolucaoautocorrMetXX01.png} 
% \caption[Metodologia XX: evolução da autocorrelação nos conjuntos D, E e
% F]{Gráfico com comparativo da autocorrelação do resíduo gerado sem e com a
% utilização do modelo GARCH em relação ao dado original nos conjuntos D, E e F na
% Metodologia XX}
% \label{Figura:autocorrelacaoDEFMet20}
% \end{figure}
% 
% \begin{figure}[!h]
% \centering
% \includegraphics[scale=0.69]{fig/res/evolucaoautocorrMetXX02.png} 
% \caption[Metodologia XX: evolução da autocorrelação nos conjuntos G, H e
% I]{Gráfico com comparativo da autocorrelação do resíduo gerado sem e com a
% utilização do modelo GARCH em relação ao dado original nos conjuntos G, H e I na
% Metodologia XX}
% \label{Figura:autocorrelacaoGHIMet20}
% \end{figure}
% 
% \begin{figure}[!h]
% \centering
% \includegraphics[scale=0.69]{fig/res/evolucaoautocorrMetXX03.png} 
% \caption[Metodologia XX: evolução da autocorrelação nos conjuntos J e
% K]{Gráfico com comparativo da autocorrelação do resíduo gerado sem e com a
% utilização do modelo GARCH em relação ao dado original nos conjuntos J e K na
% Metodologia XX}
% \label{Figura:autocorrelacaoJKMet20}
% \end{figure}
% 
% \begin{figure}[!h]
% \centering
% \includegraphics[scale=0.69]{fig/res/evolucaoautocorrMetXX04.png} 
% \caption[Metodologia XX: evolução da autocorrelação nos conjuntos L]{Gráfico
% com comparativo da autocorrelação do resíduo gerado sem e com a utilização do modelo GARCH em relação ao dado original nos conjuntos L na
% Metodologia XX}
% \label{Figura:autocorrelacaoLMet20}
% \end{figure}

% \begin{figure}[!h]
% \centering
% \includegraphics[scale=0.75]{fig/res/evolucaoautocorrMetXX05.png} 
% \caption[Metodologia XX: tempo total relativo gasto no cálculo dos
% parâmetros do modelo]{Gráfico com comparativo da redução relativa total da
% autocorrelação do resíduo sem e com a utilização do modelo GARCH na
% Metodologia XX}
% \label{Figura:tempocalculoPizzaMet20}
% \end{figure}

\clearpage

\begin{center}
\begin{longtable}{ccccccccc}
\toprule
\rowcolor{white}
\caption[Metodologia XX: dados estatísticos]{Média e variância do dado original
comparadas às do resíduo calculado sem e com a utilização do modelo GARCH na
Metodologia XX} \label{tab:DadosEstatisticosMet20}\\
\midrule
    Conjunto & \specialcell{Média\\Original} &
    \specialcell{Var.\\Original} & \specialcell{Média\\Sem\\GARCH} &
    \specialcell{Var.\\Sem\\GARCH} & \specialcell{Média\\Com\\GARCH}&
    \specialcell{Var.\\Com\\GARCH} \\

\midrule
\endfirsthead 
%\multicolumn{8}{c}%
%{\tablename\ \thetable\ -- \textit{Continuação da página anterior}} \\
\midrule
\rowcolor{white}
    Conjunto & \specialcell{Média\\Orig.} &
    \specialcell{Var.\\Orig.} & \specialcell{Média\\Sem\\GARCH} &
    \specialcell{Var.\\Sem\\GARCH} & \specialcell{Média\\Com\\GARCH}&
    \specialcell{Var.\\Com\\GARCH} \\

\toprule
\endhead
\midrule \\ % \multicolumn{8}{r}{\textit{Continua na próxima página}} \\
\endfoot
\bottomrule 
\endlastfoot
A1    & 3,0E+04 & 1,8E+07 & 0,3   & 7,3E+05 & 0,0   & 7,3E+05 \\
A2    & 3,2E+04 & 1,1E+07 & 0,5   & 5,0E+05 & 2,0   & 5,0E+05 \\
A3    & 3,1E+04 & 1,4E+07 & 8,4   & 5,8E+05 & 0,5   & 5,8E+05 \\
B1    & 2,8E+04 & 4,5E+05 & 0,6   & 1,2E+04 & 0,4   & 1,2E+04 \\
B2    & 2,8E+04 & 4,5E+05 & 0,6   & 1,2E+04 & 0,4   & 1,2E+04 \\
B3    & 2,8E+04 & 4,5E+05 & 0,6   & 1,2E+04 & 0,4   & 1,2E+04 \\
C1    & 3,3E+04 & 8,1E+07 & 0,5   & 3,0E+07 & 0,3   & 3,0E+07 \\
C2    & 3,3E+04 & 4,0E+07 & 0,5   & 1,8E+07 & 0,5   & 1,8E+07 \\
C3    & 3,3E+04 & 5,7E+07 & 0,5   & 1,8E+07 & 0,6   & 1,8E+07 \\
D1    & 3,7E+04 & 4,1E+07 & 0,6   & 9,9E+06 & -7,3  & 9,9E+06 \\
D2    & 3,3E+04 & 1,2E+07 & -7,3  & 3,7E+06 & 6,8   & 3,7E+06 \\
D3    & 3,1E+04 & 1,0E+07 & 0,8   & 2,8E+06 & 3,1   & 2,8E+06 \\
E1    & 2,9E+04 & 5,8E+07 & 0,6   & 8,1E+07 & -7,3  & 8,1E+07 \\
E2    & 3,0E+04 & 5,8E+07 & 0,7   & 8,1E+07 & 0,5   & 8,1E+07 \\
E3    & 3,0E+04 & 6,0E+07 & 0,6   & 8,5E+07 & 0,6   & 8,5E+07 \\
F1    & 3,8E+04 & 3,9E+07 & 0,5   & 1,5E+07 & -6,9  & 1,5E+07 \\
F2    & 2,3E+04 & 5,4E+06 & -5,8  & 1,4E+06 & 0,4   & 1,4E+06 \\
F3    & 2,6E+04 & 6,0E+06 & 0,6   & 1,4E+06 & 2,3   & 1,4E+06 \\
G1    & 3,3E+04 & 3,3E+07 & 0,6   & 1,0E+07 & 0,6   & 1,0E+07 \\
G2    & 3,8E+04 & 1,9E+07 & -7,4  & 4,6E+06 & 0,1   & 4,6E+06 \\
G3    & 2,9E+04 & 3,6E+07 & 0,5   & 9,9E+06 & -7,3  & 9,9E+06 \\
H1    & 3,1E+04 & 3,6E+07 & 0,7   & 8,9E+07 & 0,7   & 8,9E+07 \\
H2    & 3,4E+04 & 8,1E+06 & 0,5   & 6,0E+06 & 8,4   & 6,0E+06 \\
H3    & 3,2E+04 & 7,3E+06 & 0,5   & 5,0E+06 & -7,4  & 5,0E+06 \\
I1    & 3,6E+04 & 1,2E+07 & -7,2  & 2,9E+06 & 3,9   & 2,9E+06 \\
I2    & 2,9E+04 & 1,2E+06 & -0,2  & 4,2E+05 & 0,7   & 4,2E+05 \\
I3    & 3,1E+04 & 3,3E+07 & 0,3   & 1,2E+07 & 0,2   & 1,2E+07 \\
J1    & 3,7E+04 & 1,2E+06 & 0,5   & 9,4E+05 & 0,5   & 9,4E+05 \\
J2    & 3,5E+04 & 1,5E+06 & 8,4   & 1,2E+06 & 0,5   & 1,2E+06 \\
J3    & 3,3E+04 & 1,3E+06 & 8,4   & 1,0E+06 & 0,5   & 1,0E+06 \\
K1    & 3,9E+04 & 6,9E+06 & -7,4  & 1,2E+06 & 0,8   & 1,2E+06 \\
K2    & 4,0E+04 & 6,7E+06 & 1,0   & 1,1E+06 & 0,5   & 1,1E+06 \\
K3    & 3,6E+04 & 5,8E+06 & 8,4   & 1,1E+06 & 0,6   & 1,1E+06 \\
L1    & 3,4E+04 & 2,9E+07 & 0,5   & 9,7E+06 & 0,4   & 9,7E+06 \\
L2    & 3,1E+04 & 1,5E+07 & 0,5   & 7,3E+06 & 5,0   & 7,3E+06 \\
L3    & 3,5E+04 & 1,3E+07 & 0,5   & 1,1E+07 & 0,5   & 1,1E+07 \\
L4    & 3,7E+04 & 1,8E+07 & 0,5   & 6,1E+06 & 0,0   & 6,1E+06 \\
L5    & 3,1E+04 & 5,1E+07 & 8,4   & 1,9E+06 & 1,0   & 1,9E+06 \\
L6    & 3,2E+04 & 2,6E+07 & 8,4   & 1,5E+06 & 0,5   & 1,5E+06 \\
\end{longtable}
\end{center}

% \begin{figure}[!h]
% \centering
% \includegraphics[scale=0.69]{fig/res/estatisticasMetXX03.png} 
% \caption[Metodologia XX: Variância do conjunto A]{Gráfico com
% comparativo da variância original do dado e dos resíduos gerados pelos modelos
% sem e com GARCH do conjunto A na Metodologia XX}
% \label{Figura:estatisticaAMet20}
% \end{figure}
% 
% \begin{figure}[!h]
% \centering
% \includegraphics[scale=0.69]{fig/res/estatisticasMetXX00.png} 
% \caption[Metodologia XX: Variância do conjunto B]{Gráfico com
% comparativo da variância original do dado e dos resíduos gerados pelos modelos
% sem e com GARCH do conjunto B na Metodologia XX}
% \label{Figura:estatisticaBMet20}
% \end{figure}
% 
% \begin{figure}[!h]
% \centering
% \includegraphics[scale=0.69]{fig/res/estatisticasMetXX01.png} 
% \caption[Metodologia XX: Variância do conjunto C]{Gráfico com
% comparativo da variância original do dado e dos resíduos gerados pelos modelos
% sem e com GARCH do conjunto C na Metodologia XX}
% \label{Figura:estatisticaCMet20}
% \end{figure}
% 
% \begin{figure}[!h]
% \centering
% \includegraphics[scale=0.69]{fig/res/estatisticasMetXX02.png} 
% \caption[Metodologia XX: Variância dos conjuntos D e E]{Gráfico com comparativo
% da variância original do dado e dos resíduos gerados pelos modelos sem e com
% GARCH dos conjuntos D e E na Metodologia XX}
% \label{Figura:estatisticaDEMet20}
% \end{figure}
% 
% \begin{figure}[!h]
% \centering
% \includegraphics[scale=0.69]{fig/res/estatisticasMetXX04.png} 
% \caption[Metodologia XX: Variância do conjunto F]{Gráfico com
% comparativo da variância original do dado e dos resíduos gerados pelos modelos
% sem e com GARCH do conjunto F na Metodologia XX}
% \label{Figura:estatisticaFMet20}
% \end{figure}
% 
% \begin{figure}[!h]
% \centering
% \includegraphics[scale=0.8, angle=90]{fig/res/estatisticasMetXX05.png} 
% \caption[Metodologia XX: Variância dos conjuntos G, H e I]{Gráfico com
% comparativo da variância original do dado e dos resíduos gerados pelos modelos
% sem e com GARCH dos conjuntos G, H e I na Metodologia XX}
% \label{Figura:estatisticaGHIMet20}
% \end{figure}
% 
% \begin{figure}[!h]
% \centering
% \includegraphics[scale=0.8, angle=90]{fig/res/estatisticasMetXX06.png} 
% \caption[Metodologia XX: Variância dos conjuntos J e K]{Gráfico
% com comparativo da variância original do dado e dos resíduos gerados pelos modelos
% sem e com GARCH dos conjuntos J e K na Metodologia XX}
% \label{Figura:estatisticaJKMet20}
% \end{figure}
% 
% \begin{figure}[!h]
% \centering
% \includegraphics[scale=0.8, angle=90]{fig/res/estatisticasMetXX07.png} 
% \caption[Metodologia XX: Variância do conjunto L]{Gráfico com
% comparativo da variância original do dado e dos resíduos gerados pelos modelos
% sem e com GARCH do conjunto L na Metodologia XX}
% \label{Figura:estatisticaLMet20}
% \end{figure}

% \begin{figure}[!h]
% \centering
% \includegraphics[scale=0.65]{fig/res/estatisticasMetXX08.png} 
% \caption[Metodologia XX: redução relativa da variância]{Gráfico com comparativo
% da redução relativa total da variância do resíduo sem e com a utilização do modelo GARCH na
% Metodologia XX}
% \label{Figura:estatisticaPizzaMet20}
% \end{figure}