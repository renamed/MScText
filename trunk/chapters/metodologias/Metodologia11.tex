
\begin{center}
\begin{longtable}{cccccc}
\toprule
\rowcolor{white}
\caption[Metodologia XI: comparativo de convergência de soluções]{Comparativo
   de quantidade de experimentos cujas soluções convergiram com e sem a
   utilização do GARCH na metodologia XI} \label{Tab:convergenciaMet11} \\
\midrule
   Cenário & \specialcell{Total experimentos} & Convergiram &
   \specialcell{Não convergiram} & \% sucesso \\
\midrule
\endfirsthead
%\multicolumn{8}{c}%
%{\tablename\ \thetable\ -- \textit{Continuação da página anterior}} \\
\midrule
\rowcolor{white}
   Cenário & \specialcell{Total experimentos} & Convergiram &
   \specialcell{Não convergiram} & \% sucesso \\
\toprule
\endhead
\midrule \\ % \multicolumn{8}{r}{\textit{Continua na próxima página}} \\
\endfoot
\bottomrule
\endlastfoot
	Sem GARCH & 39 & 39 & 0 & 100\% \\
	Com GARCH & 39 & 39 & 0 & 100\% \\
\end{longtable}
\end{center}

%%%%%%%%%%%%%%%%%%%%%%%%%%%%%%%%%%%%%%%%%%%%%%%%%%%%%%%%%%%%%%%%%%%%%%%%%%%%%%%%%%%%%%%%%
\begin{center}
\begin{longtable}{cccccc}
\toprule
\rowcolor{white}
\caption[Metodologia XI: Razão de compressão]{Razão de compressão dos
experimentos sem e com GARCH na Metodologia XI.
Valores em bytes.} \label{Tab:razaocompressaoMet} \\
\midrule
Conjunto & \specialcell{Tamanho \\Original} & \specialcell{Tamanho
\\Comprimido\\Com GARCH} & \specialcell{Tamanho
\\Comprimido\\Sem GARCH} & \specialcell{Razão \\Compressão
\\Sem GARCH} & \specialcell{Razão \\Compressão
\\Com GARCH} \\
\midrule
\endfirsthead
%\multicolumn{8}{c}%
%{\tablename\ \thetable\ -- \textit{Continuação da página anterior}} \\
\midrule
\rowcolor{white}
Conjunto & \specialcell{Tamanho \\Original} & \specialcell{Tamanho
\\Comprimido\\Com GARCH} & \specialcell{Tamanho
\\Comprimido\\Sem GARCH} & \specialcell{Razão \\Compressão
\\Sem GARCH} & \specialcell{Razão \\Compressão
\\Com GARCH} \\
\toprule
\endhead
\midrule \\ % \multicolumn{8}{r}{\textit{Continua na próxima página}} \\
\endfoot
\bottomrule
\endlastfoot
    A1    & 1152000 & 856545 & 858576 & 1,34  & 1,34 \\
    A2    & 1152000 & 802099 & 798943 & 1,44  & 1,44 \\
    A3    & 1152000 & 847813 & 844911 & 1,36  & 1,36 \\
    B1    & 518592 & 133361 & 169322 & 3,89  & 3,06 \\
    B2    & 518592 & 133361 & 169322 & 3,89  & 3,06 \\
    B3    & 518592 & 133361 & 169322 & 3,89  & 3,06 \\
    C1    & 288192 & 264062 & 261054 & 1,09  & 1,10 \\
    C2    & 288192 & 250907 & 244979 & 1,15  & 1,18 \\
    C3    & 288192 & 258755 & 256470 & 1,11  & 1,12 \\
    D1    & 331200 & 273642 & 272800 & 1,21  & 1,21 \\
    D2    & 331200 & 268948 & 266256 & 1,23  & 1,24 \\
    D3    & 331200 & 232585 & 246461 & 1,42  & 1,34 \\
    E1    & 33792 & 32569 & 32561 & 1,04  & 1,04 \\
    E2    & 33792 & 32727 & 32756 & 1,03  & 1,03 \\
    E3    & 33792 & 32751 & 32829 & 1,03  & 1,03 \\
    F1    & 220992 & 197978 & 197568 & 1,12  & 1,12 \\
    F2    & 220992 & 130887 & 126098 & 1,69  & 1,75 \\
    F3    & 220992 & 126840 & 124159 & 1,74  & 1,78 \\
    G1    & 139392 & 125258 & 125341 & 1,11  & 1,11 \\
    G2    & 139392 & 74114 & 71175 & 1,88  & 1,96 \\
    G3    & 139392 & 94546 & 94616 & 1,47  & 1,47 \\
    H1    & 360192 & 329017 & 328935 & 1,09  & 1,10 \\
    H2    & 360192 & 298639 & 295897 & 1,21  & 1,22 \\
    H3    & 360192 & 297406 & 297740 & 1,21  & 1,21 \\
    I1    & 221184 & 163059 & 160210 & 1,36  & 1,38 \\
    I2    & 221184 & 121015 & 130559 & 1,83  & 1,69 \\
    I3    & 221184 & 168751 & 168833 & 1,31  & 1,31 \\
    J1    & 591936 & 299654 & 315133 & 1,98  & 1,88 \\
    J2    & 591936 & 285205 & 314593 & 2,08  & 1,88 \\
    J3    & 591936 & 304686 & 321799 & 1,94  & 1,84 \\
    K1    & 288000 & 196269 & 201172 & 1,47  & 1,43 \\
    K2    & 288000 & 201664 & 200144 & 1,43  & 1,44 \\
    K3    & 288000 & 192825 & 194431 & 1,49  & 1,48 \\
    L1    & 480192 & 391905 & 391323 & 1,23  & 1,23 \\
    L2    & 480192 & 388929 & 402971 & 1,23  & 1,19 \\
    L3    & 480192 & 407820 & 404180 & 1,18  & 1,19 \\
    L4    & 480192 & 384451 & 396833 & 1,25  & 1,21 \\
    L5    & 480192 & 378081 & 376861 & 1,27  & 1,27 \\
    L6    & 480192 & 381242 & 380545 & 1,26  & 1,26 \\
\end{longtable}
\end{center}

% \begin{figure}[!h]
% \centering
% \includegraphics[scale=1, angle=90]{fig/res/razaocompMetXI00.png}
% \caption[Metodologia XI: razão de compressão dos conjuntos A, B e C]{Gráfico
% com comparativo da razão de compressão dos conjuntos A, B e C sem e com GARCH na
% Metodologia XI}
% \label{Figura:razaocompressaoABCMet11}
% \end{figure}
%  
% \begin{figure}[!h]
% \centering
% \includegraphics[scale=1, angle=90]{fig/res/razaocompMetXI01.png}
% \caption[Metodologia XI: razão de compressão dos conjuntos D, E e F]{Gráfico
% com comparativo da razão de compressão dos conjuntos D, E e F sem e com GARCH na
% Metodologia XI}
% \label{Figura:razaocompressaoDEFMet11}
% \end{figure}
% 
% \begin{figure}[!h]
% \centering
% \includegraphics[scale=1, angle=90]{fig/res/razaocompMetXI02.png}
% \caption[Metodologia XI: razão de compressão dos conjuntos G, H e I]{Gráfico
% com comparativo da razão de compressão dos conjuntos G, H e I sem e com GARCH na
% Metodologia XI}
% \label{Figura:razaocompressaoGHIMet11}
% \end{figure}
% 
% \begin{figure}[!h]
% \centering
% \includegraphics[scale=1, angle=90]{fig/res/razaocompMetXI03.png}
% \caption[Metodologia XI: razão de compressão dos conjuntos J, K e L]{Gráfico
% com comparativo da razão de compressão dos conjuntos J, K e L sem e com GARCH na
% Metodologia XI}
% \label{Figura:razaocompressaoJKLMet11}
% \end{figure}

% \begin{figure}[!h]
% \centering
% \includegraphics[scale=0.9]{fig/res/razaocompMetXI04.png}
% \caption[Metodologia XI: razão de compressão]{Gráfico com comparativo da razão
% de compressão na Metodologia XI}
% \label{Figura:razaocompressaoPizzaMet11}
% \end{figure}

\clearpage

\begin{center}
\begin{longtable}{cccc}
\toprule
\rowcolor{white}
\caption[Metodologia XI: evolução da entropia]{Evolução da entropia do dado
original e do resíduo calculado na metodologia XI}
\label{tab:EvolucaoEntropiaMet11}\\
\midrule
Conjunto & \specialcell{Entropia \\Inicial} & \specialcell{Entropia do
\\Resíduo sem GARC} & \specialcell{Entropia do
\\Resíduo com GARC}  \\
\midrule
\endfirsthead
%\multicolumn{8}{c}%
%{\tablename\ \thetable\ -- \textit{Continuação da página anterior}} \\
\midrule
\rowcolor{white}
Conjunto & \specialcell{Entropia \\Inicial} & \specialcell{Entropia do
\\Resíduo sem GARC} & \specialcell{Entropia do
\\Resíduo com GARC}  \\
\toprule
\endhead
\midrule \\ % \multicolumn{8}{r}{\textit{Continua na próxima página}} \\
\endfoot
\bottomrule 
\endlastfoot
    A1    & 11,30 & 10,94 & 10,92 \\
    A2    & 11,30 & 10,52 & 10,50 \\
    A3    & 11,27 & 10,64 & 10,63 \\
    B1    & 7,64  & 3,10  & 2,76 \\
    B2    & 7,64  & 3,10  & 2,76 \\
    B3    & 7,64  & 3,10  & 2,76 \\
    C1    & 12,34 & 12,02 & 11,99 \\
    C2    & 13,18 & 12,65 & 12,59 \\
    C3    & 13,17 & 12,65 & 12,64 \\
    D1    & 9,48  & 9,12  & 9,12 \\
    D2    & 12,38 & 11,23 & 11,16 \\
    D3    & 6,45  & 6,15  & 6,07 \\
    E1    & 10,80 & 10,80 & 10,80 \\
    E2    & 10,78 & 10,78 & 10,78 \\
    E3    & 10,80 & 10,80 & 10,80 \\
    F1    & 10,20 & 10,07 & 10,06 \\
    F2    & 8,20  & 7,07  & 7,05 \\
    F3    & 9,27  & 7,36  & 7,24 \\
    G1    & 12,03 & 11,71 & 11,72 \\
    G2    & 11,79 & 7,80  & 7,52 \\
    G3    & 12,06 & 9,23  & 9,24 \\
    H1    & 8,44  & 8,44  & 8,44 \\
    H2    & 12,29 & 12,15 & 12,14 \\
    H3    & 12,33 & 12,18 & 12,17 \\
    I1    & 8,14  & 7,62  & 7,60 \\
    I2    & 9,59  & 7,28  & 6,54 \\
    I3    & 8,15  & 7,79  & 7,78 \\
    J1    & 8,50  & 6,76  & 6,25 \\
    J2    & 8,52  & 6,78  & 6,33 \\
    J3    & 8,53  & 6,81  & 6,29 \\
    K1    & 10,94 & 10,07 & 10,01 \\
    K2    & 10,89 & 9,99  & 9,95 \\
    K3    & 10,87 & 9,97  & 9,93 \\
    L1    & 11,27 & 11,27 & 11,27 \\
    L2    & 11,08 & 11,08 & 11,08 \\
    L3    & 11,31 & 11,31 & 11,31 \\
    L4    & 12,80 & 11,97 & 11,80 \\
    L5    & 10,67 & 10,67 & 10,67 \\
    L6    & 11,58 & 11,58 & 11,58 \\

\end{longtable}
\end{center}

% \begin{figure}[!h]
% \centering
% \includegraphics[scale=0.8, angle=90]{fig/res/evolucaoentropiaMetXI00.png} 
% \caption[Metodologia XI: evolução da entropia nos conjuntos A, B e C]{Gráfico
% com comparativo da evolução da entropia dos conjuntos A, B e C sem e com GARCH na
% Metodologia XI}
% \label{Figura:evolucaoentropiaABCMet11}
% \end{figure}
% 
% \begin{figure}[!h]
% \centering
% \includegraphics[scale=0.8, angle=90]{fig/res/evolucaoentropiaMetXI01.png} 
% \caption[Metodologia XI: evolução da entropia nos conjuntos D, E e F]{Gráfico
% com comparativo da evolução da entropia dos conjuntos D, E e F sem e com GARCH na
% Metodologia XI}
% \label{Figura:evolucaoentropiaDEFMet11}
% \end{figure}
% 
% \begin{figure}[!h]
% \centering
% \includegraphics[scale=0.8, angle=90]{fig/res/evolucaoentropiaMetXI02.png} 
% \caption[Metodologia XI: evolução da entropia nos conjuntos G, H e I]{Gráfico
% com comparativo da evolução da entropia dos conjuntos G, H e I sem e com GARCH na
% Metodologia XI}
% \label{Figura:evolucaoentropiaGHIMet11}
% \end{figure}
% 
% \begin{figure}[!h]
% \centering
% \includegraphics[scale=0.6]{fig/res/evolucaoentropiaMetXI03.png} 
% \caption[Metodologia XI: evolução da entropia nos conjuntos J e K]{Gráfico com
% comparativo da evolução da entropia dos conjuntos J e K sem e com GARCH na
% Metodologia XI}
% \label{Figura:evolucaoentropiaJKMet11}
% \end{figure}
% 
% \begin{figure}[!h]
% \centering
% \includegraphics[scale=0.6]{fig/res/evolucaoentropiaMetXI04.png} 
% \caption[Metodologia XI: evolução da entropia nos conjuntos L]{Gráfico com
% comparativo da evolução da entropia dos conjuntos L sem e com GARCH na
% Metodologia XI}
% \label{Figura:evolucaoentropiaLMet11}
% \end{figure}
% 
% \begin{figure}[!h]
% \centering
% \includegraphics[scale=1]{fig/res/evolucaoentropiaMetXI05.png} 
% \caption[Metodologia XI: evolução da entropia]{Gráfico com comparativo da
% evolução da entopia na Metodologia XI}
% \label{Figura:evolucaoentropiaPizzaMet11}
% \end{figure}

\clearpage

\begin{center}
\begin{longtable}{ccccc|cccc}
\toprule
\rowcolor{white}
\caption[Metodologia XI: tempo de execução]{Tempo de execução (em segundos)
dos algoritmos sem e com GARCH na Metodologia XI. Primeiro é exibido o tempo de
execução sem a utilização do modelo GARCH, depois com o modelo. Parâmetros
modelo se refere ao tempo gasto pelo algoritmo para o cálculo dos parâmetros do
modelo, Resíduo refere-se ao tempo gasto pelo modelo para calcular o resíduo do
modelo, Cod. Arit. refere-se ao tempo gasto pela codificação aritmética para
comprimir o resíduo.} \label{tab:EvolucaoEntropiaMet11}\\
\midrule
Conj & \specialcell{Parâmetros\\modelo} &
Resíduo & \specialcell{Cod.\\Arit.} & \specialcell{Tempo\\total} &
\specialcell{Parâmetros\\modelo} &
Resíduo & \specialcell{Cod.\\Arit.} & \specialcell{Tempo\\total} \\
\midrule
\endfirsthead 
%\multicolumn{8}{c}%
%{\tablename\ \thetable\ -- \textit{Continuação da página anterior}} \\
\midrule
\rowcolor{white}
Conj & \specialcell{Parâmetros\\modelo} &
Resíduo & \specialcell{Cod.\\Arit.} & \specialcell{Tempo\\total} &
\specialcell{Parâmetros\\modelo} &
Resíduo & \specialcell{Cod.\\Arit.} & \specialcell{Tempo\\total} \\
\toprule
\endhead
\midrule \\ % \multicolumn{8}{r}{\textit{Continua na próxima página}} \\
\endfoot
\bottomrule 
\endlastfoot
A1&256&2&4&262&460&1&2&463\\
A2&71&2&3&76&1.272&3&8&1.283\\
A3&159&1&3&163&1.932&1&4&1.937\\
B1&30&1&3&34&590&1&$<1$&592\\
B2&35&1&1&37&583&1&2&586\\
B3&32&1&3&36&582&$<1$&$<1$&583\\
C1&7&$<1$&1&8&96&$<1$&1&97\\
C2&15&1&3&18&113&1&2&116\\
C3&13&$<1$&1&14&105&$<1$&1&105\\
D1&15&1&3&19&106&1&2&109\\
D2&15&1&3&19&335&$<1$&1&336\\
D3&11&1&2&14&311&1&2&314\\
E1&6&$<1$&$<1$&7&44&$<1$&$<1$&44\\
E2&4&$<1$&$<1$&4&40&$<1$&$<1$&40\\
E3&10&$<1$&$<1$&10&97&$<1$&$<1$&97\\
F1&14&$<1$&1&15&66&$<1$&2&68\\
F2&13&$<1$&$<1$&14&468&$<1$&$<1$&468\\
F3&13&$<1$&$<1$&14&408&$<1$&$<1$&408\\
G1&6&$<1$&$<1$&7&48&$<1$&1&50\\
G2&13&$<1$&$<1$&14&333&1&1&335\\
G3&4&$<1$&$<1$&5&165&$<1$&$<1$&166\\
H1&14&1&4&19&65&1&4&70\\
H2&25&$<1$&1&26&572&$<1$&1&574\\
H3&30&1&3&34&121&$<1$&1&122\\
I1&14&1&2&16&204&$<1$&1&204\\
I2&83&$<1$&$<1$&83&353&1&1&355\\
I3&13&$<1$&1&14&381&$<1$&1&382\\
J1&39&$<1$&1&40&849&2&1&851\\
J2&38&1&2&42&903&$<1$&1&904\\
J3&40&1&3&44&867&2&2&870\\
K1&14&$<1$&1&15&250&$<1$&1&251\\
K2&14&1&2&17&225&1&2&228\\
K3&15&$<1$&1&16&259&1&2&262\\
L1&315&1&1&317&839&1&2&841\\
L2&238&$<1$&4&242&710&$<1$&1&712\\
L3&298&1&5&304&872&2&5&879\\
L4&313&1&4&318&1.120&$<1$&1&1.122\\
L5&288&1&5&294&1.093&$<1$&1&1.095\\
L6&106&$<1$&1&107&319&1&5&325\\
\end{longtable}
\end{center}

% \begin{figure}[!h]
% \centering
% \includegraphics[scale=1, angle=90]{fig/res/tempoexecMetXI00.png} 
% \caption[Metodologia XI: tempo de cálculo dos parâmetros dos modelos dos
% conjuntos A, B, C e D]{Gráfico com comparativo do tempo de cálculo dos
% parâmetros dos modelos dos conjuntos A, B, C e D sem e com GARCH na Metodologia
% XI}
% \label{Figura:tempocalculoABCDMet11}
% \end{figure}
% 
% \begin{figure}[!h]
% \centering
% \includegraphics[scale=0.75]{fig/res/tempoexecMetXI01.png} 
% \caption[Metodologia XI: tempo de cálculo dos parâmetros dos modelos dos
% conjuntos E, F e G]{Gráfico com comparativo do tempo de cálculo dos
% parâmetros dos modelos dos conjuntos E, F e G sem e com GARCH na Metodologia
% XI}
% \label{Figura:tempocalculoEFGMet11}
% \end{figure}
% 
% \begin{figure}[!h]
% \centering
% \includegraphics[scale=0.75]{fig/res/tempoexecMetXI02.png} 
% \caption[Metodologia XI: tempo de cálculo dos parâmetros dos modelos dos
% conjuntos H, I e J]{Gráfico com comparativo do tempo de cálculo dos
% parâmetros dos modelos dos conjuntos H, I e J sem e com GARCH na Metodologia
% XI}
% \label{Figura:tempocalculoHIJMet11}
% \end{figure}
% 
% \begin{figure}[!h]
% \centering 
% \includegraphics[scale=1, angle=90]{fig/res/tempoexecMetXI03.png} 
% \caption[Metodologia XI: tempo de cálculo dos parâmetros dos modelos dos
% conjuntos K e L]{Gráfico com comparativo do tempo de cálculo dos
% parâmetros dos modelos dos conjuntos K e L sem e com GARCH na Metodologia XI}
% \label{Figura:tempocalculoKLMet11}
% \end{figure}

% \begin{figure}[!h]
% \centering
% \includegraphics[scale=0.75]{fig/res/tempoexecMetXI04.png} 
% \caption[Metodologia XI: tempo total relativo gasto no cálculo dos
% parâmetros do modelo]{Gráfico com comparativo do tempo total relativo de cálculo
% dos parâmetros dos modelos sem e com GARCH na Metodologia XI}
% \label{Figura:tempocalculoPizzaMet11}
% \end{figure}

\clearpage

\begin{center}
\begin{longtable}{ccccc|cccc}
\toprule
\rowcolor{white}
\caption[Metodologia XI: evolução da autocorrelação]{Autocorrelação do dado
original e dos resíduos gerados sem e com a utilização do modelo GARCH na
Metodologia XI} \label{tab:EvolucaoAutocorrelacaoMet11}\\
\midrule
Conjunto & \specialcell{Autocorrelação\\Inicial} & \specialcell{Autocorrelação\\Sem
GARCH} & \specialcell{Autocorrelação\\Com GARCH} \\
\midrule
\endfirsthead 
%\multicolumn{8}{c}%
%{\tablename\ \thetable\ -- \textit{Continuação da página anterior}} \\
\midrule
\rowcolor{white}
Conjunto & \specialcell{Autocorrelação\\Inicial} & \specialcell{Autocorrelação\\Sem
GARCH} & \specialcell{Autocorrelação\\Com GARCH} \\
\toprule
\endhead
\midrule \\ % \multicolumn{8}{r}{\textit{Continua na próxima página}} \\
\endfoot
\bottomrule 
\endlastfoot
    A1    & 6     & 0     & 0 \\
    A2    & 5     & 0     & 0 \\
    A3    & 6     & 0     & 0 \\
    B1    & 6     & 0     & 3 \\
    B2    & 6     & 0     & 3 \\
    B3    & 6     & 0     & 3 \\
    C1    & 2     & 0     & 0 \\
    C2    & 1     & 0     & 0 \\
    C3    & 2     & 0     & 0 \\
    D1    & 2     & 1     & 1 \\
    D2    & 2     & 2     & 0 \\
    D3    & 2     & 4     & 7 \\
    E1    & 4     & 0     & 0 \\
    E2    & 4     & 0     & 0 \\
    E3    & 4     & 0     & 0 \\
    F1    & 1     & 4     & 4 \\
    F2    & 6     & 0     & 1 \\
    F3    & 6     & 0     & 0 \\
    G1    & 1     & 1     & 1 \\
    G2    & 2     & 0     & 0 \\
    G3    & 6     & 0     & 0 \\
    H1    & 1     & 0     & 0 \\
    H2    & 1     & 0     & 0 \\
    H3    & 1     & 0     & 0 \\
    I1    & 7     & 0     & 0 \\
    I2    & 1     & 0     & 0 \\
    I3    & 1     & 0     & 0 \\
    J1    & 3     & 2     & 4 \\
    J2    & 3     & 2     & 0 \\
    J3    & 8     & 2     & 0 \\
    K1    & 3     & 2     & 1 \\
    K2    & 3     & 2     & 1 \\
    K3    & 2     & 0     & 1 \\
    L1    & 2     & 0     & 0 \\
    L2    & 7     & 0     & 2 \\
    L3    & 11    & 0     & 0 \\
    L4    & 7     & 0     & 1 \\
    L5    & 7     & 0     & 0 \\
    L6    & 6     & 0     & 0 \\
\end{longtable}
\end{center}

% \begin{figure}[!h]
% \centering
% \includegraphics[scale=0.75]{fig/res/evolucaoautocorrMetXI00.png} 
% \caption[Metodologia XI: evolução da autocorrelação nos conjuntos A, B e
% C]{Gráfico com comparativo da autocorrelação do resíduo gerado sem e com a
% utilização do modelo GARCH em relação ao dado original nos conjuntos A, B e C na
% Metodologia XI}
% \label{Figura:autocorrelacaoABCMet11}
% \end{figure}
% 
% \begin{figure}[!h]
% \centering
% \includegraphics[scale=0.69]{fig/res/evolucaoautocorrMetXI01.png} 
% \caption[Metodologia XI: evolução da autocorrelação nos conjuntos D, E e
% F]{Gráfico com comparativo da autocorrelação do resíduo gerado sem e com a
% utilização do modelo GARCH em relação ao dado original nos conjuntos D, E e F na
% Metodologia XI}
% \label{Figura:autocorrelacaoDEFMet11}
% \end{figure}
% 
% \begin{figure}[!h]
% \centering
% \includegraphics[scale=0.69]{fig/res/evolucaoautocorrMetXI02.png} 
% \caption[Metodologia XI: evolução da autocorrelação nos conjuntos G, H e
% I]{Gráfico com comparativo da autocorrelação do resíduo gerado sem e com a
% utilização do modelo GARCH em relação ao dado original nos conjuntos G, H e I na
% Metodologia XI}
% \label{Figura:autocorrelacaoGHIMet11}
% \end{figure}
% 
% \begin{figure}[!h]
% \centering
% \includegraphics[scale=0.69]{fig/res/evolucaoautocorrMetXI03.png} 
% \caption[Metodologia XI: evolução da autocorrelação nos conjuntos J e
% K]{Gráfico com comparativo da autocorrelação do resíduo gerado sem e com a
% utilização do modelo GARCH em relação ao dado original nos conjuntos J e K na
% Metodologia XI}
% \label{Figura:autocorrelacaoJKMet11}
% \end{figure}
% 
% \begin{figure}[!h]
% \centering
% \includegraphics[scale=0.69]{fig/res/evolucaoautocorrMetXI04.png} 
% \caption[Metodologia XI: evolução da autocorrelação nos conjuntos L]{Gráfico
% com comparativo da autocorrelação do resíduo gerado sem e com a utilização do modelo GARCH em relação ao dado original nos conjuntos L na
% Metodologia XI}
% \label{Figura:autocorrelacaoLMet11}
% \end{figure}

% \begin{figure}[!h]
% \centering
% \includegraphics[scale=0.75]{fig/res/evolucaoautocorrMetXI05.png} 
% \caption[Metodologia XI: tempo total relativo gasto no cálculo dos
% parâmetros do modelo]{Gráfico com comparativo da redução relativa total da
% autocorrelação do resíduo sem e com a utilização do modelo GARCH na
% Metodologia XI}
% \label{Figura:tempocalculoPizzaMet11}
% \end{figure}

\clearpage

\begin{center}
\begin{longtable}{ccccccccc}
\toprule
\rowcolor{white}
\caption[Metodologia XI: dados estatísticos]{Média e variância do dado original
comparadas às do resíduo calculado sem e com a utilização do modelo GARCH na
Metodologia XI} \label{tab:DadosEstatisticosMet11}\\
\midrule
    Conjunto & \specialcell{Média\\Original} &
    \specialcell{Var.\\Original} & \specialcell{Média\\Sem\\GARCH} &
    \specialcell{Var.\\Sem\\GARCH} & \specialcell{Média\\Com\\GARCH}&
    \specialcell{Var.\\Com\\GARCH} \\

\midrule
\endfirsthead 
%\multicolumn{8}{c}%
%{\tablename\ \thetable\ -- \textit{Continuação da página anterior}} \\
\midrule
\rowcolor{white}
    Conjunto & \specialcell{Média\\Orig.} &
    \specialcell{Var.\\Orig.} & \specialcell{Média\\Sem\\GARCH} &
    \specialcell{Var.\\Sem\\GARCH} & \specialcell{Média\\Com\\GARCH}&
    \specialcell{Var.\\Com\\GARCH} \\

\toprule
\endhead
\midrule \\ % \multicolumn{8}{r}{\textit{Continua na próxima página}} \\
\endfoot
\bottomrule 
\endlastfoot
A1    & 3,0E+04 & 1,8E+07 & 0,5   & 3,0E+05 & 2,2   & 3,0E+05 \\
A2    & 3,2E+04 & 1,1E+07 & 0,5   & 2,6E+05 & -0,5  & 2,7E+05 \\
A3    & 3,1E+04 & 1,4E+07 & 0,5   & 2,8E+05 & -0,2  & 2,8E+05 \\
B1    & 2,8E+04 & 4,5E+05 & 0,4   & 1,1E+01 & 0,4   & 1,0E+02 \\
B2    & 2,8E+04 & 4,5E+05 & 0,4   & 1,1E+01 & 0,4   & 1,0E+02 \\
B3    & 2,8E+04 & 4,5E+05 & 0,4   & 1,1E+01 & 0,4   & 1,0E+02 \\
C1    & 3,3E+04 & 8,1E+07 & 0,6   & 2,0E+07 & -0,9  & 2,3E+07 \\
C2    & 3,3E+04 & 4,0E+07 & 0,5   & 1,2E+07 & 0,6   & 1,4E+07 \\
C3    & 3,3E+04 & 5,7E+07 & 0,5   & 1,1E+07 & 0,7   & 1,2E+07 \\
D1    & 3,7E+04 & 4,1E+07 & 0,7   & 2,4E+06 & -0,8  & 2,4E+06 \\
D2    & 3,3E+04 & 1,2E+07 & 0,5   & 1,5E+06 & 0,4   & 2,0E+06 \\
D3    & 3,1E+04 & 1,0E+07 & 0,2   & 1,2E+06 & 0,2   & 1,7E+06 \\
E1    & 2,9E+04 & 5,8E+07 & -0,1  & 1,7E+07 & -6,8  & 1,7E+07 \\
E2    & 3,0E+04 & 5,8E+07 & 0,3   & 1,8E+07 & 14,2  & 1,8E+07 \\
E3    & 3,0E+04 & 6,0E+07 & 0,2   & 1,9E+07 & 2,5   & 1,9E+07 \\
F1    & 3,8E+04 & 3,9E+07 & 0,5   & 1,5E+07 & 0,8   & 1,5E+07 \\
F2    & 2,3E+04 & 5,4E+06 & 0,6   & 1,8E+05 & 1,1   & 2,9E+05 \\
F3    & 2,6E+04 & 6,0E+06 & 0,6   & 6,9E+04 & 0,6   & 8,5E+04 \\
G1    & 3,3E+04 & 3,3E+07 & 0,5   & 8,7E+06 & 0,7   & 8,7E+06 \\
G2    & 3,8E+04 & 1,9E+07 & 0,5   & 1,6E+04 & 2,1   & 2,4E+04 \\
G3    & 2,9E+04 & 3,6E+07 & 0,5   & 6,7E+04 & 1,2   & 6,7E+04 \\
H1    & 3,1E+04 & 3,6E+07 & 0,7   & 5,2E+07 & 0,3   & 5,3E+07 \\
H2    & 3,4E+04 & 8,1E+06 & 0,5   & 4,3E+06 & -8,1  & 4,6E+06 \\
H3    & 3,2E+04 & 7,3E+06 & 0,5   & 5,2E+06 & 0,4   & 5,4E+06 \\
I1    & 3,6E+04 & 1,2E+07 & 0,3   & 7,0E+05 & 7,1   & 7,7E+05 \\
I2    & 2,9E+04 & 1,2E+06 & 0,2   & 6,1E+04 & 0,2   & 1,7E+05 \\
I3    & 3,1E+04 & 3,3E+07 & 0,3   & 2,7E+06 & 0,3   & 2,8E+06 \\
J1    & 3,7E+04 & 1,2E+06 & 0,5   & 1,3E+04 & 0,5   & 5,0E+04 \\
J2    & 3,5E+04 & 1,5E+06 & 0,5   & 1,8E+04 & 0,5   & 6,4E+04 \\
J3    & 3,3E+04 & 1,3E+06 & 0,5   & 1,4E+04 & 0,5   & 5,8E+04 \\
K1    & 3,9E+04 & 6,9E+06 & 0,5   & 4,4E+05 & 0,6   & 5,5E+05 \\
K2    & 4,0E+04 & 6,7E+06 & 0,5   & 4,3E+05 & 0,4   & 5,5E+05 \\
K3    & 3,6E+04 & 5,8E+06 & 0,5   & 5,2E+05 & 0,6   & 5,8E+05 \\
L1    & 3,4E+04 & 2,9E+07 & 0,6   & 2,2E+06 & -0,6  & 2,7E+06 \\
L2    & 3,1E+04 & 1,5E+07 & 0,2   & 4,3E+06 & 1,6   & 5,4E+06 \\
L3    & 3,5E+04 & 1,3E+07 & 0,4   & 4,9E+06 & -3,8  & 4,9E+06 \\
L4    & 3,7E+04 & 1,8E+07 & 0,1   & 2,6E+06 & 2,8   & 4,4E+06 \\
L5    & 3,1E+04 & 5,1E+07 & 0,5   & 1,0E+06 & 0,3   & 1,0E+06 \\
L6    & 3,2E+04 & 2,6E+07 & 0,7   & 1,1E+06 & -0,1  & 1,1E+06 \\
\end{longtable}
\end{center}

% \begin{figure}[!h]
% \centering
% \includegraphics[scale=0.69]{fig/res/estatisticasMetXI03.png} 
% \caption[Metodologia XI: Variância do conjunto A]{Gráfico com
% comparativo da variância original do dado e dos resíduos gerados pelos modelos
% sem e com GARCH do conjunto A na Metodologia XI}
% \label{Figura:estatisticaAMet11}
% \end{figure}
% 
% \begin{figure}[!h]
% \centering
% \includegraphics[scale=0.69]{fig/res/estatisticasMetXI00.png} 
% \caption[Metodologia XI: Variância do conjunto B]{Gráfico com
% comparativo da variância original do dado e dos resíduos gerados pelos modelos
% sem e com GARCH do conjunto B na Metodologia XI}
% \label{Figura:estatisticaBMet11}
% \end{figure}
% 
% \begin{figure}[!h]
% \centering
% \includegraphics[scale=0.69]{fig/res/estatisticasMetXI01.png} 
% \caption[Metodologia XI: Variância do conjunto C]{Gráfico com
% comparativo da variância original do dado e dos resíduos gerados pelos modelos
% sem e com GARCH do conjunto C na Metodologia XI}
% \label{Figura:estatisticaCMet11}
% \end{figure}
% 
% \begin{figure}[!h]
% \centering
% \includegraphics[scale=0.69]{fig/res/estatisticasMetXI02.png} 
% \caption[Metodologia XI: Variância dos conjuntos D e E]{Gráfico com comparativo
% da variância original do dado e dos resíduos gerados pelos modelos sem e com
% GARCH dos conjuntos D e E na Metodologia XI}
% \label{Figura:estatisticaDEMet11}
% \end{figure}
% 
% \begin{figure}[!h]
% \centering
% \includegraphics[scale=0.69]{fig/res/estatisticasMetXI04.png} 
% \caption[Metodologia XI: Variância do conjunto F]{Gráfico com
% comparativo da variância original do dado e dos resíduos gerados pelos modelos
% sem e com GARCH do conjunto F na Metodologia XI}
% \label{Figura:estatisticaFMet11}
% \end{figure}
% 
% \begin{figure}[!h]
% \centering
% \includegraphics[scale=0.8, angle=90]{fig/res/estatisticasMetXI05.png} 
% \caption[Metodologia XI: Variância dos conjuntos G, H e I]{Gráfico com
% comparativo da variância original do dado e dos resíduos gerados pelos modelos
% sem e com GARCH dos conjuntos G, H e I na Metodologia XI}
% \label{Figura:estatisticaGHIMet11}
% \end{figure}
% 
% \begin{figure}[!h]
% \centering
% \includegraphics[scale=0.8, angle=90]{fig/res/estatisticasMetXI06.png} 
% \caption[Metodologia XI: Variância dos conjuntos J e K]{Gráfico
% com comparativo da variância original do dado e dos resíduos gerados pelos modelos
% sem e com GARCH dos conjuntos J e K na Metodologia XI}
% \label{Figura:estatisticaJKMet11}
% \end{figure}
% 
% \begin{figure}[!h]
% \centering
% \includegraphics[scale=0.8, angle=90]{fig/res/estatisticasMetXI07.png} 
% \caption[Metodologia XI: Variância do conjunto L]{Gráfico com
% comparativo da variância original do dado e dos resíduos gerados pelos modelos
% sem e com GARCH do conjunto L na Metodologia XI}
% \label{Figura:estatisticaLMet11}
% \end{figure}

% \begin{figure}[!h]
% \centering
% \includegraphics[scale=0.65]{fig/res/estatisticasMetXI08.png} 
% \caption[Metodologia XI: redução relativa da variância]{Gráfico com comparativo
% da redução relativa total da variância do resíduo sem e com a utilização do modelo GARCH na
% Metodologia XI}
% \label{Figura:estatisticaPizzaMet11}
% \end{figure}