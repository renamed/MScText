
\begin{center}
\begin{longtable}{cccccc}
\toprule
\rowcolor{white}
\caption[Metodologia XIII: comparativo de convergência de soluções]{Comparativo
   de quantidade de experimentos cujas soluções convergiram com e sem a
   utilização do GARCH na metodologia XIII} \label{Tab:convergenciaMet13} \\
\midrule
   Cenário & \specialcell{Total experimentos} & Convergiram &
   \specialcell{Não convergiram} & \% sucesso \\
\midrule
\endfirsthead
%\multicolumn{8}{c}%
%{\tablename\ \thetable\ -- \textit{Continuação da página anterior}} \\
\midrule
\rowcolor{white}
   Cenário & \specialcell{Total experimentos} & Convergiram &
   \specialcell{Não convergiram} & \% sucesso \\
\toprule
\endhead
\midrule \\ % \multicolumn{8}{r}{\textit{Continua na próxima página}} \\
\endfoot
\bottomrule
\endlastfoot
	Sem GARCH & 39 & 39 & 0 & 100\% \\
	Com GARCH & 39 & 37 & 2 & 94,87\% \\
\end{longtable}
\end{center}

%%%%%%%%%%%%%%%%%%%%%%%%%%%%%%%%%%%%%%%%%%%%%%%%%%%%%%%%%%%%%%%%%%%%%%%%%%%%%%%%%%%%%%%%%
\begin{center}
\begin{longtable}{cccccc}
\toprule
\rowcolor{white}
\caption[Metodologia XIII: Razão de compressão]{Razão de compressão dos
experimentos sem e com GARCH na Metodologia XIII.
Valores em bytes.} \label{Tab:razaocompressaoMet} \\
\midrule
Conjunto & \specialcell{Tamanho \\Original} & \specialcell{Tamanho
\\Comprimido\\Com GARCH} & \specialcell{Tamanho
\\Comprimido\\Sem GARCH} & \specialcell{Razão \\Compressão
\\Sem GARCH} & \specialcell{Razão \\Compressão
\\Com GARCH} \\
\midrule
\endfirsthead
%\multicolumn{8}{c}%
%{\tablename\ \thetable\ -- \textit{Continuação da página anterior}} \\
\midrule
\rowcolor{white}
Conjunto & \specialcell{Tamanho \\Original} & \specialcell{Tamanho
\\Comprimido\\Com GARCH} & \specialcell{Tamanho
\\Comprimido\\Sem GARCH} & \specialcell{Razão \\Compressão
\\Sem GARCH} & \specialcell{Razão \\Compressão
\\Com GARCH} \\
\toprule
\endhead
\midrule \\ % \multicolumn{8}{r}{\textit{Continua na próxima página}} \\
\endfoot
\bottomrule
\endlastfoot
    A1    & 1152000 & 885269 & 885744 & 1,30  & 1,30 \\
    A2    & 1152000 & 888539 & 885606 & 1,30  & 1,30 \\
    A3    & 1152000 & 862568 & 863281 & 1,34  & 1,33 \\
    B1    & 518592 & 171927 & 171991 & 3,02  & 3,02 \\
    B2    & 518592 & 171927 & 171991 & 3,02  & 3,02 \\
    B3    & 518592 & 171927 & 171991 & 3,02  & 3,02 \\
    C1    & 288192 & 292443 & 292798 & 0,99  & 0,98 \\
    C2    & 288192 & 275156 & 273898 & 1,05  & 1,05 \\
    C3    & 288192 & 282897 & 282600 & 1,02  & 1,02 \\
    D1    & 331200 & 308738 & 308963 & 1,07  & 1,07 \\
    D2    & 331200 & 294281 & 294358 & 1,13  & 1,13 \\
    D3    & 331200 & 274436 & 274323 & 1,21  & 1,21 \\
    E1    & 33792 & 36227 & 36288 & 0,93  & 0,93 \\
    E2    & 33792 & 36289 & 36352 & 0,93  & 0,93 \\
    E3    & 33792 & 34297 & 34347 & 0,99  & 0,98 \\
    F1    & 220992 & 210456 & 210362 & 1,05  & 1,05 \\
    F2    & 220992 & 157768 & 152991 & 1,40  & 1,44 \\
    F3    & 220992 & 140826 & 151823 & 1,57  & 1,46 \\
    G1    & 139392 & 134184 & 134153 & 1,04  & 1,04 \\
    G2    & 139392 & 130807 & 130906 & 1,07  & 1,06 \\
    G3    & 139392 & 135142 & 135215 & 1,03  & 1,03 \\
    H1    & 360192 & 352413 & 352484 & 1,02  & 1,02 \\
    H2    & 360192 & 317409 & 317619 & 1,13  & 1,13 \\
    H3    & 360192 & 320116 & 320343 & 1,13  & 1,12 \\
    I1    & 221184 & 190123 & 190395 & 1,16  & 1,16 \\
    I2    & 221184 & 148700 & 150820 & 1,49  & 1,47 \\
    I3    & 221184 & 197306 & 197265 & 1,12  & 1,12 \\
    J1    & 591936 & 322818 & 322882 & 1,83  & 1,83 \\
    J2    & 591936 & 390329 & 386644 & 1,52  & 1,53 \\
    J3    & 591936 & 353439 & 353503 & 1,67  & 1,67 \\
    K1    & 288000 & 209319 & 213897 & 1,38  & 1,35 \\
    K2    & 288000 & 208785 & 201350 & 1,38  & 1,43 \\
    K3    & 288000 & 229327 & 217941 & 1,26  & 1,32 \\
    L1    & 480192 & 443419 & 440569 & 1,08  & 1,09 \\
    L2    & 480192 & 445429 & 441060 & 1,08  & 1,09 \\
    L3    & 480192 & 453451 & 451755 & 1,06  & 1,06 \\
    L4    & 480192 & 442098 & 425185 & 1,09  & 1,13 \\
    L5    & 480192 & 407081 & 405680 & 1,18  & 1,18 \\
    L6    & 480192 & 418976 & 416322 & 1,15  & 1,15 \\
\end{longtable}
\end{center}

% \begin{figure}[!h]
% \centering
% \includegraphics[scale=1, angle=90]{fig/res/razaocompMetXIII00.png}
% \caption[Metodologia XIII: razão de compressão dos conjuntos A, B e C]{Gráfico
% com comparativo da razão de compressão dos conjuntos A, B e C sem e com GARCH na
% Metodologia XIII}
% \label{Figura:razaocompressaoABCMet13}
% \end{figure}
%  
% \begin{figure}[!h]
% \centering
% \includegraphics[scale=1, angle=90]{fig/res/razaocompMetXIII01.png}
% \caption[Metodologia XIII: razão de compressão dos conjuntos D, E e F]{Gráfico
% com comparativo da razão de compressão dos conjuntos D, E e F sem e com GARCH na
% Metodologia XIII}
% \label{Figura:razaocompressaoDEFMet13}
% \end{figure}
% 
% \begin{figure}[!h]
% \centering
% \includegraphics[scale=1, angle=90]{fig/res/razaocompMetXIII02.png}
% \caption[Metodologia XIII: razão de compressão dos conjuntos G, H e I]{Gráfico
% com comparativo da razão de compressão dos conjuntos G, H e I sem e com GARCH na
% Metodologia XIII}
% \label{Figura:razaocompressaoGHIMet13}
% \end{figure}
% 
% \begin{figure}[!h]
% \centering
% \includegraphics[scale=1, angle=90]{fig/res/razaocompMetXIII03.png}
% \caption[Metodologia XIII: razão de compressão dos conjuntos J, K e L]{Gráfico
% com comparativo da razão de compressão dos conjuntos J, K e L sem e com GARCH na
% Metodologia XIII}
% \label{Figura:razaocompressaoJKLMet13}
% \end{figure}

% \begin{figure}[!h]
% \centering
% \includegraphics[scale=0.9]{fig/res/razaocompMetXIII04.png}
% \caption[Metodologia XIII: razão de compressão]{Gráfico com comparativo da razão
% de compressão na Metodologia XIII}
% \label{Figura:razaocompressaoPizzaMet13}
% \end{figure}

\clearpage

\begin{center}
\begin{longtable}{cccc}
\toprule
\rowcolor{white}
\caption[Metodologia XIII: evolução da entropia]{Evolução da entropia do dado
original e do resíduo calculado na metodologia XIII}
\label{tab:EvolucaoEntropiaMet13}\\
\midrule
Conjunto & \specialcell{Entropia \\Inicial} & \specialcell{Entropia do
\\Resíduo sem GARC} & \specialcell{Entropia do
\\Resíduo com GARC}  \\
\midrule
\endfirsthead
%\multicolumn{8}{c}%
%{\tablename\ \thetable\ -- \textit{Continuação da página anterior}} \\
\midrule
\rowcolor{white}
Conjunto & \specialcell{Entropia \\Inicial} & \specialcell{Entropia do
\\Resíduo sem GARC} & \specialcell{Entropia do
\\Resíduo com GARC}  \\
\toprule
\endhead
\midrule \\ % \multicolumn{8}{r}{\textit{Continua na próxima página}} \\
\endfoot
\bottomrule 
\endlastfoot
    A1    & 11,30 & 10,76 & 10,76 \\
    A2    & 11,30 & 10,38 & 10,36 \\
    A3    & 11,27 & 10,60 & 10,52 \\
    B1    & 7,64  & 2,91  & 2,48 \\
    B2    & 7,64  & 2,91  & 2,48 \\
    B3    & 7,64  & 2,91  & 2,48 \\
    C1    & 12,34 & 12,30 & 12,12 \\
    C2    & 13,18 & 12,53 & 12,21 \\
    C3    & 13,17 & 12,59 & 12,44 \\
    D1    & 9,48  & 8,80  & 8,74 \\
    D2    & 12,38 & 10,83 & 10,41 \\
    D3    & 6,45  & 6,35  & 5,97 \\
    E1    & 10,80 & 10,80 & 10,80 \\
    E2    & 10,78 & 10,78 & 10,78 \\
    E3    & 10,80 & 10,80 & 10,80 \\
    F1    & 10,20 & 10,20 & 10,20 \\
    F2    & 8,20  & 7,75  & 7,45 \\
    F3    & 9,27  & 7,95  & 7,25 \\
    G1    & 12,03 & 11,71 & 11,71 \\
    G2    & 11,79 & 7,71  & 7,63 \\
    G3    & 12,06 & 8,74  & 7,65 \\
    H1    & 8,44  & 8,44  & 8,44 \\
    H2    & 12,29 & 12,15 & 12,29 \\
    H3    & 12,33 & 12,20 & 12,20 \\
    I1    & 8,14  & 8,14  & 8,14 \\
    I2    & 9,59  & 6,98  & 6,47 \\
    I3    & 8,15  & 8,15  & 8,15 \\
    J1    & 8,50  & 7,11  & 8,50 \\
    J2    & 8,52  & 7,40  & 6,80 \\
    J3    & 8,53  & 7,04  & 6,19 \\
    K1    & 10,94 & 9,78  & 9,73 \\
    K2    & 10,89 & 9,94  & 9,60 \\
    K3    & 10,87 & 9,70  & 9,58 \\
    L1    & 11,27 & 11,27 & 11,27 \\
    L2    & 11,08 & 11,08 & 11,08 \\
    L3    & 11,31 & 11,31 & 11,31 \\
    L4    & 12,80 & 11,89 & 11,60 \\
    L5    & 10,67 & 10,67 & 10,67 \\
    L6    & 11,58 & 11,58 & 11,58 \\


\end{longtable}
\end{center}

% \begin{figure}[!h]
% \centering
% \includegraphics[scale=0.8, angle=90]{fig/res/evolucaoentropiaMetXIII00.png} 
% \caption[Metodologia XIII: evolução da entropia nos conjuntos A, B e C]{Gráfico
% com comparativo da evolução da entropia dos conjuntos A, B e C sem e com GARCH na
% Metodologia XIII}
% \label{Figura:evolucaoentropiaABCMet13}
% \end{figure}
% 
% \begin{figure}[!h]
% \centering
% \includegraphics[scale=0.8, angle=90]{fig/res/evolucaoentropiaMetXIII01.png} 
% \caption[Metodologia XIII: evolução da entropia nos conjuntos D, E e F]{Gráfico
% com comparativo da evolução da entropia dos conjuntos D, E e F sem e com GARCH na
% Metodologia XIII}
% \label{Figura:evolucaoentropiaDEFMet13}
% \end{figure}
% 
% \begin{figure}[!h]
% \centering
% \includegraphics[scale=0.8, angle=90]{fig/res/evolucaoentropiaMetXIII02.png} 
% \caption[Metodologia XIII: evolução da entropia nos conjuntos G, H e I]{Gráfico
% com comparativo da evolução da entropia dos conjuntos G, H e I sem e com GARCH na
% Metodologia XIII}
% \label{Figura:evolucaoentropiaGHIMet13}
% \end{figure}
% 
% \begin{figure}[!h]
% \centering
% \includegraphics[scale=0.6]{fig/res/evolucaoentropiaMetXIII03.png} 
% \caption[Metodologia XIII: evolução da entropia nos conjuntos J e K]{Gráfico com
% comparativo da evolução da entropia dos conjuntos J e K sem e com GARCH na
% Metodologia XIII}
% \label{Figura:evolucaoentropiaJKMet13}
% \end{figure}
% 
% \begin{figure}[!h]
% \centering
% \includegraphics[scale=0.6]{fig/res/evolucaoentropiaMetXIII04.png} 
% \caption[Metodologia XIII: evolução da entropia nos conjuntos L]{Gráfico com
% comparativo da evolução da entropia dos conjuntos L sem e com GARCH na
% Metodologia XIII}
% \label{Figura:evolucaoentropiaLMet13}
% \end{figure}

% \begin{figure}[!h]
% \centering
% \includegraphics[scale=1]{fig/res/evolucaoentropiaMetXIII05.png} 
% \caption[Metodologia XIII: evolução da entropia]{Gráfico com comparativo da
% evolução da entopia na Metodologia XIII}
% \label{Figura:evolucaoentropiaPizzaMet13}
% \end{figure}

\clearpage

\begin{center}
\begin{longtable}{ccccc|cccc}
\toprule
\rowcolor{white}
\caption[Metodologia XIII: tempo de execução]{Tempo de execução (em segundos)
dos algoritmos sem e com GARCH na Metodologia XIII. Primeiro é exibido o tempo de
execução sem a utilização do modelo GARCH, depois com o modelo. Parâmetros
modelo se refere ao tempo gasto pelo algoritmo para o cálculo dos parâmetros do
modelo, Resíduo refere-se ao tempo gasto pelo modelo para calcular o resíduo do
modelo, Cod. Arit. refere-se ao tempo gasto pela codificação aritmética para
comprimir o resíduo.} \label{tab:EvolucaoEntropiaMet13}\\
\midrule
Conj & \specialcell{Parâmetros\\modelo} &
Resíduo & \specialcell{Cod.\\Arit.} & \specialcell{Tempo\\total} &
\specialcell{Parâmetros\\modelo} &
Resíduo & \specialcell{Cod.\\Arit.} & \specialcell{Tempo\\total} \\
\midrule
\endfirsthead 
%\multicolumn{8}{c}%
%{\tablename\ \thetable\ -- \textit{Continuação da página anterior}} \\
\midrule
\rowcolor{white}
Conj & \specialcell{Parâmetros\\modelo} &
Resíduo & \specialcell{Cod.\\Arit.} & \specialcell{Tempo\\total} &
\specialcell{Parâmetros\\modelo} &
Resíduo & \specialcell{Cod.\\Arit.} & \specialcell{Tempo\\total} \\
\toprule
\endhead
\midrule \\ % \multicolumn{8}{r}{\textit{Continua na próxima página}} \\
\endfoot
\bottomrule 
\endlastfoot
A1&816&$<1$&2&819&1.181&1&2&1.184\\
A2&1.504&2&4&1.510&2.322&1&2&2.325\\
A3&1.514&1&5&1.519&3.036&1&4&3.041\\
B1&683&$<1$&1&684&1.373&2&3&1.377\\
B2&668&1&2&671&1.554&$<1$&1&1.555\\
B3&670&$<1$&1&671&1.361&1&3&1.365\\
C1&89&1&3&93&413&$<1$&1&414\\
C2&95&1&2&98&347&$<1$&$<1$&348\\
C3&80&$<1$&1&81&199&$<1$&$<1$&200\\
D1&172&1&2&175&214&$<1$&1&215\\
D2&182&$<1$&1&183&464&$<1$&$<1$&465\\
D3&91&$<1$&1&92&460&$<1$&1&462\\
E1&28&$<1$&$<1$&28&80&$<1$&$<1$&80\\
E2&22&$<1$&$<1$&23&84&$<1$&$<1$&85\\
E3&43&$<1$&$<1$&44&98&$<1$&$<1$&98\\
F1&25&$<1$&1&26&83&1&2&85\\
F2&210&1&1&212&808&$<1$&$<1$&809\\
F3&352&1&1&353&739&$<1$&$<1$&739\\
G1&17&$<1$&2&19&85&$<1$&1&86\\
G2&485&1&$<1$&486&975&$<1$&$<1$&976\\
G3&177&$<1$&1&178&460&$<1$&$<1$&460\\
H1&69&$<1$&4&73&232&$<1$&1&234\\
H2&510&1&3&514&690&$<1$&21&711\\
H3&33&1&3&37&133&1&3&138\\
I1&298&$<1$&1&299&727&1&2&730\\
I2&307&$<1$&$<1$&307&663&1&1&664\\
I3&213&1&1&215&452&1&2&455\\
J1&1.020&1&2&1.024&2.052&$<1$&14&2.066\\
J2&1.026&$<1$&1&1.028&2.016&2&3&2.021\\
J3&1.056&1&5&1.061&2.099&1&1&2.101\\
K1&226&1&2&228&531&$<1$&1&533\\
K2&230&$<1$&1&231&523&$<1$&1&524\\
K3&225&$<1$&1&226&530&1&2&533\\
L1&1.424&1&4&1.429&2.569&2&3&2.574\\
L2&1.010&1&3&1.014&2.080&2&2&2.084\\
L3&1.442&$<1$&2&1.445&2.629&2&4&2.634\\
L4&1.333&1&4&1.338&2.408&$<1$&3&2.411\\
L5&1.999&1&1&2.002&2.291&$<1$&1&2.293\\
L6&643&$<1$&4&648&1.282&$<1$&2&1.284\\
\end{longtable}
\end{center}

% \begin{figure}[!h]
% \centering
% \includegraphics[scale=1, angle=90]{fig/res/tempoexecMetXIII00.png} 
% \caption[Metodologia XIII: tempo de cálculo dos parâmetros dos modelos dos
% conjuntos A, B, C e D]{Gráfico com comparativo do tempo de cálculo dos
% parâmetros dos modelos dos conjuntos A, B, C e D sem e com GARCH na Metodologia
% XIII}
% \label{Figura:tempocalculoABCDMet13}
% \end{figure}
% 
% \begin{figure}[!h]
% \centering
% \includegraphics[scale=0.75]{fig/res/tempoexecMetXIII01.png} 
% \caption[Metodologia XIII: tempo de cálculo dos parâmetros dos modelos dos
% conjuntos E, F e G]{Gráfico com comparativo do tempo de cálculo dos
% parâmetros dos modelos dos conjuntos E, F e G sem e com GARCH na Metodologia
% XIII}
% \label{Figura:tempocalculoEFGMet13}
% \end{figure}
% 
% \begin{figure}[!h]
% \centering
% \includegraphics[scale=0.75]{fig/res/tempoexecMetXIII02.png} 
% \caption[Metodologia XIII: tempo de cálculo dos parâmetros dos modelos dos
% conjuntos H, I e J]{Gráfico com comparativo do tempo de cálculo dos
% parâmetros dos modelos dos conjuntos H, I e J sem e com GARCH na Metodologia
% XIII}
% \label{Figura:tempocalculoHIJMet13}
% \end{figure}
% 
% \begin{figure}[!h]
% \centering 
% \includegraphics[scale=1, angle=90]{fig/res/tempoexecMetXIII03.png} 
% \caption[Metodologia XIII: tempo de cálculo dos parâmetros dos modelos dos
% conjuntos K e L]{Gráfico com comparativo do tempo de cálculo dos
% parâmetros dos modelos dos conjuntos K e L sem e com GARCH na Metodologia XIII}
% \label{Figura:tempocalculoKLMet13}
% \end{figure}

% \begin{figure}[!h]
% \centering
% \includegraphics[scale=0.75]{fig/res/tempoexecMetXIII04.png} 
% \caption[Metodologia XIII: tempo total relativo gasto no cálculo dos
% parâmetros do modelo]{Gráfico com comparativo do tempo total relativo de cálculo
% dos parâmetros dos modelos sem e com GARCH na Metodologia XIII}
% \label{Figura:tempocalculoPizzaMet13}
% \end{figure}

\clearpage

\begin{center}
\begin{longtable}{ccccc|cccc}
\toprule
\rowcolor{white}
\caption[Metodologia XIII: evolução da autocorrelação]{Autocorrelação do dado
original e dos resíduos gerados sem e com a utilização do modelo GARCH na
Metodologia XIII} \label{tab:EvolucaoAutocorrelacaoMet13}\\
\midrule
Conjunto & \specialcell{Autocorrelação\\Inicial} & \specialcell{Autocorrelação\\Sem
GARCH} & \specialcell{Autocorrelação\\Com GARCH} \\
\midrule
\endfirsthead 
%\multicolumn{8}{c}%
%{\tablename\ \thetable\ -- \textit{Continuação da página anterior}} \\
\midrule
\rowcolor{white}
Conjunto & \specialcell{Autocorrelação\\Inicial} & \specialcell{Autocorrelação\\Sem
GARCH} & \specialcell{Autocorrelação\\Com GARCH} \\
\toprule
\endhead
\midrule \\ % \multicolumn{8}{r}{\textit{Continua na próxima página}} \\
\endfoot
\bottomrule 
\endlastfoot
A1    & 6     & 0     & 0 \\
A2    & 5     & 0     & 0 \\
A3    & 6     & 0     & 0 \\
B1    & 6     & 0     & 2 \\
B2    & 6     & 0     & 2 \\
B3    & 6     & 0     & 2 \\
C1    & 2     & 0     & 5 \\
C2    & 1     & 0     & 6 \\
C3    & 2     & 0     & 6 \\
D1    & 2     & 0     & 3 \\
D2    & 2     & 0     & 3 \\
D3    & 2     & 1     & 2 \\
E1    & 4     & 0     & 0 \\
E2    & 4     & 0     & 0 \\
E3    & 4     & 0     & 0 \\
F1    & 1     & 4     & 4 \\
F2    & 6     & 0     & 1 \\
F3    & 6     & 0     & 2 \\
G1    & 1     & 1     & 1 \\
G2    & 2     & 0     & 0 \\
G3    & 6     & 0     & 4 \\
H1    & 1     & 0     & 0 \\
H2    & 1     & 0     & 1 \\
H3    & 1     & 0     & 0 \\
I1    & 7     & 0     & 0 \\
I2    & 1     & 0     & 1 \\
I3    & 1     & 0     & 0 \\
J1    & 3     & 0     & 3 \\
J2    & 3     & 0     & 3 \\
J3    & 8     & 0     & 0 \\
K1    & 3     & 0     & 1 \\
K2    & 3     & 0     & 3 \\
K3    & 2     & 0     & 1 \\
L1    & 2     & 0     & 7 \\
L2    & 7     & 0     & 1 \\
L3    & 11    & 0     & 8 \\
L4    & 7     & 0     & 13 \\
L5    & 7     & 0     & 0 \\
L6    & 6     & 0     & 7 \\

\end{longtable}
\end{center}

% \begin{figure}[!h]
% \centering
% \includegraphics[scale=0.75]{fig/res/evolucaoautocorrMetXIII00.png} 
% \caption[Metodologia XIII: evolução da autocorrelação nos conjuntos A, B e
% C]{Gráfico com comparativo da autocorrelação do resíduo gerado sem e com a
% utilização do modelo GARCH em relação ao dado original nos conjuntos A, B e C na
% Metodologia XIII}
% \label{Figura:autocorrelacaoABCMet13}
% \end{figure}
% 
% \begin{figure}[!h]
% \centering
% \includegraphics[scale=0.69]{fig/res/evolucaoautocorrMetXIII01.png} 
% \caption[Metodologia XIII: evolução da autocorrelação nos conjuntos D, E e
% F]{Gráfico com comparativo da autocorrelação do resíduo gerado sem e com a
% utilização do modelo GARCH em relação ao dado original nos conjuntos D, E e F na
% Metodologia XIII}
% \label{Figura:autocorrelacaoDEFMet13}
% \end{figure}
% 
% \begin{figure}[!h]
% \centering
% \includegraphics[scale=0.69]{fig/res/evolucaoautocorrMetXIII02.png} 
% \caption[Metodologia XIII: evolução da autocorrelação nos conjuntos G, H e
% I]{Gráfico com comparativo da autocorrelação do resíduo gerado sem e com a
% utilização do modelo GARCH em relação ao dado original nos conjuntos G, H e I na
% Metodologia XIII}
% \label{Figura:autocorrelacaoGHIMet13}
% \end{figure}
% 
% \begin{figure}[!h]
% \centering
% \includegraphics[scale=0.69]{fig/res/evolucaoautocorrMetXIII03.png} 
% \caption[Metodologia XIII: evolução da autocorrelação nos conjuntos J e
% K]{Gráfico com comparativo da autocorrelação do resíduo gerado sem e com a
% utilização do modelo GARCH em relação ao dado original nos conjuntos J e K na
% Metodologia XIII}
% \label{Figura:autocorrelacaoJKMet13}
% \end{figure}
% 
% \begin{figure}[!h]
% \centering
% \includegraphics[scale=0.69]{fig/res/evolucaoautocorrMetXIII04.png} 
% \caption[Metodologia XIII: evolução da autocorrelação nos conjuntos L]{Gráfico
% com comparativo da autocorrelação do resíduo gerado sem e com a utilização do modelo GARCH em relação ao dado original nos conjuntos L na
% Metodologia XIII}
% \label{Figura:autocorrelacaoLMet13}
% \end{figure}

% \begin{figure}[!h]
% \centering
% \includegraphics[scale=0.75]{fig/res/evolucaoautocorrMetXIII05.png} 
% \caption[Metodologia XIII: tempo total relativo gasto no cálculo dos
% parâmetros do modelo]{Gráfico com comparativo da redução relativa total da
% autocorrelação do resíduo sem e com a utilização do modelo GARCH na
% Metodologia XIII}
% \label{Figura:tempocalculoPizzaMet13}
% \end{figure}

\clearpage

\begin{center}
\begin{longtable}{ccccccccc}
\toprule
\rowcolor{white}
\caption[Metodologia XIII: dados estatísticos]{Média e variância do dado original
comparadas às do resíduo calculado sem e com a utilização do modelo GARCH na
Metodologia XIII} \label{tab:DadosEstatisticosMet13}\\
\midrule
    Conjunto & \specialcell{Média\\Original} &
    \specialcell{Var.\\Original} & \specialcell{Média\\Sem\\GARCH} &
    \specialcell{Var.\\Sem\\GARCH} & \specialcell{Média\\Com\\GARCH}&
    \specialcell{Var.\\Com\\GARCH} \\

\midrule
\endfirsthead 
%\multicolumn{8}{c}%
%{\tablename\ \thetable\ -- \textit{Continuação da página anterior}} \\
\midrule
\rowcolor{white}
    Conjunto & \specialcell{Média\\Orig.} &
    \specialcell{Var.\\Orig.} & \specialcell{Média\\Sem\\GARCH} &
    \specialcell{Var.\\Sem\\GARCH} & \specialcell{Média\\Com\\GARCH}&
    \specialcell{Var.\\Com\\GARCH} \\

\toprule
\endhead
\midrule \\ % \multicolumn{8}{r}{\textit{Continua na próxima página}} \\
\endfoot
\bottomrule 
\endlastfoot
A1    & 3,0E+04 & 1,8E+07 & -17,9 & 2,4E+05 & -3,1  & 2,5E+05 \\
A2    & 3,2E+04 & 1,1E+07 & 0,4   & 2,2E+05 & -16,8 & 2,3E+05 \\
A3    & 3,1E+04 & 1,4E+07 & -7,7  & 2,6E+05 & 10,2  & 2,5E+05 \\
B1    & 2,8E+04 & 4,5E+05 & 0,5   & 8,0E+00 & 0,4   & 4,5E+01 \\
B2    & 2,8E+04 & 4,5E+05 & 0,5   & 8,0E+00 & 0,4   & 4,5E+01 \\
B3    & 2,8E+04 & 4,5E+05 & 0,5   & 8,0E+00 & 0,4   & 4,5E+01 \\
C1    & 3,3E+04 & 8,1E+07 & -1,9  & 1,3E+07 & -2,4  & 5,4E+07 \\
C2    & 3,3E+04 & 4,0E+07 & 0,2   & 8,7E+06 & 1,7   & 8,2E+07 \\
C3    & 3,3E+04 & 5,7E+07 & 0,5   & 8,4E+06 & 1,1   & 3,2E+07 \\
D1    & 3,7E+04 & 4,1E+07 & 1,1   & 8,8E+05 & 1,2   & 1,5E+06 \\
D2    & 3,3E+04 & 1,2E+07 & 0,2   & 9,9E+05 & 0,7   & 4,6E+06 \\
D3    & 3,1E+04 & 1,0E+07 & 0,8   & 2,3E+05 & 0,2   & 3,0E+05 \\
E1    & 2,9E+04 & 5,8E+07 & -40,3 & 1,1E+07 & -119,4 & 1,1E+07 \\
E2    & 3,0E+04 & 5,8E+07 & -44,2 & 1,1E+07 & -45,8 & 1,2E+07 \\
E3    & 3,0E+04 & 6,0E+07 & 58,4  & 1,1E+07 & -43,7 & 1,2E+07 \\
F1    & 3,8E+04 & 3,9E+07 & 0,5   & 5,6E+06 & 0,4   & 5,6E+06 \\
F2    & 2,3E+04 & 5,4E+06 & 7,4   & 1,5E+05 & 11,6  & 1,8E+05 \\
F3    & 2,6E+04 & 6,0E+06 & -1,3  & 4,9E+04 & -3,5  & 9,6E+04 \\
G1    & 3,3E+04 & 3,3E+07 & 2,9   & 2,6E+06 & 0,1   & 2,6E+06 \\
G2    & 3,8E+04 & 1,9E+07 & -0,9  & 1,2E+04 & 1,3   & 1,2E+04 \\
G3    & 2,9E+04 & 3,6E+07 & -2,7  & 2,9E+04 & -3,7  & 3,5E+05 \\
H1    & 3,1E+04 & 3,6E+07 & -2,7  & 3,5E+07 & 38,2  & 3,5E+07 \\
H2    & 3,4E+04 & 8,1E+06 & 3,4E+04 & 8,1E+06 & 3,4E+04 & 8,1E+06 \\
H3    & 3,2E+04 & 7,3E+06 & 0,6   & 4,9E+06 & 0,1   & 5,1E+06 \\
I1    & 3,6E+04 & 1,2E+07 & 0,6   & 4,9E+05 & 1,9   & 5,1E+05 \\
I2    & 2,9E+04 & 1,2E+06 & 0,4   & 4,5E+04 & -4,9  & 9,8E+04 \\
I3    & 3,1E+04 & 3,3E+07 & 0,5   & 1,7E+06 & 0,5   & 1,7E+06 \\
J1    & 3,7E+04 & 1,2E+06 & 3,7E+04 & 1,2E+06 & 3,7E+04 & 1,2E+06 \\
J2    & 3,5E+04 & 1,5E+06 & 1,2   & 1,3E+04 & 2,0   & 1,8E+04 \\
J3    & 3,3E+04 & 1,3E+06 & 0,5   & 6,5E+03 & 0,5   & 4,9E+04 \\
K1    & 3,9E+04 & 6,9E+06 & 17,8  & 2,1E+05 & 6,6   & 3,6E+05 \\
K2    & 4,0E+04 & 6,7E+06 & 0,2   & 1,8E+05 & 5,2   & 3,8E+05 \\
K3    & 3,6E+04 & 5,8E+06 & -1,0  & 2,0E+05 & 2,9   & 3,4E+05 \\
L1    & 3,4E+04 & 2,9E+07 & 7,2   & 1,8E+06 & 2,0   & 2,4E+06 \\
L2    & 3,1E+04 & 1,5E+07 & -7,3  & 3,5E+06 & 5,0   & 4,1E+06 \\
L3    & 3,5E+04 & 1,3E+07 & -2,6  & 4,0E+06 & 10,8  & 5,2E+06 \\
L4    & 3,7E+04 & 1,8E+07 & 2,4   & 2,4E+06 & 9,8   & 7,1E+06 \\
L5    & 3,1E+04 & 5,1E+07 & 5,5   & 9,6E+05 & 2,9   & 9,8E+05 \\
L6    & 3,2E+04 & 2,6E+07 & 4,5   & 8,5E+05 & -7,0  & 1,1E+06 \\
\end{longtable}
\end{center}

% \begin{figure}[!h]
% \centering
% \includegraphics[scale=0.69]{fig/res/estatisticasMetXIII03.png} 
% \caption[Metodologia XIII: Variância do conjunto A]{Gráfico com
% comparativo da variância original do dado e dos resíduos gerados pelos modelos
% sem e com GARCH do conjunto A na Metodologia XIII}
% \label{Figura:estatisticaAMet13}
% \end{figure}
% 
% \begin{figure}[!h]
% \centering
% \includegraphics[scale=0.69]{fig/res/estatisticasMetXIII00.png} 
% \caption[Metodologia XIII: Variância do conjunto B]{Gráfico com
% comparativo da variância original do dado e dos resíduos gerados pelos modelos
% sem e com GARCH do conjunto B na Metodologia XIII}
% \label{Figura:estatisticaBMet13}
% \end{figure}
% 
% \begin{figure}[!h]
% \centering
% \includegraphics[scale=0.69]{fig/res/estatisticasMetXIII01.png} 
% \caption[Metodologia XIII: Variância do conjunto C]{Gráfico com
% comparativo da variância original do dado e dos resíduos gerados pelos modelos
% sem e com GARCH do conjunto C na Metodologia XIII}
% \label{Figura:estatisticaCMet13}
% \end{figure}
% 
% \begin{figure}[!h]
% \centering
% \includegraphics[scale=0.69]{fig/res/estatisticasMetXIII02.png} 
% \caption[Metodologia XIII: Variância dos conjuntos D e E]{Gráfico com comparativo
% da variância original do dado e dos resíduos gerados pelos modelos sem e com
% GARCH dos conjuntos D e E na Metodologia XIII}
% \label{Figura:estatisticaDEMet13}
% \end{figure}
% 
% \begin{figure}[!h]
% \centering
% \includegraphics[scale=0.69]{fig/res/estatisticasMetXIII04.png} 
% \caption[Metodologia XIII: Variância do conjunto F]{Gráfico com
% comparativo da variância original do dado e dos resíduos gerados pelos modelos
% sem e com GARCH do conjunto F na Metodologia XIII}
% \label{Figura:estatisticaFMet13}
% \end{figure}
% 
% \begin{figure}[!h]
% \centering
% \includegraphics[scale=0.8, angle=90]{fig/res/estatisticasMetXIII05.png} 
% \caption[Metodologia XIII: Variância dos conjuntos G, H e I]{Gráfico com
% comparativo da variância original do dado e dos resíduos gerados pelos modelos
% sem e com GARCH dos conjuntos G, H e I na Metodologia XIII}
% \label{Figura:estatisticaGHIMet13}
% \end{figure}
% 
% \begin{figure}[!h]
% \centering
% \includegraphics[scale=0.8, angle=90]{fig/res/estatisticasMetXIII06.png} 
% \caption[Metodologia XIII: Variância dos conjuntos J e K]{Gráfico
% com comparativo da variância original do dado e dos resíduos gerados pelos modelos
% sem e com GARCH dos conjuntos J e K na Metodologia XIII}
% \label{Figura:estatisticaJKMet13}
% \end{figure}
% 
% \begin{figure}[!h]
% \centering
% \includegraphics[scale=0.8, angle=90]{fig/res/estatisticasMetXIII07.png} 
% \caption[Metodologia XIII: Variância do conjunto L]{Gráfico com
% comparativo da variância original do dado e dos resíduos gerados pelos modelos
% sem e com GARCH do conjunto L na Metodologia XIII}
% \label{Figura:estatisticaLMet13}
% \end{figure}

% \begin{figure}[!h]
% \centering
% \includegraphics[scale=0.65]{fig/res/estatisticasMetXIII08.png} 
% \caption[Metodologia XIII: redução relativa da variância]{Gráfico com comparativo
% da redução relativa total da variância do resíduo sem e com a utilização do modelo GARCH na
% Metodologia XIII}
% \label{Figura:estatisticaPizzaMet13}
% \end{figure}