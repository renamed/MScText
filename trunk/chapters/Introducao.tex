\chapter{Introdução}
\pagenumbering{arabic} 

A compressão de dados sísmicos vem sendo estudada há
decadas~\citep{Artigo:bordley1983}. Tal duração pode ser explicada pela grande
quantidade de dados gerados durante uma aquisição sísmica, pela necessidade de
replicação deles e pela geração de novos conjuntos de dados depois do
processamento dos dados coletados.

Muitos trabalhos utilizam modelos de predição linear como uma
transformada matemática em dados sísmicos para facilitar a sua
compressão~\citep{Artigo:bordley1983, Artigo:linearmelhor, Artigo:lpemaistres,
Artigo:nenhumamudanca}.
Dados sísmicos, entretanto, não são lineares. Além disso, dados sísmicos são
heteroscedásticos, ou seja, sua variância não pode ser considerada constante ao
longo do tempo. Embora tenhamos descoberto trabalhos que utilizam, com sucesso,
modelos não lineares para comprimir dados
sísmicos~\citep{Artigo:naolinearestat}, não fomos capazes de encontrar um que
lide com sua heteroscedasticidade. Neste trabalho, propomos estudar o modelo de
predição ARIMA-GARCH a fim de obtermos um conjunto de resíduos mais propícios
para compressão.

Nossa hipótese é que um modelo capaz de explorar a heteroscedasticidade do dado
sísmico apresente um conjunto de resíduos mais decorrelacionados, facilitando a
compressão dele.

O modelo ARIMA-GARCH será aplicado no esquema de compressão em duas
etapas~\citep{Artigo:stearnsincompleto}, onde a primeira tem por objetivo
reduzir a entropia do dado. A entropia é um limite teórico e representa a média
mínima de bits necessária para representar uma informação sem perdas. A segunda
etapa comprime o dado efetivamente.

Visualizando um dado sísmico como uma série temporal, é possível calcular os
coeficientes de um modelo ARIMA-GARCH. O resíduo dá-se pela diferença entre o
valor calculado pelo modelo e o valor real do dado. É possível provar
matematicamente que o conjunto de resíduos gerado é
decorrelacionado~\citep{Livro:analiseseriestemporais}, abrindo caminho para um
resultado de compressão melhor. De posse dos coeficientes calculados e do
conjunto de resíduos, o decodificador pode, sem perdas, regerar o dado original.

Neste trabalho, comparando a solução proposta com um modelo de
predição linear, verificou-se que o modelo proposto foi capaz de melhorar a
razão de compressão na maioria dos testes rodados, demorou mais tempo para
executar também na maioria dos testes, não foi capaz de reduzir a autocorrelação
e a variância tão bem como o modelo de predição linear, mas foi superior na
redução da entropia do resíduo gerado.

O trabalho encontra-se organizado da seguinte forma: no
Capítulo~\ref{Capitulo:sismicadereflexao} é explicado como dados sísmicos são
coletados e apresenta algumas nomenclaturas. O
Capítulo~\ref{chap:compressaodados} introduz os principais conceitos sobre
compressão de dados. O Capítulo~\ref{Capitulo:analisedeseriestemporais}
apresenta o modelo ARIMA-GARCH, bem como alguns conceitos sobre séries
temporais. O Capítulo~\ref{Capitulo:propostatrabalho} formaliza o estudo feito
nesse trabalho. O Capítulo~\ref{Capitulo:testes} apresenta
metodologias e resultados de testes e, finalmente, o
Capítulo~\ref{Cap:conclusao} apresenta a conclusão desse trabalho.
